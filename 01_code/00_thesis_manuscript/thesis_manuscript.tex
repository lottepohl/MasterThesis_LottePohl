% Options for packages loaded elsewhere
\PassOptionsToPackage{unicode}{hyperref}
\PassOptionsToPackage{hyphens}{url}
\PassOptionsToPackage{dvipsnames,svgnames,x11names}{xcolor}
%
\documentclass[
  authoryear,
  review,
  3p]{elsarticle}

\usepackage{amsmath,amssymb}
\usepackage{iftex}
\ifPDFTeX
  \usepackage[T1]{fontenc}
  \usepackage[utf8]{inputenc}
  \usepackage{textcomp} % provide euro and other symbols
\else % if luatex or xetex
  \usepackage{unicode-math}
  \defaultfontfeatures{Scale=MatchLowercase}
  \defaultfontfeatures[\rmfamily]{Ligatures=TeX,Scale=1}
\fi
\usepackage{lmodern}
\ifPDFTeX\else  
    % xetex/luatex font selection
\fi
% Use upquote if available, for straight quotes in verbatim environments
\IfFileExists{upquote.sty}{\usepackage{upquote}}{}
\IfFileExists{microtype.sty}{% use microtype if available
  \usepackage[]{microtype}
  \UseMicrotypeSet[protrusion]{basicmath} % disable protrusion for tt fonts
}{}
\makeatletter
\@ifundefined{KOMAClassName}{% if non-KOMA class
  \IfFileExists{parskip.sty}{%
    \usepackage{parskip}
  }{% else
    \setlength{\parindent}{0pt}
    \setlength{\parskip}{6pt plus 2pt minus 1pt}}
}{% if KOMA class
  \KOMAoptions{parskip=half}}
\makeatother
\usepackage{xcolor}
\setlength{\emergencystretch}{3em} % prevent overfull lines
\setcounter{secnumdepth}{5}
% Make \paragraph and \subparagraph free-standing
\ifx\paragraph\undefined\else
  \let\oldparagraph\paragraph
  \renewcommand{\paragraph}[1]{\oldparagraph{#1}\mbox{}}
\fi
\ifx\subparagraph\undefined\else
  \let\oldsubparagraph\subparagraph
  \renewcommand{\subparagraph}[1]{\oldsubparagraph{#1}\mbox{}}
\fi


\providecommand{\tightlist}{%
  \setlength{\itemsep}{0pt}\setlength{\parskip}{0pt}}\usepackage{longtable,booktabs,array}
\usepackage{calc} % for calculating minipage widths
% Correct order of tables after \paragraph or \subparagraph
\usepackage{etoolbox}
\makeatletter
\patchcmd\longtable{\par}{\if@noskipsec\mbox{}\fi\par}{}{}
\makeatother
% Allow footnotes in longtable head/foot
\IfFileExists{footnotehyper.sty}{\usepackage{footnotehyper}}{\usepackage{footnote}}
\makesavenoteenv{longtable}
\usepackage{graphicx}
\makeatletter
\def\maxwidth{\ifdim\Gin@nat@width>\linewidth\linewidth\else\Gin@nat@width\fi}
\def\maxheight{\ifdim\Gin@nat@height>\textheight\textheight\else\Gin@nat@height\fi}
\makeatother
% Scale images if necessary, so that they will not overflow the page
% margins by default, and it is still possible to overwrite the defaults
% using explicit options in \includegraphics[width, height, ...]{}
\setkeys{Gin}{width=\maxwidth,height=\maxheight,keepaspectratio}
% Set default figure placement to htbp
\makeatletter
\def\fps@figure{htbp}
\makeatother

\usepackage{booktabs}
\usepackage{longtable}
\usepackage{array}
\usepackage{multirow}
\usepackage{wrapfig}
\usepackage{float}
\usepackage{colortbl}
\usepackage{pdflscape}
\usepackage{tabu}
\usepackage{threeparttable}
\usepackage{threeparttablex}
\usepackage[normalem]{ulem}
\usepackage{makecell}
\usepackage{xcolor}
\newcommand{\mycommand}[1]{\textbf{#1}}
\makeatletter
\makeatother
\makeatletter
\makeatother
\makeatletter
\@ifpackageloaded{caption}{}{\usepackage{caption}}
\AtBeginDocument{%
\ifdefined\contentsname
  \renewcommand*\contentsname{Table of contents}
\else
  \newcommand\contentsname{Table of contents}
\fi
\ifdefined\listfigurename
  \renewcommand*\listfigurename{List of Figures}
\else
  \newcommand\listfigurename{List of Figures}
\fi
\ifdefined\listtablename
  \renewcommand*\listtablename{List of Tables}
\else
  \newcommand\listtablename{List of Tables}
\fi
\ifdefined\figurename
  \renewcommand*\figurename{Figure}
\else
  \newcommand\figurename{Figure}
\fi
\ifdefined\tablename
  \renewcommand*\tablename{Table}
\else
  \newcommand\tablename{Table}
\fi
}
\@ifpackageloaded{float}{}{\usepackage{float}}
\floatstyle{ruled}
\@ifundefined{c@chapter}{\newfloat{codelisting}{h}{lop}}{\newfloat{codelisting}{h}{lop}[chapter]}
\floatname{codelisting}{Listing}
\newcommand*\listoflistings{\listof{codelisting}{List of Listings}}
\makeatother
\makeatletter
\@ifpackageloaded{caption}{}{\usepackage{caption}}
\@ifpackageloaded{subcaption}{}{\usepackage{subcaption}}
\makeatother
\makeatletter
\@ifpackageloaded{tcolorbox}{}{\usepackage[skins,breakable]{tcolorbox}}
\makeatother
\makeatletter
\@ifundefined{shadecolor}{\definecolor{shadecolor}{rgb}{.97, .97, .97}}
\makeatother
\makeatletter
\makeatother
\makeatletter
\makeatother
\journal{Journal of Animal Biotelemetry}
\ifLuaTeX
  \usepackage{selnolig}  % disable illegal ligatures
\fi
\usepackage[]{natbib}
\bibliographystyle{elsarticle-harv}
\IfFileExists{bookmark.sty}{\usepackage{bookmark}}{\usepackage{hyperref}}
\IfFileExists{xurl.sty}{\usepackage{xurl}}{} % add URL line breaks if available
\urlstyle{same} % disable monospaced font for URLs
\hypersetup{
  pdftitle={Vertical movement behaviour of the Starry smooth-hound shark Mustelus asterias in the North Sea},
  pdfauthor={Lotte Pohl; Niels Brevé; Carlota Muñiz; Jan Reubens},
  pdfkeywords={acoustic telemetry, geolocation modelling, Mustelus
asterias},
  colorlinks=true,
  linkcolor={blue},
  filecolor={Maroon},
  citecolor={Blue},
  urlcolor={Blue},
  pdfcreator={LaTeX via pandoc}}

\setlength{\parindent}{6pt}
\begin{document}

\begin{frontmatter}
\title{Vertical movement behaviour of the Starry smooth-hound shark
\emph{Mustelus asterias} in the North Sea \\\large{Master Thesis} }
\author[1,2]{Lotte Pohl%
\corref{cor1}%
}
 \ead{lotte.pohl@imbrsea.eu} 
\author[3,4]{Niels Brevé%
%
}
 \ead{breve@sportvisserijnederland.nl} 
\author[1]{Carlota Muñiz%
%
}
 \ead{carlota.muniz@vliz.be} 
\author[1]{Jan Reubens%
%
}
 \ead{jan.reubens@vliz.be} 

\affiliation[1]{organization={Flemish Marine Institute, Marine
Observation Centre},addressline={Slipwaykaai
2},city={Ostend},postcode={8400},postcodesep={}}
\affiliation[2]{organization={Ghent University, Marine Biology research
group},addressline={Krijgslaan
281/S8},city={Ghent},postcode={9000},postcodesep={}}
\affiliation[3]{organization={Wageningen University and Research, Marine
Ecology Group},addressline={Droevendaalsesteeg
1},city={Wageningen},postcode={6700},postcodesep={}}
\affiliation[4]{organization={Sportvisserij
Nederlands},addressline={Leyenseweg
115},city={Bilthoven},postcode={3721},postcodesep={}}

\cortext[cor1]{Corresponding author}




        
\begin{abstract}
tempor incididunt ut labore et dolore magna aliqua.ffffffff
ffffffffffffffffffffff fffffffffffffffffff ffffffffffffffffffffffffff
fffffffffffffffffffffff fffffffffffffffffffff ffffffffff
ffffffffffffffff ffffffffffffffff fffffffffffffffffff
fffffffffffffffffffffffff fffffffffffffff ffffffffffffffffff
fffffffffffffff fffffffffffffffffffffff.tempor incididunt ut labore et
dolore magna aliqua.ffffff ffffffffffffffffffffffff fffffff
ffffffffffffffff ffffffffffffffffff fffff ffffffffffffffffff
fffffffffffffffffffff ffffffffffffffffff fffffffffffffffff
fffffffffffffffffffff ffffffffffffffff ffffffffffffff
ffffffffffffffffffffffff fffffffffffffffffffffffffffffffff
ffffffffffffffffftempor incididunt ut labore et dolore magna
aliqua.fffffffffffff fffffffffffff ffffffffff fffffffff
ffffffffffffffffffff fffffffff fffffffffffffffffffffff
fffffffffffffffffff fffffffff ffffffffffffffff ffffffff fffffff f
fffffffffffffff ffffff fffffffffffffffffff fffffffffffffff
ffffffffffffffffffff fffffffffffff fffffffffffff
fffffffffffffffffftempor incididunt ut labore et dolore magna
aliqua.fffffffffffff ffffffffffffffffffff ffffffffffffffffffffff
ffffffffffffffffff fffffffffffff ffffffffffffffffff fffffffffffffffff
ffffffffffffffffffffffffffffff fffffffffffffff ffffffffff
fffffffffffffff fffffffffffffffffff fffffffffffff
fffffffffffffffffffffffff fffffffffffffffffffffffffffftempor incididunt
ut labore et dolore magna aliqua.fffffffffffffff
ffffffffffffffffffffffff ffffffffffffffff fffffffffffffffffffffff
fffffffffffffffff ffffffffffffffffff fffffffffffffffffff ffffffffffffff
fffff ffffffffffffffffffff fffff fffffffffffffffffffffffffffffff
fffffffffffffffffff fffffffffffffffffffffffffffffffff
fffffffffffffffftempor incididunt ut labore et dolore magna aliqua
tempor incididunt ut labore et dolore magna aliqua.ffffffff
ffffffffffffffffffffff fffffffffffffffffff ffffffffffffffffffffffffff
fffffffffffffffffffffff fffffffffffffffffffff ffffffffff
ffffffffffffffff ffffffffffffffff fffffffffffffffffff
fffffffffffffffffffffffff fffffffffffffff ffffffffffffffffff
fffffffffffffff fffffffffffffffffffffff.tempor incididunt ut labore et
dolore magna aliqua.ffffff ffffffffffffffffffffffff fffffff
ffffffffffffffff ffffffffffffffffff fffff ffffffffffffffffff
fffffffffffffffffffff ffffffffffffffffff fffffffffffffffff
fffffffffffffffffffff ffffffffffffffff ffffffffffffff
ffffffffffffffffffffffff fffffffffffffffffffffffffffffffff
ffffffffffffffffftempor incididunt ut labore et dolore magna
aliqua.fffffffffffff fffffffffffff ffffffffff fffffffff
ffffffffffffffffffff fffffffff fffffffffffffffffffffff
fffffffffffffffffff fffffffff ffffffffffffffff ffffffff fffffff f
fffffffffffffff ffffff fffffffffffffffffff fffffffffffffff
ffffffffffffffffffff fffffffffffff fffffffffffff
fffffffffffffffffftempor incididunt ut labore et dolore magna
aliqua.fffffffffffff ffffffffffffffffffff ffffffffffffffffffffff
ffffffffffffffffff fffffffffffff ffffffffffffffffff fffffffffffffffff
ffffffffffffffffffffffffffffff fffffffffffffff ffffffffff
fffffffffffffff fffffffffffffffffff fffffffffffff
fffffffffffffffffffffffff fffffffffffffffffffffffffffftempor incididunt
ut labore et dolore magna aliqua.fffffffffffffff
ffffffffffffffffffffffff ffffffffffffffff fffffffffffffffffffffff
fffffffffffffffff ffffffffffffffffff fffffffffffffffffff ffffffffffffff
fffff ffffffffffffffffffff fffff fffffffffffffffffffffffffffffff
fffffffffffffffffff fffffffffffffffffffffffffffffffff
fffffffffffffffftempor incididunt ut labore et dolore magna aliqua
tempor incididunt ut labore et dolore magna aliqua.ffffffff
ffffffffffffffffffffff fffffffffffffffffff ffffffffffffffffffffffffff
fffffffffffffffffffffff fffffffffffffffffffff ffffffffff
ffffffffffffffff ffffffffffffffff fffffffffffffffffff
fffffffffffffffffffffffff fffffffffffffff ffffffffffffffffff
fffffffffffffff fffffffffffffffffffffff.tempor incididunt ut labore et
dolore magna aliqua.ffffff ffffffffffffffffffffffff fffffff
ffffffffffffffff ffffffffffffffffff fffff ffffffffffffffffff
fffffffffffffffffffff ffffffffffffffffff fffffffffffffffff
fffffffffffffffffffff ffffffffffffffff ffffffffffffff
ffffffffffffffffffffffff fffffffffffffffffffffffffffffffff
ffffffffffffffffftempor incididunt ut labore et dolore magna
aliqua.fffffffffffff fffffffffffff ffffffffff fffffffff
ffffffffffffffffffff fffffffff fffffffffffffffffffffff
fffffffffffffffffff fffffffff ffffffffffffffff ffffffff fffffff f
fffffffffffffff ffffff fffffffffffffffffff fffffffffffffff
ffffffffffffffffffff fffffffffffff fffffffffffff
fffffffffffffffffftempor incididunt ut labore et dolore magna
aliqua.fffffffffffff ffffffffffffffffffff ffffffffffffffffffffff
ffffffffffffffffff fffffffffffff ffffffffffffffffff fffffffffffffffff
ffffffffffffffffffffffffffffff fffffffffffffff ffffffffff
fffffffffffffff fffffffffffffffffff fffffffffffff
fffffffffffffffffffffffff fffffffffffffffffffffffffffftempor incididunt
ut labore et dolore magna aliqua.fffffffffffffff
ffffffffffffffffffffffff ffffffffffffffff fffffffffffffffffffffff
fffffffffffffffff ffffffffffffffffff fffffffffffffffffff ffffffffffffff
fffff ffffffffffffffffffff fffff fffffffffffffffffffffffffffffff
fffffffffffffffffff fffffffffffffffffffffffffffffffff
fffffffffffffffftempor incididunt ut labore et dolore magna aliqua
tempor incididunt ut labore et dolore magna aliqua.ffffffff
ffffffffffffffffffffff fffffffffffffffffff ffffffffffffffffffffffffff
fffffffffffffffffffffff fffffffffffffffffffff ffffffffff
ffffffffffffffff ffffffffffffffff fffffffffffffffffff
fffffffffffffffffffffffff fffffffffffffff ffffffffffffffffff
fffffffffffffff fffffffffffffffffffffff.tempor incididunt ut labore et
dolore magna aliqua.ffffff ffffffffffffffffffffffff fffffff
ffffffffffffffff ffffffffffffffffff fffff ffffffffffffffffff
fffffffffffffffffffff ffffffffffffffffff fffffffffffffffff
fffffffffffffffffffff ffffffffffffffff ffffffffffffff
ffffffffffffffffffffffff fffffffffffffffffffffffffffffffff
fffffffffffffffff fffffffffffffffffffffff.tempor incididunt ut labore et
dolore magna aliqua.ffffff ffffffffffffffffffffffff fffffff
ffffffffffffffff ffffffffffffffffff fffff ffffffffffffffffff
fffffffffffffffffffff ffffffffffffffffff fffffffffffffffff
fffffffffffffffffffff ffffffffffffffff ffffffffffffff
ffffffffffffffffffffffff fffffffffffffffffffffffffffffffff
fffffffffffffffff fffffffffffffffffffffff.tempor incididunt ut labore et
dolore magna aliqua.ffffff ffffffffffffffffffffffff fffffff
ffffffffffffffff ffffffffffffffffff fffff ffffffffffffffffff
fffffffffffffffffffff ffffffffffffffffff fffffffffffffffff
fffffffffffffffffffff ffffffffffffffff ffffffffffffff
ffffffffffffffffffffffff fffffffffffffffffffffffffffffffff
fffffffffffffffff tempor incididunt ut labore et dolore magna
aliqua.ffffff ffffffffffffffffffffffff fffffff ffffffffffffffff
ffffffffffffffffff fffff ffffffffffffffffff fffffffffffffffffffff
ffffffffffffffffff fffffffffffffffff fffffffffffffffffffff
ffffffffffffffff ffffffffffffff ffffffffffffffffffffffff
fffffffffffffffffffffffffffffffff fffffffffffffffff
fffffffffffffffffffffff.tempor incididunt ut labore et dolore magna
aliqua.
\end{abstract}





\begin{keyword}
    acoustic telemetry \sep geolocation modelling \sep 
    Mustelus asterias
\end{keyword}
\end{frontmatter}
    \ifdefined\Shaded\renewenvironment{Shaded}{\begin{tcolorbox}[enhanced, boxrule=0pt, sharp corners, frame hidden, interior hidden, borderline west={3pt}{0pt}{shadecolor}, breakable]}{\end{tcolorbox}}\fi

\renewcommand*\contentsname{Table of contents}
{
\hypersetup{linkcolor=}
\setcounter{tocdepth}{3}
\tableofcontents
}
\newpage{}

\hypertarget{list-of-abbreviations}{%
\section*{List of Abbreviations}\label{list-of-abbreviations}}
\addcontentsline{toc}{section}{List of Abbreviations}

\begin{longtable}[]{@{}ll@{}}
\toprule\noalign{}
Abbreviation & Explanation \\
\midrule\noalign{}
\endhead
\bottomrule\noalign{}
\endlastfoot
ADST & Acoustic Data Storage Tag \\
BPNS & Belgian Part of the North Sea \\
CWT & Continuous Wavelet Transform \\
DST & Data Storage Tag \\
ETN & European Tracking Network \\
FFT & Fast Fourier Transform \\
FT & Fourier Transformation \\
HMM & Hidden Markov Model \\
ICES & International Council for the Exploration of the Sea \\
PBARN & Permanent Belgian Receiver Network \\
RQ & Research Question \\
TL & Total Length \\
UTC & Coordinated Universal Time \\
f & female \\
m & male \\
\end{longtable}

\newpage{}

\hypertarget{executive-summary}{%
\section*{Executive Summary}\label{executive-summary}}
\addcontentsline{toc}{section}{Executive Summary}

The Starry smooth-hound \emph{Mustelus asterias} (Cloquet, 1819) is a
widely distributed demersal shark in the Northeast Atlantic, yet under
increasing fishing pressure. To lay the grounds for future species
management plans, information on its complex annual migration and
residency behaviour is needed.

The movement of fish is commonly studied with electronic tags. A novel
technology is the Acoustic Data Storage Tag (ADST), which can be
detected by acoustic receivers and additionally logs temperature and
water depth in predefined intervals (access requires the tag's
retrieval). This study employed ADST to characterise the migration and
residency of \emph{M. asterias} in the North Sea regarding seasonality
and sex. In 2018 and 2019, 30 adult \emph{M. asterias} (19 females, 11
males) were equipped with ADST-V13TP (Innovasea, 518 days estimated
battery life) in the Scheldt estuary. The Permanent Belgian Receiver
Network (PBARN) encompasses 160 active receivers distributed across the
Belgian Part of the North Sea (BPNS) and the Scheldt estuary.
Exploratory and spectral analyses were conducted on a July 2018 to July
2020 ADST dataset.

14 females and 4 males were detected 10940 times by 27 acoustic
receivers. All detections were between April and November, indicating
seasonal presence of \emph{M. asterias} in the receiver area. 96 \% of
detections were in the outer Western Scheldt, and 97 \% of detections
were from females. This suggests the outer Western Scheldt to be a
relevant habitat for females between April and November. 8 ADST were
recovered, one female and one male logging data for over a year. Both
tags logged depths between 0 and 20 m from June to October, and between
10 and 75 m from October to June. The changing depth range reflects the
migration of the sharks into deeper waters during winter. Continuous
Wavelet Transform (CWT) resolves periodic patterns in the time domain
and displays differences in periodicities between seasons and
individuals. The CWT results suggest that the female is resting during
summer and feeding in a biweekly pattern in winter, and that the male
feeds continuously.

This study provides first details about seasonality in spatiotemporal
movement behaviour of \emph{M. asterias} and confirms its seasonal
presence in the Western Scheldt. Data on the vertical movement of more
individuals is needed to determine if the patterns observed here are
related to sex. The findings on the spatiotemporal presence of \emph{M.
asterias} in the BPNS and the Scheldt estuary establish the primary
groundwork for future species management plans.

\newpage{}

\hypertarget{abstract}{%
\section*{Abstract}\label{abstract}}
\addcontentsline{toc}{section}{Abstract}

The Starry smooth-hound \emph{Mustelus asterias} is widely distributed
in the Northeast Atlantic, yet under increasing fishing pressure. To lay
the grounds for future species management plans, information on its
complex annual migration and residency behaviour is needed. Fish
movement is commonly studied with electronic tags, one novel technology
being the Acoustic Data Storage Tag (ADST). It can be detected by
acoustic receivers and additionally logs temperature and water depth in
predefined intervals. This study characterises the migration and
residency of \emph{M. asterias} in the North Sea regarding season and
sex. 30 \emph{M. asterias} were equipped with ADST in the Scheldt
estuary in 2018 and 2019. Acoustic detections of 18 individuals in the
Western Scheldt estuary between July 2018 and July 2020 suggest seasonal
presence of females just outside of the Western Scheldt between April
and November. Spectral analysis of the depth logs of two individuals
logging for over a year indicates feeding behaviour of the male
throughout most of the year. The female shark presumably rests on the
seafloor during summer and feeds in a biweekly rhythm during winter.
This study presents the first detailed insights into vertical movement
behaviour differences between seasons and individuals of \emph{M.
asterias}.

\textbf{Keywords}: acoustic telemetry, \emph{Mustelus asterias}, data
storage tag, electronic tagging, North Sea, movement ecology

\newpage{}

\hypertarget{sec-intro}{%
\section{Introduction}\label{sec-intro}}

Gathering knowledge on non-commercial, migratory fish species with
complex life cycles is challenging, yet necessary to improve management
of such species in the future \citep{brownscombe_2022}. Fish with late
reproduction and low offspring numbers such as elasmobranchs are
especially vulnerable to overfishing and require special attention in
fisheries management \citep{stevens_2000}.

An elasmobranch with a complex life cycle involving biannual
reproduction and a sex-specific annual migration is the starry
smooth-hound \emph{Mustelus asterias} (Cloquet, 1819), a demersal
triakid shark widely distributed in the Northeast Atlantic and
Mediterranean Sea \citep{breve_2016, griffiths_2020}. In the past
decades, both commercial landings
\citep{ices_2022, bitonporsmoguer_2022} and abundance in the North Sea
\citep{mccullyphillips_2015} have increased. The International Council
for the Exploration of the Sea (ICES) classified the species to be
\emph{Near Threatened} \citep{ices_2022}. Meanwhile, knowledge on the
seasonal presence of \emph{M. asterias} in the North Sea, especially in
Belgian Part of the North Sea (BPNS) and the estuary of the Western
Scheldt is missing. Tracking the movement and presence of migratory
aquatic species in space and time, requires advanced technologies which
have improved substantially in recent years \citep{whoriskey_2019}.

\hypertarget{aquatic-telemetry}{%
\subsection{Aquatic telemetry}\label{aquatic-telemetry}}

Telemetry refers to the remote monitoring of animals using electronic
tags and receivers \citep{whoriskey_2019}, allowing for the tracking of
marine animal movement. A multitude of telemetry methods exists today,
and popular technologies to study aquatic animals include acoustic
telemetry and archival telemetry \citep{thorstad_2013}. Telemetry
furthermore enables the quantification of animal migrations and both
vertical and horizontal habitat use \citep{hussey_2015}, which informs
species management and conservation spanning marine, freshwater and
terrestrial ecosystems \citep{beger_2010}.

In acoustic telemetry, animals are internally or externally equipped
with electronic tags that emit acoustic signals unique to each tag.
Acoustic receivers are able to detect these acoustic signals within a
certain range of the device and identify the individual tag. Such
networks exist primarily in shallow, coastal areas \citep{hussey_2015}
such as the Acoustic Tracking Array Platform (ATAP) in South Africa
\citep{cowley_2017} and the Permanent Belgian Acoustic Receiver Network
\citep[PBARN,][]{reubens_2019}. Archival telemetry uses Data Storage
Tags (DST) that record and locally store environmental variables such as
water pressure and temperature. DST require the retrieval of the tag for
downloading data \citep{thorstad_2013}.

Acoustic data storage tags (ADST) are a combination tag, merging
acoustic and data storage tags \citep{goossens_2023}. As such, the tag
emits unique acoustic signals that can be detected by an acoustic
receiver within a certain range, and additionally records water pressure
and temperature in predefined intervals. Per acoustic detection, one
value of either temperature or water depth gets transmitted. ADST gather
data on horizontal fish movement inside acoustic receiver networks while
simultaneously collecting fine-scale vertical movement data that can be
used to estimate the fish's movement beyond acoustic receiver arrays.

Different telemetry methods result in different datasets. In acoustic
telemetry, recorded data contains the time that a tag was detected by a
specific receiver. These datasets come in a horizontal dimension, i.e.,
as the latitude and longitude position of the acoustic receiver.
Archival telemetrys generates regular time series in pre-defined
intervals, logging water depth and other environmental variables. These
data come in a vertical dimension, spanning the range of the animal's
depth use. Datasets resulting from ADST contain time stamps of the tag's
detection at a specific acoustic receiver, and, in case the tag was
recovered, logs of water depth and other environmental variables.

To determine the approximate location (i.e., latitude and longitude), of
a DST log (which represents the vertical dimension), stochastic
geolocation models are commonly used \citep{gatti_2021}. These models
involve bathymetry and environmental factors such as water temperature
to estimate the trajectory followed by the fish \citep{nielsen_2004}.
Common geolocation modelling approaches use Hidden Markov Models
\citep[HMM,][]{pedersen_2008, woillez_2016}. For certain species with
high economic value (such as Atlantic cod, \emph{Gadus morhua}, and
European seabass, \emph{Dicentrarchus labrax}), those models have been
improved by including different behavioural states of the species
\citep{pedersen_2008, heerah_2017}. Behavioural states (such as low or
high activity) are driven by the animal's environment and internal
ecophysiological processes \citep{gurarie_2016}. Geolocation modelling
has been employed for \emph{M. asterias} with the implementation of
behavioural states \citep[following the approach of][]{pedersen_2008}
and without \citep[following the approach of
\citet{woillez_2016}]{goossens_2023}. Geolocation model outputs for one
female and one male \emph{M. asterias} are shown in
Figure~\ref{fig-mapgeolocation}.

\begin{figure}

\begin{minipage}[t]{\linewidth}

{\centering 

\raisebox{-\height}{

\includegraphics{thesis_manuscript_files/figure-pdf/fig-mapgeolocation-1.pdf}

}

}

\subcaption{\label{fig-mapgeolocation-1}Female shark, tag 308. Adapted
from \citet{goossens_2023}.}
\end{minipage}%
\newline
\begin{minipage}[t]{\linewidth}

{\centering 

\raisebox{-\height}{

\includegraphics{thesis_manuscript_files/figure-pdf/fig-mapgeolocation-2.pdf}

}

}

\subcaption{\label{fig-mapgeolocation-2}Male shark, tag 321.}
\end{minipage}%

\caption{\label{fig-mapgeolocation}Results of geolocation modelling
(without behavioural states) of two DST logging depth for over a year.
The estimated trajectory of a female shark is shown in panel
\textbf{a)}, the estimated trajectory of a male shark is shown in panel
\textbf{b)}. The colour of the trajectories corresponds to the date.
Tagging locations are marked in yellow.}

\end{figure}

\hypertarget{the-starry-smooth-hound-shark-mustelus-asterias}{%
\subsection{\texorpdfstring{The Starry smooth-hound shark \emph{Mustelus
asterias}}{The Starry smooth-hound shark Mustelus asterias}}\label{the-starry-smooth-hound-shark-mustelus-asterias}}

The Starry smooth-hound shark (\emph{Mustelus asterias}, family:
Triakidae) inhabits waters of the Northeast Atlantic and the
Mediterranean Sea, spanning from the Irish Sea and Northern North Sea in
the North to an unknown limit in the south \citep{ices_2022}. This
species typically measures around 30 cm in Total Length (TL) at birth
and can grow up to 140 cm TL \citep{mccullyphillips_2015}. Sexual
maturity is attained at 100 cm TL for males and 120 cm TL for females
\citep{farrell_2010}, which usually occurs at 4-5 years for males and 6
years for females. The average lifespan of Starry smooth-hound sharks is
around 13 years for males and 18 years for females \citep{farrell_2010}.

\newpage{}

The species is matrotrophic aplacentally viviparous \citep[i.e., embryos
absorb nutrients from a yolk sack that is used up during
gestation,][]{farrell_2010a, mccullyphillips_2015}. \citet{farrell_2014}
found \emph{M. asterias} to display polyandry (i.e., one female mating
with several males within one breeding season) and multiple paternity
(i.e., pups within one litter originate from several males). Moreover,
it has a biannual reproduction cycle with a gestation period of about 12
months, followed by a 12-month resting period. \citet{farrell_2010a}
found that M. asterias females can store sperm for up to 12 months and
exhibit embryo asynchronism (i.e., carrying embryos at different
developmental stages). \citet{breve_2016} and \citet{griffiths_2020}
reported philopatric behaviour (i.e., females returning to the same area
for parturition). Identified parturition grounds for \emph{M. asterias}
include the estuary of the Eastern Scheldt and, more generally, the
Bristol Channel, the Southern North Sea and the English Channel
\citep{dureuil_2013, mccullyphillips_2015, breve_2016}. Pupping
presumably takes place between April and September
\citep{farrell_2010a, mccullyphillips_2015}.

\emph{M. asterias} is morphologically adapted to a demersal life style,
exhibiting a ventrally located snout that enables it to feed on prey on
the seafloor. It is a specialist feeder on benthic and suprabenthic
crustaceans \citep{mccullyphillips_2020, bitonporsmoguer_2022}. Common
prey include the hermit crab (\emph{Pagurus bernhardus}), the flying
crab (\emph{Liocarcinus holsatus}), the common shore crab
(\emph{Carcinus maenas}) and the edible crab (\emph{Cancer pagurus})
\citep{mccullyphillips_2015, mccullyphillips_2020}. Potential predators
of \emph{M. asterias} are the grey seal (\emph{Halichoerus grypus})
which occurs in the North Sea, the blue shark (\emph{Prionace glauca})
and the common dolphin (\emph{Delphinus delphis}), both of which occur
in the Western English Channel \citep{griffiths_2020}.

\emph{M. asterias} shows higher vertical activity during night than
during the day \citep{griffiths_2020}. This indicates foraging and
preying behaviour at night, a commonly observed behaviour in
elasmobranchs \citep{sims_2006}. During the day, \emph{M. asterias}
potentially avoids predation by resting on the seafloor
\citep{griffiths_2020}. The species displays annual, and potentially
sex-specific migration. Females are thought to move into the English
Channel and the Bay of Biscay in winter, and males supposedly move into
the Northern North Sea and the English Channel
\citep{breve_2016, breve_2020, griffiths_2020}. seasonal grounds.

\hypertarget{sec-intro-aims}{%
\subsection{Aims of this work}\label{sec-intro-aims}}

The seasonal presence of \emph{M. asterias} in the Western Scheldt
estuary, a presumed summer habitat of the species, has not been studied
yet. The confirmation of seasonal presence of \emph{M. asterias},
however, would be relevant to species management, possibly leading to
stock assessments or seasonal restrictions for commercial fishing in the
future \citep{benot_2003}. Moreover, detailed information on potential
differences in vertical movement regarding seasonality and sex,
including the timing and duration of migrations, are currently missing.
Such information can improve geolocation models through the
implementation of behavioural states \citep{pedersen_2008}. With the aim
of addressing these knowledge gaps, this study pursues two lines of
research.

\hypertarget{presence-of-m.-asterias-in-the-western-scheldt-and-the-bpns-research-line-1}{%
\subsubsection*{\texorpdfstring{Presence of \emph{M. asterias} in the
Western Scheldt and the BPNS (Research Line
1)}{Presence of M. asterias in the Western Scheldt and the BPNS (Research Line 1)}}\label{presence-of-m.-asterias-in-the-western-scheldt-and-the-bpns-research-line-1}}
\addcontentsline{toc}{subsubsection}{Presence of \emph{M. asterias} in
the Western Scheldt and the BPNS (Research Line 1)}

\emph{M. asterias} was rarely caught at the Dutch and Belgian North Sea
coast before 2009 \citep{breve_2016}. These waters are influenced by the
outflows of the rivers Rhine, Meuse and Scheldt. \emph{M. asterias} is
not known to move into brackish or freshwater but instead presumably
remains in the saline parts of the Scheldt estuary
\citep{dewysiecki_2017}. However, no study has yet been conducted on the
presence of this species in the Western Scheldt. The Eastern Scheldt,
located adjacent to the Western Scheldt, is a confirmed pupping ground
with presence of adult females in summer months
\citep{breve_2016, breve_2020}. This study intends to fill the knowledge
gap about seasonal presence of \emph{M. asterias} in the Western Scheldt
and the BPNS, and thus aims to answer the following questions:

\begin{itemize}
\tightlist
\item
  Research Question 1: What seasonal patterns of presence does \emph{M.
  asterias} display in the Western Scheldt and the BPNS?
\item
  Research Question 2: Are there differences in seasonal presence
  between males and females?
\item
  Research Question 3: How do the spatial preferences of \emph{M.
  asterias} vary within the Western Scheldt and the BPNS?
\end{itemize}

\hypertarget{depth-utilisation-of-m.-asterias-research-line-2}{%
\subsubsection*{\texorpdfstring{Depth utilisation of \emph{M. asterias}
(Research Line
2)}{Depth utilisation of M. asterias (Research Line 2)}}\label{depth-utilisation-of-m.-asterias-research-line-2}}
\addcontentsline{toc}{subsubsection}{Depth utilisation of \emph{M.
asterias} (Research Line 2)}

Information on depth use of \emph{M. asterias} is scarce, yet vital for
the investigation of behavioural patterns such as feeding, resting, and
migrating. \citet{griffiths_2020} reported a maximum depth of 118 m from
one data storage tag implanted in a female \emph{M. asterias}. This
suggests that \emph{M. asterias} uses waters deeper than previously
thought \citep[depths \textless100 m,][]{ices_2019}. A study focusing on
seasonal depth utilisation patterns of \emph{M. asterias} has not been
conducted yet, and this work aims to contribute to filling this
knowledge gap. Therefore, the following research questions will be
assessed:

\begin{itemize}
\tightlist
\item
  Research Question 4: What seasonal patterns of vertical movement does
  \emph{M. asterias} show?
\item
  Research Question 5: Are there differences in depth utilisation
  between males and females?
\end{itemize}

Due to its ability to collect information both within the area of
acoustic receiver networks and beyond those by locally storing data,
Acoustic Data Storage Tags (ADST) are a suitable technology to assess
these questions.

\hypertarget{sec-mm}{%
\section{Materials and methods}\label{sec-mm}}

\hypertarget{acoustic-tags-and-tagging-procedure}{%
\subsection{Acoustic tags and tagging
procedure}\label{acoustic-tags-and-tagging-procedure}}

Tagging of 30 \emph{M. asterias} individuals occurred in July and
August, 2018 (July 19, n = 13, August 02 and 03, n = 8 and n = 1) and in
July, 2019 (July 11 and 12, n = 4 for both dates). Tagged female
\emph{M. asterias} (n = 19, mean TL = 0.86 cm) were significantly larger
than male \emph{M. asterias} (n = 11, mean TL = 0.78 cm, one-sided Welch
Two Sample t-test, \(\alpha\) = 0.95, \emph{p} = 0.0021448). Details on
the tagged individuals are listed in Table~\ref{tbl-animalinfo}.

Innovasea ADST-V13TP tags (Innovasea Ltd., Boston, MA, USA) were used
for this study, measuring 46 mm in length and weighing 13 g. Each tag
had an estimated battery lifetime of 518 days, and contained a pressure
and a temperature sensor with logging intervals of 120 s and 240 s,
respectively. For each acoustic detection, a pressure or temperature
value was transmitted with a ratio of 1:3 (pressure:temperature).
Further details about the tags can be found in the Supporting
Information of \citet{goossens_2023}. The pressure sensors recorded with
a resolution of 0.3 m and an accuracy of \(\pm\) 3.4 m until the depth
of 68 m, according to the manufacturer \citep{Innovasea_ADST}. The
maximum depth value recorded in this study is 75.2 m. Since this is
deeper than the maximum depth measurable according to the manufacturers,
depth records around the limit of 68 m have to be interpreted with
caution.

Tagged animals were both caught and released in two locations, just west
of \emph{Neeltje Jans} (an artificial island in the Dutch province of
Zeeland, individuals tagged in July, both 2018 and 2019, Latitude:
51.6135652, Longitude: 3.6561763), and outside the mouth of the Western
Scheldt (individuals tagged in August 2018, Latitude: 51.4285383,
Longitude: 3.4074816). All animals were released within a 2 km radius
around the two tagging locations.

Animals were caught using line fishing or handline. After capture, each
animal was placed in a holding tank filled with sea water. Prior to tag
insertion, the animals were turned with the ventral side facing upwards
to induce tonic immobility \citep{kessel_2015}. Next, the midventral
line was incised 2-3 cm with a scalpel and the Acoustic Data Storage Tag
was inserted into the abdominal cavity with forceps. Finally, the
incision was closed with 2-3 stitches of mono-filamentous surgical
thread \citep{goossens_2023}. Photos of the capture and tagging of
\emph{M. asterias} are shown in Figure~\ref{fig-taggingpics}.

\hypertarget{tbl-animalinfo}{}
\begin{table}[H]
\caption{\label{tbl-animalinfo}Information on the tagged individuals. Release location 1 refers to
Neeltje Jans, release location 2 refers to the outer Western Scheldt. d
= days, h = hours, Det. = detected, Loc. = location, RI = Residency
Index (days detected / 518 days). The 518 days correspond to the battery
life estimated by the tag manufacturer. Days at liberty are the period
between release and last detection, or death date from the DST depth
log. Recovered tags highlighted in grey. }\tabularnewline

\centering
\fontsize{7}{9}\selectfont
\begin{tabular}{llrrlrrrlrrr}
\toprule
\textbf{\makecell[c]{Tag Serial\\No.}} & \textbf{Sex} & \textbf{\makecell[c]{TL\\ in m}} & \textbf{\makecell[c]{Body Mass\\ in kg}} & \textbf{\makecell[c]{Release\\Date}} & \textbf{\makecell[c]{Times\\Det.}} & \textbf{\makecell[c]{d\\Det.}} & \textbf{\makecell[c]{h\\Det.}} & \textbf{\makecell[c]{Date last\\Det.}} & \textbf{\makecell[c]{Days at\\Liberty}} & \textbf{RI} & \textbf{\makecell[c]{Release\\Loc.}}\\
\midrule
1293314 & f & 0.71 & NA & 2018-07-19 & 19 & 1 & 3 & 2018-07-31 & 12 & 0.002 & 1\\
\cellcolor[HTML]{E2E2E2}{\textcolor{black}{1293321}} & \cellcolor[HTML]{E2E2E2}{\textcolor{black}{m}} & \cellcolor[HTML]{E2E2E2}{\textcolor{black}{0.81}} & \cellcolor[HTML]{E2E2E2}{\textcolor{black}{NA}} & \cellcolor[HTML]{E2E2E2}{\textcolor{black}{2018-07-19}} & \cellcolor[HTML]{E2E2E2}{\textcolor{black}{0}} & \cellcolor[HTML]{E2E2E2}{\textcolor{black}{0}} & \cellcolor[HTML]{E2E2E2}{\textcolor{black}{0}} & \cellcolor[HTML]{E2E2E2}{\textcolor{black}{NA}} & \cellcolor[HTML]{E2E2E2}{\textcolor{black}{485}} & \cellcolor[HTML]{E2E2E2}{\textcolor{black}{NA}} & \cellcolor[HTML]{E2E2E2}{\textcolor{black}{1}}\\
1293315 & m & 0.82 & NA & 2018-07-19 & 159 & 6 & 20 & 2019-05-22 & 307 & 0.012 & 1\\
1293316 & m & 0.73 & NA & 2018-07-19 & 41 & 1 & 4 & 2019-07-09 & 355 & 0.002 & 1\\
\cellcolor[HTML]{E2E2E2}{\textcolor{black}{1293322}} & \cellcolor[HTML]{E2E2E2}{\textcolor{black}{f}} & \cellcolor[HTML]{E2E2E2}{\textcolor{black}{0.80}} & \cellcolor[HTML]{E2E2E2}{\textcolor{black}{NA}} & \cellcolor[HTML]{E2E2E2}{\textcolor{black}{2018-07-19}} & \cellcolor[HTML]{E2E2E2}{\textcolor{black}{0}} & \cellcolor[HTML]{E2E2E2}{\textcolor{black}{0}} & \cellcolor[HTML]{E2E2E2}{\textcolor{black}{0}} & \cellcolor[HTML]{E2E2E2}{\textcolor{black}{NA}} & \cellcolor[HTML]{E2E2E2}{\textcolor{black}{19}} & \cellcolor[HTML]{E2E2E2}{\textcolor{black}{NA}} & \cellcolor[HTML]{E2E2E2}{\textcolor{black}{1}}\\
\addlinespace
1293317 & f & 0.80 & NA & 2018-07-19 & 4 & 1 & 1 & 2018-07-21 & 2 & 0.002 & 1\\
1293318 & f & 0.79 & NA & 2018-07-19 & 14 & 3 & 4 & 2019-09-24 & 432 & 0.006 & 1\\
\cellcolor[HTML]{E2E2E2}{\textcolor{black}{1293319}} & \cellcolor[HTML]{E2E2E2}{\textcolor{black}{f}} & \cellcolor[HTML]{E2E2E2}{\textcolor{black}{0.73}} & \cellcolor[HTML]{E2E2E2}{\textcolor{black}{NA}} & \cellcolor[HTML]{E2E2E2}{\textcolor{black}{2018-07-19}} & \cellcolor[HTML]{E2E2E2}{\textcolor{black}{7}} & \cellcolor[HTML]{E2E2E2}{\textcolor{black}{1}} & \cellcolor[HTML]{E2E2E2}{\textcolor{black}{1}} & \cellcolor[HTML]{E2E2E2}{\textcolor{black}{2018-08-01}} & \cellcolor[HTML]{E2E2E2}{\textcolor{black}{19}} & \cellcolor[HTML]{E2E2E2}{\textcolor{black}{0.002}} & \cellcolor[HTML]{E2E2E2}{\textcolor{black}{1}}\\
1293320 & m & 0.81 & NA & 2018-07-19 & 0 & 0 & 0 & NA & 0 & NA & 1\\
1293293 & m & 0.78 & NA & 2018-07-19 & 0 & 0 & 0 & NA & 0 & NA & 1\\
\addlinespace
1293294 & f & 0.70 & NA & 2018-07-19 & 0 & 0 & 0 & NA & 0 & NA & 1\\
\cellcolor[HTML]{E2E2E2}{\textcolor{black}{1293295}} & \cellcolor[HTML]{E2E2E2}{\textcolor{black}{m}} & \cellcolor[HTML]{E2E2E2}{\textcolor{black}{0.78}} & \cellcolor[HTML]{E2E2E2}{\textcolor{black}{NA}} & \cellcolor[HTML]{E2E2E2}{\textcolor{black}{2018-07-19}} & \cellcolor[HTML]{E2E2E2}{\textcolor{black}{21}} & \cellcolor[HTML]{E2E2E2}{\textcolor{black}{2}} & \cellcolor[HTML]{E2E2E2}{\textcolor{black}{5}} & \cellcolor[HTML]{E2E2E2}{\textcolor{black}{2018-08-19}} & \cellcolor[HTML]{E2E2E2}{\textcolor{black}{31}} & \cellcolor[HTML]{E2E2E2}{\textcolor{black}{0.004}} & \cellcolor[HTML]{E2E2E2}{\textcolor{black}{1}}\\
1293296 & m & 0.77 & NA & 2018-07-19 & 1 & 1 & 0 & NA & 1 & NA & 1\\
1293297 & f & 0.88 & 3.30 & 2018-08-02 & 1134 & 49 & 138 & 2019-09-26 & 420 & 0.095 & 2\\
1293300 & f & 0.98 & 3.00 & 2018-08-02 & 304 & 28 & 69 & 2018-10-27 & 86 & 0.054 & 2\\
\addlinespace
1293298 & f & 0.95 & 3.80 & 2018-08-02 & 239 & 8 & 24 & 2019-09-18 & 412 & 0.015 & 2\\
1293307 & f & 0.83 & 4.20 & 2018-08-02 & 596 & 36 & 81 & 2019-09-23 & 417 & 0.069 & 2\\
\cellcolor[HTML]{E2E2E2}{\textcolor{black}{1293308}} & \cellcolor[HTML]{E2E2E2}{\textcolor{black}{f}} & \cellcolor[HTML]{E2E2E2}{\textcolor{black}{0.99}} & \cellcolor[HTML]{E2E2E2}{\textcolor{black}{3.70}} & \cellcolor[HTML]{E2E2E2}{\textcolor{black}{2018-08-02}} & \cellcolor[HTML]{E2E2E2}{\textcolor{black}{2067}} & \cellcolor[HTML]{E2E2E2}{\textcolor{black}{55}} & \cellcolor[HTML]{E2E2E2}{\textcolor{black}{257}} & \cellcolor[HTML]{E2E2E2}{\textcolor{black}{2019-10-05}} & \cellcolor[HTML]{E2E2E2}{\textcolor{black}{429}} & \cellcolor[HTML]{E2E2E2}{\textcolor{black}{0.106}} & \cellcolor[HTML]{E2E2E2}{\textcolor{black}{2}}\\
1293301 & f & 0.95 & 3.30 & 2018-08-02 & 1 & 1 & 0 & NA & 1 & NA & 2\\
1293299 & f & 0.91 & 3.50 & 2018-08-02 & 3520 & 84 & 410 & 2019-10-30 & 454 & 0.162 & 2\\
\addlinespace
1293309 & f & 0.92 & 3.30 & 2018-08-02 & 9 & 1 & 2 & 2018-09-24 & 53 & 0.002 & 2\\
1293302 & f & 1.02 & 4.60 & 2018-08-03 & 77 & 6 & 14 & 2019-08-23 & 385 & 0.012 & 2\\
1293303 & f & 0.93 & 3.20 & 2019-07-11 & 2210 & 31 & 196 & 2020-06-18 & 343 & 0.060 & 1\\
\cellcolor[HTML]{E2E2E2}{\textcolor{black}{1293304}} & \cellcolor[HTML]{E2E2E2}{\textcolor{black}{f}} & \cellcolor[HTML]{E2E2E2}{\textcolor{black}{0.72}} & \cellcolor[HTML]{E2E2E2}{\textcolor{black}{1.35}} & \cellcolor[HTML]{E2E2E2}{\textcolor{black}{2019-07-11}} & \cellcolor[HTML]{E2E2E2}{\textcolor{black}{0}} & \cellcolor[HTML]{E2E2E2}{\textcolor{black}{0}} & \cellcolor[HTML]{E2E2E2}{\textcolor{black}{0}} & \cellcolor[HTML]{E2E2E2}{\textcolor{black}{NA}} & \cellcolor[HTML]{E2E2E2}{\textcolor{black}{9}} & \cellcolor[HTML]{E2E2E2}{\textcolor{black}{NA}} & \cellcolor[HTML]{E2E2E2}{\textcolor{black}{1}}\\
1293305 & m & 0.72 & 0.90 & 2019-07-11 & 0 & 0 & 0 & NA & 0 & NA & 1\\
\addlinespace
\cellcolor[HTML]{E2E2E2}{\textcolor{black}{1293310}} & \cellcolor[HTML]{E2E2E2}{\textcolor{black}{m}} & \cellcolor[HTML]{E2E2E2}{\textcolor{black}{0.72}} & \cellcolor[HTML]{E2E2E2}{\textcolor{black}{1.40}} & \cellcolor[HTML]{E2E2E2}{\textcolor{black}{2019-07-11}} & \cellcolor[HTML]{E2E2E2}{\textcolor{black}{0}} & \cellcolor[HTML]{E2E2E2}{\textcolor{black}{0}} & \cellcolor[HTML]{E2E2E2}{\textcolor{black}{0}} & \cellcolor[HTML]{E2E2E2}{\textcolor{black}{NA}} & \cellcolor[HTML]{E2E2E2}{\textcolor{black}{32}} & \cellcolor[HTML]{E2E2E2}{\textcolor{black}{NA}} & \cellcolor[HTML]{E2E2E2}{\textcolor{black}{1}}\\
1293306 & m & 0.79 & 1.70 & 2019-07-12 & 0 & 0 & 0 & NA & 0 & NA & 1\\
\cellcolor[HTML]{E2E2E2}{\textcolor{black}{1293312}} & \cellcolor[HTML]{E2E2E2}{\textcolor{black}{f}} & \cellcolor[HTML]{E2E2E2}{\textcolor{black}{0.91}} & \cellcolor[HTML]{E2E2E2}{\textcolor{black}{2.80}} & \cellcolor[HTML]{E2E2E2}{\textcolor{black}{2019-07-12}} & \cellcolor[HTML]{E2E2E2}{\textcolor{black}{0}} & \cellcolor[HTML]{E2E2E2}{\textcolor{black}{0}} & \cellcolor[HTML]{E2E2E2}{\textcolor{black}{0}} & \cellcolor[HTML]{E2E2E2}{\textcolor{black}{NA}} & \cellcolor[HTML]{E2E2E2}{\textcolor{black}{25}} & \cellcolor[HTML]{E2E2E2}{\textcolor{black}{NA}} & \cellcolor[HTML]{E2E2E2}{\textcolor{black}{1}}\\
1293311 & m & 0.81 & 1.70 & 2019-07-12 & 64 & 1 & 6 & 2020-04-15 & 278 & 0.002 & 1\\
1293313 & f & 0.79 & 1.90 & 2019-07-12 & 455 & 21 & 62 & 2020-07-09 & 363 & 0.041 & 1\\
\bottomrule
\end{tabular}
\end{table}

\begin{figure}

\begin{minipage}[t]{0.33\linewidth}

{\centering 

\raisebox{-\height}{

\includegraphics{C:/Users/lotte.pohl/Documents/github_repos/MasterThesis_LottePohl/00_data/MustelusAsterias_scalebar.png}

}

}

\subcaption{\label{fig-taggingpics-1}A \emph{M. asterias} individual
after capture.}
\end{minipage}%
%
\begin{minipage}[t]{0.33\linewidth}

{\centering 

\raisebox{-\height}{

\includegraphics{C:/Users/lotte.pohl/Documents/github_repos/MasterThesis_LottePohl/00_data/MustelusAsterias_tagging_scalebar.png}

}

}

\subcaption{\label{fig-taggingpics-2}Tag implantation surgery.}
\end{minipage}%
%
\begin{minipage}[t]{0.33\linewidth}

{\centering 

\raisebox{-\height}{

\includegraphics{C:/Users/lotte.pohl/Documents/github_repos/MasterThesis_LottePohl/00_data/MustelusAsterias_suture_scalebar.png}

}

}

\subcaption{\label{fig-taggingpics-3}Suture after ADST implantation.}
\end{minipage}%

\caption{\label{fig-taggingpics}Images from the tagging of \emph{M.
asterias}. The white scale bars in the bottom left corners correspond to
5 cm. ©Pieterjan Verhelst (ORCID:
\href{https://orcid.org/0000-0002-2610-6941}{0000-0002-2610-6941}).}

\end{figure}

\hypertarget{ethics-statement}{%
\subsubsection{Ethics statement}\label{ethics-statement}}

The care and use of the animals complied with the Belgian Animal Welfare
Act, guidelines and policies as approved by the \emph{Ethische Commissie
Dierproeven} (references 2016.D-0041.004 and 2016.D-0041.008). All
animal treatments were undertaken in 2018 and 2019, and all tagging
procedures were carried out by competent and trained personal license
holders.

\hypertarget{tag-recoveries}{%
\subsubsection{Tag recoveries}\label{tag-recoveries}}

Upon writing (June, 2023), eight tags were recovered. Information about
the tags with available raw data is listed in Table~\ref{tbl-dstsum}.

\hypertarget{tbl-dstsum}{}
\begin{table}[H]
\caption{\label{tbl-dstsum}Information on the recovered Acoustic Data Storage Tags. The death date
was visually determined from the raw depth log and marks the date after
which only a tidal signal remains in the depth timeseries. The days at
liberty is the period between the release date and death date, or the
last acoustic detection (if the animal lived longer than the tag was
active). The tag recapture date was not recorded for tags 1293321,
1293308 and 1293304. }\tabularnewline

\centering
\fontsize{10}{12}\selectfont
\begin{tabular}{llllrl}
\toprule
\textbf{tag serial no.} & \textbf{sex} & \textbf{release date} & \textbf{death date} & \textbf{days at liberty} & \textbf{recapture date}\\
\midrule
1293319 & f & 2018-07-19 & 2018-08-07 & 19 & 2018-08-21\\
1293322 & f & 2018-07-19 & 2018-08-07 & 19 & 2019-09-29\\
1293295 & m & 2018-07-19 & 2018-08-18 & 30 & 2018-09-27\\
1293321 & m & 2018-07-19 & 2019-11-16 & 485 & NA\\
1293308 & f & 2018-08-02 & 2019-08-03 & 366 & NA\\
\addlinespace
1293304 & f & 2019-07-11 & 2019-07-20 & 9 & NA\\
1293310 & m & 2019-07-11 & 2019-08-12 & 32 & 2019-09-03\\
1293312 & f & 2019-07-12 & 2019-08-06 & 25 & 2019-08-29\\
\bottomrule
\end{tabular}
\end{table}

Six out of eight recovered tags logged for approximately one month or
less (median days at liberty = 22, max = 32, min = 9), only the tags
with the serial numbers 1293308 (f) and 1293321 (m) logged for a year or
more, allowing for the analysis of seasonal vertical behaviour. These
two tags will be referred to as \emph{long-term DST} in the following.
The rest of the tags will be referred to as \emph{short-term DST}. All
tag serial numbers will be abbreviated to their last three digits (i.e.,
\emph{1293308} will be referred to as \emph{308}).

The Acoustic Data Storage Tags (ADST) used in this study combine
acoustic detections with data storage logs. For maximum data overlap,
the tag should both be acoustically detected and retrieved (required to
access the depth and temperature logs). This was the case for three out
of the 30 tags deployed in this study (see Table~\ref{tbl-animalinfo}
and Table~\ref{tbl-dstsum}). Tag 319 was detected seven times, and the
individual (f) died 19 days post-release, tag 295 was detected 21 times
and the shark (m) died 30 days post-release, and tag 308 was detected
2067 times and this female was acoustically detected last 366 days
post-release. Thus, only one tag (tag 308) shows sufficient overlap
between acoustic detections and the depth and temperature logs.

\hypertarget{sec-mm_studyarea}{%
\subsection{Study areas}\label{sec-mm_studyarea}}

The study areas encompass two scales, reflecting the two lines of
research of this study (Section~\ref{sec-intro-aims}). The first
research line focuses on the Belgian Part of the North Sea (BPNS) and
the estuary of the Western Scheldt. It also encompasses the tagging
location \emph{Neeltje Jans}, which is not part of the Western Scheldt
or the BPNS but lies within the Eastern Scheldt. The second research
line includes year-round behaviour and thus, the study area is broader
and less defined. Since the species is thought to move both into the
English Channel and into the North Sea north of the BPNS
\citep{breve_2016, griffiths_2020}, both regions constitute the study
area for the second line of research. An overview on the two study areas
is displayed in Figure~\ref{fig-mapstudyarea}.

\hypertarget{sec-mmscheldtBPNS}{%
\subsubsection{The BPNS and Scheldt estuary (Research Line
1)}\label{sec-mmscheldtBPNS}}

The Belgian Part of the North Sea (BPNS) is generally characterised by
shallow water depths below 40 m \citep{thierry_2019}, sandy or muddy
sediment \citep{wolff_1973, vasquez_2021} and multiple sandbanks and
troughs. The river Scheldt meets the North Sea north of the border
between the Netherlands and Belgium, forming an estuary with a complex
regime of sandflats and channels \citep{claessens_1988}. Both the BPNS
and the Scheldt estuary are a highly dynamic environment due to strong
currents and tides \citep{otto_1990}, introducing saline waters well
into the Scheldt estuary \citep{ouboter_1998}. The Scheldt estuary is
characterised by medium to coarse sand and salinity between ranges 2 and
33.5 ppt \citep{baeyens_1998}, peaking in summer \citep{maes_1998}. The
seafloor depth averages around 11 m with occasional troughs of up to 20
or 30 m \citep{thierry_2019}. The BPNS and the Scheldt estuary are
influenced by two water bodies: The Channel water and the Continental
coastal water, meeting about 30 km off the Belgian coast
\citep{wolff_1973}. The former is characterised by high salinity
(\textgreater{} 18 ppt), low nutrient levels and water temperatures
ranging from 6 °C in winter to 16 °C in summer. The latter has a lower
salinity, high nutrient levels and a larger temperature range than the
Channel water from 3 °C in winter and 17 °C in summer. An overview map
of the BPNS and the Scheldt estuary is provided in
Figure~\ref{fig-mapstudyarea-1}. Fish and crustacean species composition
and abundance underlies strong seasonal patterns, peaking in summer
\citep{maes_1998, maes_2005}. Seasonally occuring species are marine
fish like Atlantic herring (\emph{Clupea harengus}), Sprat
(\emph{Sprattus sprattus}) or Whiting (\emph{Merlangius merlangus}), and
abundant crustacean species include the Common shrimp (\emph{Crangon
crangon}) and the Palaemonid shrimp \citep[\emph{Palaemon
varians},][]{maes_1998}. The lower part of the Scheldt estuary from
Flushing (Dutch: Vlissingen) to 55-60 km upstream is referred to as
Western Scheldt estuary \citep{baeyens_1998, ouboter_1998}. Between
Bergen op Zoom and the island Neeltje Jans (both Netherlands), the
estuary is referred to as the Eastern Scheldt estuary.

\begin{figure}

\begin{minipage}[t]{\linewidth}

{\centering 

\raisebox{-\height}{

\includegraphics{thesis_manuscript_files/figure-pdf/fig-mapstudyarea-1.pdf}

}

}

\subcaption{\label{fig-mapstudyarea-1}The Belgian Part of the North Sea
(BPNS) and the Scheldt estuary. ES = Eastern Scheldt, WS = Western
Scheldt. Each black circle marks one acoustic receiver station.}
\end{minipage}%
\newline
\begin{minipage}[t]{\linewidth}

{\centering 

\raisebox{-\height}{

\includegraphics{thesis_manuscript_files/figure-pdf/fig-mapstudyarea-2.pdf}

}

}

\subcaption{\label{fig-mapstudyarea-2}The Southern North Sea and the
English Channel.}
\end{minipage}%

\caption{\label{fig-mapstudyarea}Overview maps of the study area. Panel
\textbf{a)} shows the Belgian Part of the North Sea (BPNS) and the
Scheldt estuary. Panel \textbf{b)} shows the Southern North Sea and the
English Channel. Bold labels refer to relevant locations, italicised
labels refer to relevant receiver stations.}

\end{figure}

\hypertarget{sec-mmstudyareaec}{%
\subsubsection{The East Anglian coast and the English Channel (Research
Line 2)}\label{sec-mmstudyareaec}}

The East Anglian coast lies within the Southern North Sea and is
characterised by shallow coastal waters between 20 and 25 m water depth,
fast tidal streams and sandy and muddy seafloor sediments
\citep{harrison_1990}. Tidal currents are strongest at 1.3
\(\frac{m}{s}\) during new and full moon \citep{arnold_1994}. Abundant
crustacea include \emph{Hyas coarctatus} \citep{dyer_1985},
\emph{Pandalus montagui}, \emph{Carcinus spp.} and \emph{Eupagurus spp.}
\citep{sergeant_1951}.

The English Channel extends from the Dover Strait to England's
southwestern tip, spanning 77 000 km\textsuperscript{2}
\citep{dauvin_2012}. The mean seafloor depth increases from around 25 m
in the east to 75 m in the west \citep{dauvin_2012}, except for the 174
m deep Hurd Deep, a trench located close to the channel island Guernsey.
An overview of the English Channel and the North Sea is provided in
Figure~\ref{fig-mapstudyarea-1}. Strong tidal currents from west to east
are present, which are higher on the French than on the English coast
due to Coriolis force, reaching maximum values of 4 \(\frac{m}{s}\) at
the Cap de la Hague \citep{salomon_1993}. Bottom temperatures vary less
in the Western English Channel (range from 9 °C to 14 °C) than in the
Eastern English Channel (range from 8 °C to 16 °C). Species living on
the seafloor include Crustacea like the edible crab (\emph{Cancer
pagurus}), the common spider crab (\emph{Maja squinado}), the European
lobster (\emph{Homarus gammarus}) and \emph{Upogebia spp.}
\citep{holme_1966, vaz_2007}.

\hypertarget{the-permanent-belgian-receiver-network}{%
\subsection{The Permanent Belgian Receiver
Network}\label{the-permanent-belgian-receiver-network}}

The acoustic receivers of the Permanent Belgian Receiver Network (PBARN)
were used in this study \citep{reubens_2018}. The PBARN is part of the
European Tracking Network (ETN), which was established in 2017 as a
pan-european aquatic telemetry network. As of 2019, the PBARN consisted
of 160 permanently installed and monitored acoustic receivers within the
BPNS, the Scheldt estuary, and adjacent rivers or canals.
Figure~\ref{fig-mapWS} displays an overview of the acoustic receivers in
the estuary of the Western Scheldt. A detection probability of
\textgreater50 \% was estimated for a radius from 500 to 700 m around
the receiver \citep{goossens_2022}, and \textgreater70 \% for 200 m
radius \citep{reubens_2018}.

\begin{figure}

\begin{minipage}[t]{\linewidth}

{\centering 

\raisebox{-\height}{

\includegraphics{thesis_manuscript_files/figure-pdf/fig-mapWS-1.pdf}

}

\caption{\label{fig-mapWS}Acoustic receivers of the Permanent Belgian
Acoustic Receiver Network (PBARN) in the Western Scheldt estuary. The
receiver stations (circles) are categorised into three different arrays:
Western Scheldt 1 (WS1), Western Scheldt 2 (WS2) and Western Scheldt 3
(WS3). Bold labels refer to relevant locations, italicised labels refer
to relevant receiver stations.}

}

\end{minipage}%

\end{figure}

\hypertarget{data-analysis-and-visualisation}{%
\subsection{Data analysis and
visualisation}\label{data-analysis-and-visualisation}}

All data analyses were carried out using R \citep{R_2022}, primarily
using the packages \texttt{dplyr} \citep{dplyr} and \texttt{lubridate}
\citep{lubridate}. Geospatial data was queried from the marineregions
database \citep{claus_2014} in the case of (marine) boundaries
\citep[with the \texttt{mregions2}-package,][]{mregions2}, and EMODnet
Human Activities \citep{solaun_2021} data \citep[with the
\texttt{EMODnetWFS}-package,][]{EMODnetWFS}. All plots were generated
using the \texttt{ggplot2}-package \citep{ggplot2}. Packages used for
specific data analyses are indicated in the respective section.

For all tags, days at liberty were calculated. Days at liberty were
considered the maximum number of days between the animal's release and
its last detection by a receiver, or the death date determined from the
retrieved depth log.

\hypertarget{acoustic-detections-research-line-1}{%
\subsubsection{Acoustic detections (Research Line
1)}\label{acoustic-detections-research-line-1}}

Detections were accessed from the ETN database
(https://www.lifewatch.be/etn/) using the \texttt{etn}-package
\citep{etn}. Tags with only a single detection were omitted from all
further analyses. Acoustic detections were visualised on an individual
and acoustic receiver station level. Residency Indices were calculated
following the two definitions provided by \citet{kraft_2023}. Firstly,
the overall RI was calculated for all individuals, dividing the number
of days with a detection by the estimated battery life of 518 days.
Secondly, a subset RI was calculated for station OG10 in 2019, which had
the highest number of individuals detected and detections. This
calculation involved dividing the number of days detected at this
receiver by the time period between the first and last detection of any
individual at that receiver.

\hypertarget{sec-mm_dsts}{%
\subsubsection{Data Storage Tags (Research Line 2)}\label{sec-mm_dsts}}

Depth and temperature logs from the recovered tags were accessed via the
Marine Data Archive (MDA, dataset DOI: https://doi.org/10.14284/605).
These logs are the same dataset as the \emph{M. asterias} data from
\citet{goossens_2023}. The temperature log was not included in any
analysis. The first week of depth values was excluded from all analyses
to remove potential bias resulting from the tagging procedure
\citep[following][]{flavio_2021}. The date of death, used to calculate
the time at liberty for the recovered tags, was visually extracted from
plots of the raw depth series, and marks the date after which only a
tidal signal remains in the depth log \citep[characterised by cyclically
rising and falling depths within a rough 2 m depth
range,][]{kvale_2006}.

The analyses described in the following were only performed for tags
that logged for \(\geq\) 365 days, i.e., tag 308 (female) and tag 321
(male). Those will be referred to as \emph{long-term DST} in the
following, returned tags logging for less than a year will be referred
to as \emph{short-term DST}.

Summary statistics, i.e., median (since depth did not follow a normal
distribution), maximum and minimum depth were calculated per day,
daytime and nighttime. For the daytime/nighttime summaries, sunrise and
sunset times were extracted using the
\texttt{getSunlightTimes()}-function from the \texttt{suncalc}-package
\citep{suncalc}. The illuminated fraction of the moon was extracted
using the \texttt{moonAngle()}-function from the \texttt{oce}-package
\citep{oce}.

A Savitzky-Golay smoothing filter \citep[ filter order \emph{p = 1},
filter length \emph{n = 5}]{press_1990} was applied on all summaries
(per day, daytime and nighttime) using the
\texttt{sgolayfilt()}-function from the \texttt{signal}-package
\citep{signal}. This particular filter smoothes the data but retains
spikes in the data better than other smoothing filters like butterworth
or averaging filters \citep[such as a rolling mean,][]{schafer_2011}.

The autocorrelation was calculated using the
\texttt{stats::acf()}-function. Autocorrelation refers to the serial
correlation of a data series with itself \citep{bartlett_1946},
resulting in an autocorrelation coefficient between -1 and 1 (-1
referring to perfect anti-correlation and 1 referring to perfect
correlation, 0 referring to no correlation at all) over the lag, which
refers to the overlap between the data series with itself.

A Fast Fourier Transform (FFT) was performed on all DST logs, using the
\texttt{base::fft()}-function. This method decomposes a signal in the
time domain (which is a combination of multiple sine and cosine waves
with different frequencies) into its components in the frequency domain,
resulting in a frequency spectrum of the decomposed signal. Dominant
frequencies (or rather periods, which is the inverse of the frequency)
give insights about the scale of periodicity of a signal (in the present
case the signal is the depth-time series of the sharks). These dominant
periods appear as spikes in the so-called periodogram, which is the
visual representation of the period spectrum of the signal
\citep{cochran_1967}. This analysis gives insights into periodicities
within the whole signals but does not resolv when these periodic
components occur.

To resolve prevalent periodicities in the time domain of a signal,
Continuous Wavelet Transform (CWT, a specific type of Wavelet Analysis)
was conducted \citep{grinsted_2004}. This technique analyses signals in
both the time and frequency domains simultaneously. CWT uses wavelet
functions, wave-shaped building blocks localised in both time and
frequency, which can be adjusted in size and position to capture
different features of a signal. These wavelet functions can be
understood as a spectral microscope that focuses on different time
intervals of a signal. The simultaneous resolution of time and
frequency, however, reduces the accuracy of results. To obtain robust
and valid results despite the resolution loss, frequencies are compared
against white noise using a \(\chi^2\) test \citep{grinsted_2004}.

When applying the CWT to a signal, a two-dimensional, visual
representation called a wavelet scalogram is obtained. The wavelet
scalogram depicts the distribution of wavelet power as a function of
time and frequency. The power in the scalogram represents the strength
or magnitude of each periodicity at different time points. Areas with
higher power indicate stronger presence of the corresponding period.
Moreover, the scalogram also shows if the power of a specific period
significantly differs from white noise. Fully opaque areas in the
scalogram differ from white noise, and slightly transparent areas do not
significantly differ from white noise.

In the present study, CWT was conducted using the \texttt{wt()}-function
from the \texttt{biwavelet}-package \citep{biwavelet}. CWT was performed
on two depth-time series per shark, resulting in two wavelet scalograms
per individuals. Firstly, CWT was performed on the raw depth log (which
was subsampled to one depth measurement every 10 minutes to reduce
computation time), resulting in the power (the proxy for prevalence) of
periodicities in the scale of minutes and hours, hereafter referred to
as the \emph{hourly scale}. For this study, periodicities in the range
of tidal and diurnal cycles (between 6 and 24 hours) per date were of
interest. To include a margin of periodicities, the scalograms of the
CWT in the hourly scale show the prevalence (i.e., power) of periods
between 2 and 64 hours. The maximum power of a period for each date is
displayed since the highly prevalent periods are of interest. Secondly,
CWT was performed on the Savitzky-Golay filtered median depth per date
(including one depth value per date), which enables to resolve prevalent
periodicities in a signal on a scale of days (2 days to about 50 days),
hereafter referred to the CWT result on a \emph{daily scale}. This
filtered median daily depth (as opposed to the non-filtered one) was
chosen to obtain a smoother wavelet scalogram and ease the
interpretation of the results. The scalogram on the daily scale displays
the power (i.e., prevalence) of periods between 2 and 50 days, enabling
the investigation of periodical presence of weekly, biweekly or monthly
patterns in the depth-time series of the DST logs.

\newpage{}

\hypertarget{sec-results}{%
\section{Results}\label{sec-results}}

\hypertarget{sec-res-acoustic}{%
\subsection{Acoustic detections (Research Line
1)}\label{sec-res-acoustic}}

Two tags only had a single detection (serial numbers 296 and 301) and
were thus omitted from all further analyses. An overview of the acoustic
detections of the remaining tags is displayed in
Figure~\ref{fig-abacus}. Within the study period from July, 2018 to
August, 2020, there are no acoustic detections between November and
April in either year. 96.71\% of detections are in the WS1 area, and of
those, 68.26\% are at station OG10 (see Figure~\ref{fig-mapWS}). Of the
11 males that were tagged, 4 were detected more than once, 3 of which
were detected in the year following their tagging. Of the 19 tagged
females, 14 were detected more than once and 11 individuals were
detected in the year following their tagging (Figure~\ref{fig-abacus}).
Of the 4 detected males, 3 were detected inside the BPNS and one in the
WS1 array. Across all years, most detections (n = 2484) were in the
month of May, while most individuals (n = 10) were detected in August.
12 out of 18 individuals (67 \%) were detected in the year following
their release.

\begin{figure}[H]

{\centering \includegraphics[width=1\textwidth,height=\textheight]{thesis_manuscript_files/figure-pdf/fig-abacus-1.pdf}

}

\caption{\label{fig-abacus}Acoustic detections of the tagged Starry
smooth-hounds. Each circle represents one detection, the colour responds
to the receiver area. The shapes with black outlines refer to the
tagging date and the sex of the individual.}

\end{figure}

The acoustic detections at each receiver station of female \emph{M.
asterias} within the Western Scheldt are shown in
Figure~\ref{fig-heatmapdetections}. The receiver stations are ordered by
latitude (from north to south), and by receiver array (station OG10 to
W7 are in the WS1 array, and station borssele to 4 are in the WS2
array). Most detections (n = 2103) were at station OG10 in October,
2019. This is almost twice as much as the second most detections (n =
1097 at the same station in May, 2020). Most individuals (n = 6) were
detected at station WN2 in August and September, 2018.

\begin{figure}[H]

{\centering \includegraphics[width=1\textwidth,height=\textheight]{thesis_manuscript_files/figure-pdf/fig-heatmapdetections-1.pdf}

}

\caption{\label{fig-heatmapdetections}Detections per acoustic receiver
of female \emph{M. asterias} in the Western Scheldt (receiver arrays
\emph{WS1} and \emph{WS2}). The rectangles' colours correspond to the
number os acoustic detections at each station and month. No rectangle
refers to no detections at a receiver during a month. The white number
inside each rectangle refers to the amount of individuals detected at
that station that month. The stations are ordered per receiver array,
and from north to south.}

\end{figure}

Acoustic detections of female \emph{M. asterias} at receiver station
OG10 in 2019 are shown in Figure~\ref{fig-heatmapOG10}. The acoustic
detections of an individual at a day (median = 35 detections per day)
spread from 1 detection to 232 detections for tag 299 on October 13,
2019.

Tags 313, 308, 299 and 297 show a cluster of about 14 to 21 days with
consecutive detections. Tag 308 has two clusters of subsequent
detections, between May 13 and June 09, and between July 16 and August
12, 2019.

\begin{figure}[H]

{\centering \includegraphics[width=0.9\textwidth,height=\textheight]{thesis_manuscript_files/figure-pdf/fig-heatmapOG10-1.pdf}

}

\caption{\label{fig-heatmapOG10}Acoustic detections of female \emph{M.
asterias} individuals at receiver station OG10 in 2019. The colour of
the rectangles correspond to the number of detections per individual and
day. No rectangle means no detection of an individual during a day. The
Residency Index (RI, i.e.~the number of days detected divided by the
period between the first and last detection among all individuals at
station OG10 in 2019) is displayed next to the tag serial number.}

\end{figure}

\hypertarget{sec-results-dst}{%
\subsection{Data Storage Tags (Research Line 2)}\label{sec-results-dst}}

\hypertarget{raw-depth-and-temperature-logs}{%
\subsubsection{Raw depth and temperature
logs}\label{raw-depth-and-temperature-logs}}

The depth and temperature logs of tags 308 and 321 (the two long-term
DST) are shown in Figure~\ref{fig-dst308-1} and
Figure~\ref{fig-dst321-1}, respectively. Example subsets from summer
(Figure~\ref{fig-dst308-2} and Figure~\ref{fig-dst321-2}) and winter
(Figure~\ref{fig-dst308-3} and Figure~\ref{fig-dst321-3}) are also
displayed.

\newpage{}

\begin{figure}

\begin{minipage}[t]{\linewidth}

{\centering 

\raisebox{-\height}{

\includegraphics{thesis_manuscript_files/figure-pdf/fig-dst308-1.pdf}

}

}

\subcaption{\label{fig-dst308-1}Full depth log, subsampled to one value
every 30 minutes.}
\end{minipage}%
\newline
\begin{minipage}[t]{0.50\linewidth}

{\centering 

\raisebox{-\height}{

\includegraphics{thesis_manuscript_files/figure-pdf/fig-dst308-2.pdf}

}

}

\subcaption{\label{fig-dst308-2}Example depth log subset from September
2018.}
\end{minipage}%
%
\begin{minipage}[t]{0.50\linewidth}

{\centering 

\raisebox{-\height}{

\includegraphics{thesis_manuscript_files/figure-pdf/fig-dst308-3.pdf}

}

}

\subcaption{\label{fig-dst308-3}Example depth log subset from February
and March 2019.}
\end{minipage}%

\caption{\label{fig-dst308}DST deptlog from the recovered tag 308. This
female shark was tagged on August 2, 2018.}

\end{figure}

The full depth log of tag 308 (female, Figure~\ref{fig-dst308-1}),
comprising 366 days, shows depths between 0 and 20 m from the time of
tagging in August, 2018, to the beginning of October. Then the depth
gets deeper over a period of about two weeks, and between mid-November
and mid-December the sensor recorded depths between 50 and 75 m. From
December, 2018 until March, 2019, the depth ranges between 75 m and
close to the water surface. From mid-March on, the depth remains
shallower than 60 m. From May to June, 2019, depth is relatively
constant around 17 m, from mid-June to mid-July it is relatively
constant around 10 m and until the sensor stops recording the depth goes
back to just under 20 m. Figure~\ref{fig-dst308-2} shows an example
subset of the depth log of tag 308 in September, 2018. Roughly, the
depth alternates between 12 - 14 and 16 - 20 m in a sinusoidal pattern
with a wavelength of about 12 hours. Figure~\ref{fig-dst308-3} shows the
recorded depth between February and March, 2019. Here, periods of
smaller depth ranges from 40 to 60 m (from February 11 to 18 and from
February 25 to March 3) can be observed with alternating periods of
larger depth ranges from 5 to 75 m (from February 18 to 24).

\begin{figure}

\begin{minipage}[t]{\linewidth}

{\centering 

\raisebox{-\height}{

\includegraphics{thesis_manuscript_files/figure-pdf/fig-dst321-1.pdf}

}

}

\subcaption{\label{fig-dst321-1}Full depth log, subsampled to one value
every 30 minutes.}
\end{minipage}%
\newline
\begin{minipage}[t]{0.50\linewidth}

{\centering 

\raisebox{-\height}{

\includegraphics{thesis_manuscript_files/figure-pdf/fig-dst321-2.pdf}

}

}

\subcaption{\label{fig-dst321-2}Example depth log subset from September
2018.}
\end{minipage}%
%
\begin{minipage}[t]{0.50\linewidth}

{\centering 

\raisebox{-\height}{

\includegraphics{thesis_manuscript_files/figure-pdf/fig-dst321-3.pdf}

}

}

\subcaption{\label{fig-dst321-3}Example depth log subset from February
and March 2019.}
\end{minipage}%

\caption{\label{fig-dst321}DST deptlog from the recovered tag 321. This
male shark was tagged on Juy 19, 2018.}

\end{figure}

The full depth log of tag 321 (male, Figure~\ref{fig-dst321-1}) is 485
days long and shows depth ranges between 0 and 30 m from tagging until
the end of October, 2018. Then, depth gets deeper over a period of
roughly 10 days, stays at 35 to 50 m until December, 2018 and then
ranges from 5 m to 50 m until March, 2019. Until May, 2019 depth ranges
between 10 and 35 m, then the sensor recorded depth values of 60 m on
May 03, 2019. After that, depth gets shallower with ranges from 0 to
mostly 20 m until mid-September. Until the end of the depth log on
November 15, 2019, depth gets deeper to around 35 m and then to around
40 m. The example depth log of September, 2018 in
Figure~\ref{fig-dst321-2} shows a depth range between 10 and 15 m during
daytime (a sinusoidal pattern similar to the one in
Figure~\ref{fig-dst308-2}), and minimum depths of around 2 m during
nighttime. The subset from February to March, 2019
(Figure~\ref{fig-dst321-3}) shows depth ranges from 5 to 55 m with no
observable pattern of larger or smaller depth ranges on a weekly scale.

The raw depth and temperature logs of the short-term DST are included in
\textbf{Annex A}. The short-term DST logged depth and temperature data
for 22 \(\pm\) 9 days, reaching maximum water depths of 53.51 m \(\pm\)
10.24 m.

\hypertarget{sec-results-dst-summary}{%
\subsubsection{Summary statistics}\label{sec-results-dst-summary}}

The daily summary statistics of the depth logs of tags 308 and 321 are
displayed in Figure~\ref{fig-dstsum308} and Figure~\ref{fig-dstsum321},
respectively. Summaries per day, daytime and nighttime are included. The
solid lines show the Savitzky-Golay filtered median. The ribbons span
between the Savitzky-Golay filtered maximum and minimum depth. The
dotted line displays the fraction of the moon that is illuminated,
representing the moon phase.

\begin{figure}

\begin{minipage}[t]{\linewidth}

{\centering 

\raisebox{-\height}{

\includegraphics{thesis_manuscript_files/figure-pdf/fig-dstsum308-1.pdf}

}

}

\subcaption{\label{fig-dstsum308-1}Median, maximum and minimum depth per
day.}
\end{minipage}%
\newline
\begin{minipage}[t]{0.50\linewidth}

{\centering 

\raisebox{-\height}{

\includegraphics{thesis_manuscript_files/figure-pdf/fig-dstsum308-2.pdf}

}

}

\subcaption{\label{fig-dstsum308-2}Summary statistics per daytime.}
\end{minipage}%
%
\begin{minipage}[t]{0.50\linewidth}

{\centering 

\raisebox{-\height}{

\includegraphics{thesis_manuscript_files/figure-pdf/fig-dstsum308-3.pdf}

}

}

\subcaption{\label{fig-dstsum308-3}Summary statistics per nighttime.}
\end{minipage}%

\caption{\label{fig-dstsum308}Summary statistics, per day, daytime and
nighttime for tag 308 (f). Median depth as a solid line, maximum and
minimum depth as shaded ribbon. All depth summaries were Savitzky-Golay
filtered. Illuminated fraction of the moon as a dotted line.}

\end{figure}

Figure~\ref{fig-dstsum308-1} displays a daily maximum of about 20 m
until the end of September and then, until the start of October the
daily median depth gets deeper, reaching 75 m. Until the beginning of
December the shark uses waters with minimum daily depths not shallower
than 50 m. After that until mid March the daily maximum depth is between
40 and 68 m and the daily minimum depth is between 5 and 50 m. After
March 19, 2019, the daily median depth is always shallower than 60 m.
Between March until the end of the time series, some plateaus in median
depth occur: Around 35 m median depth from the end of March to the end
of April, around 17 m depth from mid May to the start of June and from
the end of July onward, and around 8 m depth between June 11 and July
13, 2019. The minimum depth during the night is significantly shallower
than during the day (one-sided Welch Two Sample t-test, \(\alpha\) =
0.95, \emph{p} = \ensuremath{8.0829717\times 10^{-9}}). From December,
2018 to May, 2019, the minimum and median depth per day
(Figure~\ref{fig-dstsum308-2}) show peaks following a rough biweekly
pattern, getting the shallowest when the illuminated moon fraction
reaches its maximum and minimum. During the night
(Figure~\ref{fig-dstsum308-3}), a rather monthly pattern can be observed
for that time period. Here, the median and minimum depth are shallowest
when the moon is fully illuminated.

\begin{figure}

\begin{minipage}[t]{\linewidth}

{\centering 

\raisebox{-\height}{

\includegraphics{thesis_manuscript_files/figure-pdf/fig-dstsum321-1.pdf}

}

}

\subcaption{\label{fig-dstsum321-1}Median, maximum and minimum depth per
day.}
\end{minipage}%
\newline
\begin{minipage}[t]{0.50\linewidth}

{\centering 

\raisebox{-\height}{

\includegraphics{thesis_manuscript_files/figure-pdf/fig-dstsum321-2.pdf}

}

}

\subcaption{\label{fig-dstsum321-2}Summary statistics per daytime.}
\end{minipage}%
%
\begin{minipage}[t]{0.50\linewidth}

{\centering 

\raisebox{-\height}{

\includegraphics{thesis_manuscript_files/figure-pdf/fig-dstsum321-3.pdf}

}

}

\subcaption{\label{fig-dstsum321-3}Summary statistics per nighttime.}
\end{minipage}%

\caption{\label{fig-dstsum321}Summary statistics, per day, daytime and
nighttime for tag 321 (m). Median depth as a solid line, maximum and
minimum depth as shaded ribbon. All depth summaries were Savitzky-Golay
filtered. Illuminated fraction of the moon as a dotted line.}

\end{figure}

The male shark (tag 321, Figure~\ref{fig-dstsum321-1}) uses waters
shallower than 20 m between August and October, 2018. From October 21
until November 03, 2018, the daily median depth gets consistently
deeper, reaching depths of around 45 m throughout November, 2018.
Between December 09, 2019 and January 21, 2019, the daily median depth
ranges from 22 to 33 m and the maximum and minimum depth between 53 and
16 m, respectively. After dropping to 39 m on January 28, 2019, the
daily median depth gets shallower until April 30, reaching depths of 22
m. Then, the daily median depth increases substantially between April 27
and May 05, ranging between 22 m and 6 m until October 05. From October
08 to 19, 2019, the daily median depth changes from 17 to 45 m. Until
the log stops on November 16, the daily median depth ranges between 45
and 32 m. The minimum depth during the night is significantly shallower
than during the day (one-sided Welch Two Sample t-test, \(\alpha\) =
0.95, \emph{p} = \ensuremath{1.6887159\times 10^{-19}}). The median
depth per day (Figure~\ref{fig-dstsum321-2}) shows slight peaks in a
biweekly pattern between January and July, 2019. The peaks of the median
depth at night (Figure~\ref{fig-dstsum321-3}) show no visually
noticeable pattern.

\hypertarget{sec-resacf}{%
\subsubsection{Autocorrelation}\label{sec-resacf}}

The autocorrelograms of the two long-term DST logs (computed using the
Savitztky-Golay filtered daily median depth) are shown in
Figure~\ref{fig-acf}. An autocorrelation value of -1 refers to perfect
anti-correlation, 1 referring to perfect correlation, and 0 refers to no
correlation at all. Results from the autocorrelation analyses display
similar patterns for both the male and female. For the female (tag 308,
Figure~\ref{fig-acf-1}), depth data shows no autocorrelation at 81 and
264 days. Values are inversely correlated after a period of 170 days,
while highest correlation occurs around 300 days, displaying a pattern
that closely resembles annual periodicity. The autocorrelation of daily
median depth for the male shark (tag 321, Figure~\ref{fig-acf-2}) is 0
at lags of 75, 256 and 383 days. Inverse autocorrelation occurs at a lag
of 138 days (around 4.5 months) and maximum autocorrelation occurs at
345 days, resembling an annual periodicity (similar to the female).

\begin{figure}

\begin{minipage}[t]{\linewidth}

{\centering 

\raisebox{-\height}{

\includegraphics{thesis_manuscript_files/figure-pdf/fig-acf-1.pdf}

}

}

\subcaption{\label{fig-acf-1}Tag 308 (f).}
\end{minipage}%
\newline
\begin{minipage}[t]{\linewidth}

{\centering 

\raisebox{-\height}{

\includegraphics{thesis_manuscript_files/figure-pdf/fig-acf-2.pdf}

}

}

\subcaption{\label{fig-acf-2}Tag 321 (m).}
\end{minipage}%

\caption{\label{fig-acf}Autocorrelograms of daily median depth. Numbers
and dots in blue display the days at which autocorrelation values are 0,
at their minimum and or at a local maximum.}

\end{figure}

\hypertarget{wavelet-analysis}{%
\subsubsection{Wavelet analysis}\label{wavelet-analysis}}

CWT was performed on both the raw depth log, subsampled to one value
every 10 minutes, and the Savitzky-Golay filtered daily median depth, as
described in Section~\ref{sec-mm_dsts}. The resulting wavelet scalograms
for the two long-term DST depth logs are shown below, in
Figure~\ref{fig-waveletresults308} for tag 308 (f) and in
Figure~\ref{fig-waveletresults321} for tag 321 (m).

\begin{figure}

\begin{minipage}[t]{\linewidth}

{\centering 

\raisebox{-\height}{

\includegraphics{thesis_manuscript_files/figure-pdf/fig-waveletresults308-1.pdf}

}

}

\subcaption{\label{fig-waveletresults308-1}Wavelet scalogram of the DST
depth log, subsampled to one value every 10 minutes. The prevalence
(with power as a proxy variable) of period patterns between 2 and 64
hours is displayed.}
\end{minipage}%
\newline
\begin{minipage}[t]{\linewidth}

{\centering 

\raisebox{-\height}{

\includegraphics{thesis_manuscript_files/figure-pdf/fig-waveletresults308-2.pdf}

}

}

\subcaption{\label{fig-waveletresults308-2}Wavelet scalogram of the
daily Savitzky-Golay filtered median depth. The power corresponds to the
prevalence of periodicities between 2 and 32 days.}
\end{minipage}%

\caption{\label{fig-waveletresults308}Continuous Wavelet Transform (CWT)
results for tag 308 (f). The fully opaque areas refer to periods that
significantly differ from white noise by means of a Chi-squared test.}

\end{figure}

Figure~\ref{fig-waveletresults308-1} shows high prevalence across all
periods (between 2 and about 50 hours) for the female (tag 308) in
October 2018, and between December, 2018 and May, 2019. In that time
period, significant periodicities spanning from 4 to 14 hours exist in a
rough interval of 2 weeks. To a lesser extent, period bands of 12 hours
are observable from August to September, 2018 and May to August, 2019.
The depth log of tag 308 (f) shows periodic cycles of approximately 14
and 30 days in October, 2018 and between December and June, 2019
(Figure~\ref{fig-waveletresults308-2}).

\begin{figure}

\begin{minipage}[t]{\linewidth}

{\centering 

\raisebox{-\height}{

\includegraphics{thesis_manuscript_files/figure-pdf/fig-waveletresults321-1.pdf}

}

}

\subcaption{\label{fig-waveletresults321-1}Wavelet scalogram of the DST
depth log, subsampled to one value every 10 minutes. The prevalence
(with power as a proxy variable) of period patterns between 2 and 64
hours is displayed.}
\end{minipage}%
\newline
\begin{minipage}[t]{\linewidth}

{\centering 

\raisebox{-\height}{

\includegraphics{thesis_manuscript_files/figure-pdf/fig-waveletresults321-2.pdf}

}

}

\subcaption{\label{fig-waveletresults321-2}Wavelet scalogram of the
daily Savitzky-Golay filtered median depth. The power corresponds to the
prevalence of periodicities between 2 and 32 days.}
\end{minipage}%

\caption{\label{fig-waveletresults321}Continuous Wavelet Transform (CWT)
results for tag 321 (f). The fully opaque areas refer to periods that
significantly differ from white noise by means of a Chi-squared test.}

\end{figure}

The wavelet scalogram of the depth log of tag 321 (male,
Figure~\ref{fig-waveletresults321-1}) shows punctually high
periodicities between 4 to 64 hours at the beginning and end of October,
2018, throughout May, 2019, and at the beginning of October, 2019.
Significant periodicities are sporadically present during these periods,
between roughly 4 and 12 hours. A period band of 24 hours is present
August to October, 2018, January and February, 2019 and between July and
October, 2019. A period band of 12 hours is present to a lesser extent
during the same periods.

The daily wavelet scalogram in Figure~\ref{fig-waveletresults321-2}
shows a significant presence of periodicities of 16 to 24 days between
October, 2018 and August, 2019. From September to November, 2019,
periods between about 8 and 50 days are present. From December, 2018 to
February, 2019 and from mid-April to June, 2019, periods between about 7
and 14 days are present.

\newpage{}

\hypertarget{combination-of-acoustic-detections-and-dst-logs}{%
\subsection{Combination of acoustic detections and DST
logs}\label{combination-of-acoustic-detections-and-dst-logs}}

Three out of the 30 tags deployed were both retrieved and detected by an
acoustic receiver. However, only one of those tags (tag 308, female)
provides sufficient data. Figure~\ref{fig-adst308} shows the raw depth
log of this tag, superposed with the acoustic detections of that femake.
After the DST had stopped recording on August 03, 2019, the shark was
recorded at station OG10. On October 05, 2019 it was recorded at the
Birkenfels station in the BPNS.

\begin{figure}

\begin{minipage}[t]{\linewidth}

{\centering 

\raisebox{-\height}{

\includegraphics{thesis_manuscript_files/figure-pdf/fig-adst308-1.pdf}

}

\caption{\label{fig-adst308}Raw depth log of tag 308 (female) in summer
2019 (subsampled to one depth value every 30 minutes) and acoustic
detections from the Permanent Belgian Acoustic Receiver Network
(PBARN).}

}

\end{minipage}%

\end{figure}

\hypertarget{sec-disc}{%
\section{Discussion}\label{sec-disc}}

The following chapter will critically reflect on the datasets and
analyses used, and interpret the findings in the context of already
existing ecological knowledge on \emph{M. asterias}. Firstly, the limits
of the telemetry methods utilised for this study and their resulting
datasets will be discussed (Section~\ref{sec-disc-limits}), followed by
a reflection on the chosen methods of analysis
(Section~\ref{sec-disc-reflection}). Thirdly, the results of this work
will be interpreted regarding present knowledge on movement ecology of
\emph{M. asterias}. The two research lines of this study encompass
assessing seasonality and sex-specificity regarding the presence of
\emph{M. asterias} in the BPNS and the Western Scheldt, as well as its
depth utilisation and five research questions were stated in
Section~\ref{sec-intro-aims}.
Section~\ref{sec-disc-seasonalpresencefemales} will cover the first line
of research (presence of \emph{M. asterias} in the BPNS and Western
Scheldt), and Section~\ref{sec-disc-dst-movementpatterns} will cover the
second line of research (depth utilisation). Section~\ref{sec-disc} will
finally discuss the potential of ADST as a novel tag technology for
research on \emph{M. asterias} (Section~\ref{sec-disc-adst}) and the
gained ecological knowledge on the species in the context of related
\emph{Mustelus} species (Section~\ref{sec-disc-otherspecies}).

\hypertarget{sec-disc-limits}{%
\subsection{Limits and biases of the methods
used}\label{sec-disc-limits}}

\hypertarget{sec-disc-tagging-effects}{%
\subsubsection{Tag recoveries and possible tagging
effects}\label{sec-disc-tagging-effects}}

Of 30 deployed ADST, eighth were recovered at the time of writing in
June, 2023. Out of those, 6 \emph{M. asterias} died within the first
month post-release. There are numerous factors that potentially
contribute to a negative effect of the tagging procedure. For instance,
the choice of tag to implant must be made considering the animal's body
mass. Generally, it is advised that the tag should not exceed 10 \% of
the individuals's body mass \citep{wagner_2011}. In the present study,
weight was recorded for 17 out of 30 tagged individuals. The lightest
individual of those sharks weighed 900 g, resulting in a tag to body
mass ratio of 1.4 \% and thereby being well under the proposed threshold
of 10 \%. In addition, \citet{smukall_2019} report the non-lethal
recovery of an acoustic tag 13 years post-release, implanted into a
female Lemon shark (\emph{Negaprion brevirostris}) at 119 cm TL. Thus,
it does not seem likely that the tag's weight had a severe negative
influence on the survival of the tagged individuals. Other possible
factors include insufficient asepsis of surgery tools and the tag
itself, or negative physiological responses due to handling
\citep{rub_2014}. The latter might include too much time outside of the
water, injuries from the fishing method (hook and line, in this case),
or the tag implantation taking too long. The cause of death for the the
6 \emph{M. asterias} individuals cannot be identified at this point.
Death due to predation by an endothermic mammal predator such as the
grey seal can be excluded, because the temperature did not substantially
rise \citep[to around 38 °C,][]{austin_2006} prior to the animals'
deaths (all raw depth an temperature logs are shown in \textbf{Annex A}.
Generally, the temperature logs of the short-term DST appear erroneous,
note for example that the temperature remains 21.2 °C for tag 319 at all
times between August 4 and 12, 2018. The possibility of predation by an
ectothermic predator cannot be ruled out at this point, since
temperature cannot provide any relevant insights but instead the depth
log would show extraordinary vertical movement \citep[see][ for examples
of predation by endothermic versus ectothermic predators]{seitz_2019}.
The possibility of predation by an ectothermic animal such as the blue
shark \emph{Prionace glauca} could be a next investigative step.

\hypertarget{sec-disc-datasetlimits}{%
\subsubsection{Limits of the datasets}\label{sec-disc-datasetlimits}}

Essentially, two types of dataset were used for study: Firstly, the
acoustic detections of the tagged sharks, and secondly, the depth logs
from recovered Data Storage Tags.

The 10 940 acoustic detections of 18 \emph{M. asterias} individuals over
a period of roughly two years are already a relevant dataset to start
with, but more tagging effort and thus more acoustic detections are
needed to gain further insights into the presence of the species in the
Scheldt estuary. For comparison, \citet{hereu_2023} tagged 33 European
seabass (\emph{Dicentrarchus labrax}), a seasonally migratory fish
\citep{pawson_2007}, and had 493 817 acoustic detections over roughly 2
years in an acoustic receiver network comprised of around 100 receivers
\citep{aspillaga_2017}. The substantial difference in acoustic
detections (approximately factor 50) indicates a difference in
explanatory power between the dataset of \emph{M. asterias} and the
dataset used in \citet{hereu_2023}.

Moreover, the gate-like receiver setup in the Western Scheldt is not
ideal to study horizontal distribution but rather serves to detect
animals that enter or leave the river Scheldt \citep{reubens_2019}. If
possible, additional receivers could be placed along the coast between
Dishoek and Zoetelande (see Figure~\ref{fig-mapWS}) to further resolve
the seasonal presence of adult females in the area and potentially gain
information on pupping grounds.

The availability of only long-term DST logs (that is, a log with more
than a year of data) poses the greatest limit of the second dataset.
More logs showing the annual migration should be analysed in the future
to assess whether the sex-specific patterns described in this study are
due to variation between individuals or due to the animal's sex. The
short-term DST logs were not further investigated in the present study
since altered behaviour resulting from the tagging procedure could not
be excluded. However, investigating the short-term DST logs could be
valuable to assess in the future.

An additional restriction of the archival data is the limit of the depth
sensor of 68 m. Since the female shark swam deeper than that limit
between October and December, 2018, limited inferences can be made from
the female's depth log during that period and the current results from
the geolocation modelling are therefore imprecise during that time
period. Potentially, this individual swam into the Hurd Deep in the
English Channel, or even into deeper waters in the Bay of Biscay
\citep[as hypothesised by][]{breve_2016}.

\hypertarget{sec-disc-reflection}{%
\subsection{Reflection of the chosen
analyses}\label{sec-disc-reflection}}

This work focused on the analysis of the acoustic detection and data
storage tag datasets. As such, the chosen analyses will be reflected
upon in the following.

\hypertarget{sec-disc-reflection-acoustic}{%
\subsubsection{Acoustic detections}\label{sec-disc-reflection-acoustic}}

For this work, the acoustic detection data were analysed in an
exploratory way. Heatmaps with the number of detections per receiver
station, individual and month proved to be a suitable method of
investigation. Furthermore, Residency Indices (RI) were calculated on
different scales (see Table~\ref{tbl-animalinfo} and
Figure~\ref{fig-heatmapOG10}). While providing a first simplistic
insight into the residency of an individual the reduction of complexity
has to be taken with caution. Since temporal patterns of residency
(i.e., a cluster of presence over several days can result in the same RI
as a sporadic presence over a long time with detections every few days)
are not resolved. Thus, it is argued that RI can be an auxiliary
variable to compare between individuals but should not be the only
analysis method \citep{kraft_2023}.

In the future, the detections of single females during summer could be
looked at in detail to assess if there are differences in presence
around certain receivers that could indicate pupping. A useful piece of
information for such investigations would be to assess the reproductive
status of tagged females (i.e., if the individual is pregnant or not)
upon release.

\hypertarget{sec-disc-reflection-acf}{%
\subsubsection{Autocorrelation}\label{sec-disc-reflection-acf}}

Autocorrelation analysis allows for a simplistic description of scales
of patterns found in time series. It is a helpful exploratory tool to
get an overview of the dataset \citep{dray_2010} and it moreover can
provide the base for the detection of behavioural switches
\citep{gurarie_2016}. While autocorrelation might not be the most
conclusive exploratory analysis, it proved to be a useful auxiliary tool
to explore the depth time series data and potentially confirm identified
periods with occurring behavioural switches.

\hypertarget{sec-disc-reflection-wavelet}{%
\subsubsection{Wavelet analysis}\label{sec-disc-reflection-wavelet}}

Conducting a continuous wavelet transform (CWT) and thereby resolving
prevalent periodicities in the time domain was the main analysis tool in
this study. Furthermore, performing CWT both on the raw depth log and
the daily summary statistics allowed for different scales of
periodicities and was able to provide further proof of patterns that
could already be seen in the daily summary statistics in
Figure~\ref{fig-dstsum308-1} and Figure~\ref{fig-dstsum321-1}. Wavelet
analysis is no ubiquitous analysis tool in movement ecology but has been
employed in the past \citep{wittemyer_2008, zhang_2020}. CWT was
preferred over Fourier Transformation (FT) as a focal method, due to its
ability to resolve periodicities in the time domain. While FT can also
be performed on subsets of the depth log, this manual separation
introduces a bias and within that subset, the periodicities are not
resolved in the time domain. The resolution in the time domain of CWT
comes at the cost of a lower resolution both in the time and frequency
domain. Thus, FT provides a first step of spectral analysis and can give
insights into generally existing patterns of periodicity (as shown in
\textbf{Annex A}), but CWT allows for unbiased assessment of change of
prevalent periodicities in time of time series data. In the wavelet
scalograms on the hourly scale, only periods in winter were
significantly present and are shown as fully opaque in the plots
(Figure~\ref{fig-waveletresults308-1} and
Figure~\ref{fig-waveletresults321-1}). In summer, periodicities of 12 h
for the female shark and 24 h of the male shark have a high prevalence
(indicated by a high power value and an orange/red colour in the
scalogram), albeit these are not significant. This is thought to be due
to the difference in magnitude of the periodicity during summer compared
to the winter period, for example, where overall vertical activity is
much higher.

\hypertarget{sec-disc-seasonalpresencefemales}{%
\subsection{\texorpdfstring{Presence of \emph{M. asterias} in the
Western Scheldt and BPNS (Research Line
1)}{Presence of M. asterias in the Western Scheldt and BPNS (Research Line 1)}}\label{sec-disc-seasonalpresencefemales}}

This section discusses the results of the acoustic detections (see
Section~\ref{sec-res-acoustic}) in the context of the three research
questions of the first line of research. Those involve seasonal patterns
(RQ 1), differences between sexes (RQ 2), and spatial preferences (RQ 3)
within the BPNS and the Western Scheldt (see
Section~\ref{sec-intro-aims}).

\hypertarget{sec-disc-acoustic-seasonality}{%
\subsubsection{Seasonality}\label{sec-disc-acoustic-seasonality}}

There are no acoustic detections at all between November and April in
all years with acoustic detections (2018 to 2020,
Figure~\ref{fig-abacus}). A tagged animal is detected with an estimated
probability of \textgreater70 \% within a 200 m radius around the
acoustic receiver \citep{reubens_2018}. Thus, no detections do not imply
that no tagged animal was present, but the consistency of no detections
between November and April throughout the whole study period suggests a
seasonal presence of \emph{M. asterias} in the Western Scheldt between
April and November, which goes in line with existing hypotheses on the
residency and migration behaviour of the species
\citep{breve_2016, breve_2020, griffiths_2020}. Most females (n = 7)
were detected in the months of July, August and September, in the WS1
array. This suggests that these months are especially relevant for
\emph{M. asterias} females in the Western Scheldt. \emph{M. asterias}
might leave the area of the Western Scheldt during winter because of
decreasing water temperature \citep{breve_2016}, a main predictor of
elasmobranch occurrence \citep{martin_2010}. The warmer waters in the
Channel area might be favourable especially for gestating females since
higher water temperatures increase metabolic rate and thereby improve
embryo development \citep{hurst_1999}.

\hypertarget{sex-specificity}{%
\subsubsection{Sex-specificity}\label{sex-specificity}}

97 \% of detections result from tags implanted into female \emph{M.
asterias}. Despite the higher number of females tagged (19 tagged
females in contrast to 11 tagged males), this still suggests higher
presence and habitat use of females in the receiver area compared to
males. Furthermore, not more than one male was ever detected in a month.
Three males were detected in the BPNS (one each on August, 2018, and in
April and May, 2019), and one male was detected in the WS1 array in
April, 2020. Three of those four individuals were detected in the year
following their release, indicating philopatric behaviour for males.
\citet{griffiths_2020} reported philopatry for a female \emph{M.
asterias} based on DST data. The return of the three males (tags 311,
315 and 316) into the BPNS and the WS1 area could be the first direct
reporting of philopatry for male \emph{M. asterias}.

\hypertarget{spatial-preferences}{%
\subsubsection{Spatial preferences}\label{spatial-preferences}}

According to the acoustic detections, a hotspot of seasonal presence of
mature \emph{M. asterias} females might be the area around the receiver
OG10 (66 \% of detections, and 10 out of 14 detected females, see
Figure~\ref{fig-heatmapOG10}). According to fishers, the area offshore
from Zoetelande is a \emph{M. asterias} fishing hotspot during summer,
so the coastal waters between Diskoek and Zoetelande are suggested to be
important for the species. Whether or not the potential aggregations of
mature females is related to parturition should be subject of further
studies. There are several indications, however, that point towards the
Western Scheldt and/or its surrounding waters to be a pupping area for
\emph{M. asterias}. Firstly, the Eastern Scheldt is a confirmed pupping
ground \citep{breve_2016}, and secondly, during the tagging for this
study in August, 2018, two neonates \citep[around 30 cm
TL,][]{farrell_2010} were caught \footnote{Verhelst, Pieterjan
  (\emph{personal communication}), April 25, 2023.}.

Shallow and sheltered coastal habitats are thought to be suitable
pupping grounds for sharks in general \citep{speed_2010} and for
\emph{M. asterias} in particular \citep{ellis_2004, ellis_2005}. This
description fits the Western Scheldt upstream past Flushing. However,
this area has very few detections of tagged \emph{M. asterias}
individuals. The low number of acoustically detected sharks (38
detections in the WS2 array throughout the whole study period from 2018
to 2020 from two individuals) indicates that the waters past Flushing
are not as frequently used as the waters around Dishoek. This could be
due to low salinities, or to anthropogenic influences like dredging.

\hypertarget{further-consideration-biannual-reproduction}{%
\subsubsection{Further consideration: biannual
reproduction}\label{further-consideration-biannual-reproduction}}

Another relevant fact to consider is the biannual reproductive cycle of
\emph{M. asterias} and whether that implies differing seasonal habitat
use between years of pregnancy and years without a pregnancy. Females
were either detected during many days (20 - 40) or few (5 - 10) days
during the summer of 2019 (see Figure~\ref{fig-heatmapOG10}). This could
indicate different behaviourial states between the females with many and
few acoustically detected days, further supported by the large range of
Residency Indices which are shown in the figure (the minimum RI of 0.005
differs from the maximum RI of 0.241 by a factor close to 50).

A relevant aspect to study in the future would be to explore reasons of
these seemingly different behaviours and whether the females could be
giving birth during their residential clusters at station OG10. As such,
it could be helpful to assess if adult females are pregnant at the time
of release, through ultrasonography or the use of sex steroid hormones,
both of which have already been employed to determine pregnancy in
sharks
\citep{awruch_2014, smukall_2019, anderson_2018, fujinami_2020, fujinami_2021}.
Generally, the Scheldt estuary can be considered a suitable pupping
habitat for \emph{M. asterias} due to high nutrient levels and food
availability of the Continental coastal water \citep[which enters the
estuary through tidal currents,][]{wolff_1973}, and monthly water
temperatures above 15 °C between June and September \citep{speed_2010}.
Assessing the reproductive stage of females upont tagging would thus
allow for further clarification if the Western Scheldt is used as a
pupping ground, and if it might be a relevant area for non-pregnant
females, too.

\hypertarget{sec-disc-dst-movementpatterns}{%
\subsection{\texorpdfstring{Depth utilisation of \emph{M. asterias}
(Research Line
2)}{Depth utilisation of M. asterias (Research Line 2)}}\label{sec-disc-dst-movementpatterns}}

The following section discusses the analysis results of the two
long-term DST logs (of tag 308, female, and tag 321, male, see
Section~\ref{sec-results-dst}) in the light of the second research line,
which addresses the depth utilisation of \emph{M. asterias}
(Section~\ref{sec-intro-aims}). The two research questions asked involve
seasonal patterns (RQ 4) and differences in depth use between sexes (RQ
5).

Overall, cyclicity can be observed for both the female (tag 308) and the
male (tag 321) shark logging data for more than a year. The
autocorrelograms (Figure~\ref{fig-acf}) show similar seasonal patterns
with an autocorrelation of zero at lags of 81 and 264 days for the
female, and 75 and 256 days for the male. The lags correspond to roughly
three and nine months, suggesting a change of vertical movement behavior
at these time scales. Global minima lie at lags of 170 and 138 days for
female and male, respectively. These lags approximately correspond to
five months and indicate that vertical behaviour is anti-correlated
within this time scale, further underlining the observation of higher
vertical activity and the use of deeper waters during winter and a
contrary behaviour with low vertical activity use of shallow waters
during summer. Local autocorrelation maxima lie at 300 and 345 days for
female and male, respectively. This corresponds to 10 to 12 months and
points towards a yearly cyclicity for both the male and the female
shark. In general, owing to the demersal lifestyle of \emph{M.
asterias}, shallower depths in the depth log indicate a shoreward
movement along the seafloor, as opposed to a vertical movement into the
water column. If, for example, the depth log shows shallower depths
during the night (as can be seen in Figure~\ref{fig-dst321-2}) this
suggests a movement into shallower, coastal waters to feed.

\hypertarget{sec-disc-summer}{%
\subsubsection{Summer}\label{sec-disc-summer}}

From both recovered DSTs that logged data for over a year, the daily
vertical range and median depth are shallower in summer (June --
September, around 17 m median depth for the female and around 10 m
median depth for the male, both with ranges between 0 and 20 m) than in
winter (December -- March). The wavelet scalograms of the raw depth logs
(Figure~\ref{fig-waveletresults308-1} and
Figure~\ref{fig-waveletresults321-1}) show a difference in periodicity
on the hourly scale between males and females. Both in the summer of
2018 and 2019, the female shows mainly a periodicity of 12 h (indicated
by band of high power in red colour), which is a pure tidal signal
\citep[since from one high tide to the next one it takes approximately
12 h, the same holds true for low tide,][]{shepard_2006}. In 2018, there
are some periodicities of 24 h (represented by yellow colours) as
opposed to 2019, where the power of the 24 h period is below zero. This
could be an indicator of the biannual reproductive cycle \citep[with
females feeding when they are not pregnant, and they stop feeding during
the migration to parturition grounds and parturition
itself,][p.~23]{michael_2006}. The 12 h periodicity, however, is the
most prevalent one in both summers for the female and thus implies that
the female itself expresses no to little vertical activity and is only
resting on the seafloor (observable in the example subset of the raw
depth log in Figure~\ref{fig-dst308-2}). The male, on the other hand,
shows a periodicity of 24 h. This can be seen as a high-power band in
red colours at the 24 h period in the wavelet scalogram on the hourly
scale (Figure~\ref{fig-waveletresults321-1}) and in
Figure~\ref{fig-dst321-2}. This suggests vertical migration, which has
been observed in several shark species, for example the spiny dogfish
\emph{Squalus acanthias} \citep{carlson_2014} and the basking shark
\emph{Cetorhinus maximus} \citep{shepard_2006}. Diel vertical migration
is believed to be related to foraging and feeding, following the nightly
vertical migration of zooplankton to feed \citep{griffiths_2020}.

\hypertarget{sec-disc-wintermig}{%
\subsubsection{Migration into winter habitat}\label{sec-disc-wintermig}}

From September 25 on for the female and October 21 for the male, the
daily median depth gets progressively deeper over a period of around 15
to 20 days, which potentially is the migration period of these
individuals. Interestingly, both individuals go to the maximum depth
logged by the tag within the logging period directly the potential
migration into the winter habitat. During this time, the female shark
goes to depths below the tag's measuring limit of 68 m, as indicated by
a straight line in the raw depth log (see Figure~\ref{fig-dst308-1}).
This provides further evidence of the species utilizing deeper waters
than previously thought, as argued by \citet{griffiths_2020} that found
a maximum recorded depth of 118 m in a female shark in December
\citep[S4]{griffiths_2020}. The male shark goes to depths between 40 and
45 m, both in 2018 and 2019. Since this tag logged for 485 days, the
migration into the winter habitat was recorded for two years. The start
date of the potential winter migration in 2019 differs to the one in
2018 by 10 days, temperatures range from around 13.5 °C to 15 °C in both
periods. The autocorrelation of the depth log of tag 321 has a local
maximum of 345 days see~\ref{fig-acf-2} which further reflects the
almost perfect yearly cyclicity of tag 321. Both individuals stay at
their respective maximum depths for approximately one month (the female
between the end of October until the end of November, 2018 and the male
between the start of November until the start of December, 2018). This
behaviour could be related to resting after the migration or escaping
potential predators. The behaviour after the completion of the migration
was not further assessed at this point.

\hypertarget{sec-disc-winter}{%
\subsubsection{Winter}\label{sec-disc-winter}}

During the winter months between December and March, both individuals
show roughly biweekly periodicities (see
Figure~\ref{fig-waveletresults308-2} and
Figure~\ref{fig-waveletresults321-2}). This vertical movement behaviour
could be linked to moon phases, which result in a 14-day cycle of neap
and spring tides \citep{shepard_2006}. The female shark shows a rather
biweekly vertical movement pattern during night and and a monthly
pattern for depth use during the day (Figure~\ref{fig-dstsum308-2} and
Figure~\ref{fig-dstsum308-3}). In the raw depth log, periods of resting
and subsequent periods of high vertical activity are visually
discernable (Figure~\ref{fig-dst308-1}). Both of these periodicities can
be seen in the wavelet scalogram at daily scale
(Figure~\ref{fig-waveletresults308-2}). During the night, this
individual goes to depths shallower than 10 m between December, 2018 and
May, 2019. The peaks in minimum depth align well with a full moon,
potentially reflecting visual predation behaviour coupled with diel
vertical migration, i.e., the shark moves to shallower depths during
full moon (when the moonlight is most intense), feeding on crustaceans
that follow the nocturnal zooplankton migration. Visual foraging and
preying of \emph{M. asterias} was already assumed by
\citet{griffiths_2020}. During the day, median and minimum depth are
deepest at full moon and new moon. These moon phases generally result in
the most intense tidal currents throughout a moon cycle
\citep{arnold_1994}. \citet{shepard_2006} found the same pattern of high
vertical activity at full and new moon in one tagged Basking shark
(\emph{Cetorhinus maximus}), and \citet{graham_2006} reported lunar
periodicity in vertical behaviour of one whale shark (\emph{Rhincodon
typus}). The exact strength of tidal currents, however, depends on the
bathymetry of the area, and since the location of the shark is not
exactly known, interpretations regarding tidal currents should be taken
cautiously. Since the female, however, is thought to have migrated into
the English Channel \citep[an area with exceptionally high tidal
currents,][]{salomon_1993}, it is a plausible explanation that this
individual was resting on the seafloor during periods of intense tidal
currents (as can be seen in Figure~\ref{fig-dst308-3}), and moving to
shallower waters to visually feed at full moon.

For the male shark, the biweekly periodicity is less present, in
general, and no apparent periodicity can be visually detected during the
night (Figure~\ref{fig-dstsum321-2} and Figure~\ref{fig-dstsum321-3}).
This could be an indication of a different feeding behaviour of the
male, involving less visual predation. The hourly wavelet scalogram
(Figure~\ref{fig-waveletresults321-1}) shows the presence of 24 h
periodicities in January and February, 2019, but the daily wavelet
scalogram (Figure~\ref{fig-waveletresults321-2}) does not contain
significantly present monthly periodicities. This suggests that the male
shark relies less on the moonlight to predate than the female shark. The
significant presence of 14 day periods between December, 2018 and May,
2019, however, could potentially relate to tidal currents. This is less
visually observable from the median and minimum depth during the day
(Figure~\ref{fig-dstsum321-2}). The male shark is thought to move
northwards from the Scheldt estuary during its winter migration
(Figure~\ref{fig-mapgeolocation-2}) where tidal currents are generally
less intense than in the English Channel
\citep{salomon_1993, arnold_1994}. This could be a potential explanation
for the weaker biweekly periodicity in the male shark's median and
minimum depth. The male does not show clearly detectable periods with
pure resting behaviour during winter (as shown in
Figure~\ref{fig-dst321-3}). Whether this difference in vertical
behaviour patterns is sex-specific and related to a potential gestation
of the female shark, or to different feeding modes, could be a point of
further investigation in the future.

In general, the effect of the moon illumination and tidal currents would
be interesting to study further since this can give relevant insights
about (potentially sex-specific) winter behaviour of \emph{M. asterias},
for example through comparing wavelet scalograms from median depth
during the day and during the night.

\hypertarget{sec-disc-summermig}{%
\subsubsection{Migration into summer habitat}\label{sec-disc-summermig}}

In spring, roughly between April and May, the daily median depth gets
shallower in a stepwise manner for both individuals which potentially
reflects the migration into the summer habitat. The female (tag 308) was
detected in the receiver areas of the BPNS and WS1 in the year following
its release, proving the individual's return, and providing further
evidence of philopatric behaviour for the species, as suggested by
\citet{breve_2016} and first directly reported by
\citet{griffiths_2020}. The start and duration of the summer migration
cannot be identified as clearly from the daily median depth as the
winter migration since the depth does substantially changes over a
period of few weeks. From the wavelet scalograms of the raw depth log
(Figure~\ref{fig-waveletresults308-1} and
Figure~\ref{fig-waveletresults321-1}), the time of potential migration
into the summer habitat shows a band of high power throughout periods
from 4 to 64 hours (i.e., high prevalence of these periods), albeit this
does not differ significantly from white noise (and is thus slightly
transparent). Since the winter period also involves vertical activity,
there is no clearly visible change in periodicity patterns between
winter months up until March and the months of April and May, during
which the daily median depth decreases stepwise for the female shark.
During April, 2019, the male shark does not show any prevailing
periodicity patterns in the wavelet scalogram at hourly scale
(Figure~\ref{fig-waveletresults321-1}). The month of May 2019, however,
shows prevalent periodicities between 12 and 64 h. The absence of
clearly discernable patterns in perodicities suggest a widely varying,
seemingly stochastic depth utilisation which potentially indicates
migration behaviour. Decreased vertical activity compared to the winter
months from mid-May onwards is observable for the female and from June
on for the male, indicating that the individuals might have arrived at
their summer habitat by then. At this point, knowledge about the exact
spectral signature of \emph{M. asterias} migration behaviour is lacking
and it is suggested to be further studied. Moreover, a more detailed
investigation of the difference in vertical movement between the
migration into the winter and the summer habitat could be an interesting
focus for future investigations.

\hypertarget{sec-disc-adst}{%
\subsection{\texorpdfstring{The potential of ADST for \emph{M.
asterias}}{The potential of ADST for M. asterias}}\label{sec-disc-adst}}

For the present study, a novel type of electronic combination tag was
utilized, the Acoustic Data Storage Tag, combining acoustic detections
with data storage logs. Here, only one tag (tag 308, female) showed data
overlap that allowed for relevant ecological interpretations.
Figure~\ref{fig-adst308} shows that tag 308 was detected at the
Birkenfels receiver station (which is in the BPNS) 63 days after the DST
depth log stopped. This suggests the departure of the individual from
the BPNS and the Scheldt estuary and aligns almost perfectly with the
depth log of that time one year before, where the female went to waters
deeper than 25 m from October 06, 2018, onwards. While the annual
cyclicity for the male could already be shown from its depth log since
it logged well over a year (see Section~\ref{sec-disc-wintermig}), the
detection of tag 308 at the Birkenfels station points towards an almost
perfect annual cyclicity for this individual, too. This example shows
the complementary potential of the nature behind combination tags. It is
unfortunate that from the tags deployed for this study only one reaches
this potential but that does not minimise the potential of the
technology in general. therefore, ADST can be considered a well-suited
tracking technology for \emph{M. asterias}, owing to the species'
seasonally differing resident and migratory behaviour. ADST allow for
gaining high-resolution presence/absence data within the areas of
receiver networks and insights about the vertical movement of an animal
independent of the acoustic detections, given that the tag is retrieved.
Moreover, the acoustic detections provide punctual ground truth for
geolocation modelling. This is especially helpful in highly mixed areas
with homogeneous bathymetry like the Scheldt estuary, as such areas
generally result in limited reliability of geolocation models since the
models cannot make out a definite position by depth and temperature if
these variables are roughly constant throughout the area.

\hypertarget{sec-disc-otherspecies}{%
\subsection{\texorpdfstring{Gained ecological knowledge on \emph{M.
asterias}}{Gained ecological knowledge on M. asterias}}\label{sec-disc-otherspecies}}

\emph{Mustelus asterias} is a common species in the North Sea and the
English Channel, both of which areas experience high anthropogenic
impact \citep{hernandezfarinas_2014}. Since knowledge about the species
and its relatives is still limited, the following section aims to give a
brief overview of how the knowledge gathered in this study fits into
already existing ecological knowledge about \emph{M. asterias} and
related species.

Comparisons about reproductive behaviour must be made cautiously since
not all members of the genus \emph{Mustelus} exhibit aplacental
viviparity like \emph{M. asterias}. \emph{M. asterias}, for instance,
was often mistaken for the common smooth-hound \emph{M. mustelus} before
the development of a simple genetic method to distinguish the two
species \citep{farrell_2009}, yet this relative is placentally
viviparous \citep{dasilva_2018}. The gummy shark \emph{M. antarcticus}
inhabits Australian waters and exhibits matrotrophic aplacental
viviparity like \emph{M. asterias}, making it well comparable
\citep{walker_2007}. \citet{walker_2007} uncovered asynchronous breeding
cycles (of one and two years) related to different locations for
\emph{M. antarcticus}. This behaviour is also known to occur in Squalid
sharks \citep{braccini_2006}, but there is no evidence of \emph{M.
asterias} exhibiting different breeding cycle lengths. The reproductive
biology of \emph{M. asterias}, however, involves complex characteristics
like embryo asynchronism, for instance. This refers to pups with
different developmental stages, potentially resulting from sperm storage
and selective fertilization by the females \citep{farrell_2010a}. While
an in-depth stock assessment has already been carried out for \emph{M.
antarcticus} by \citet{pribac_2005}, this is still lacking for \emph{M.
asterias} and should be carried out in the future
\citep{mccullyphillips_2015}.

Annual migration behaviour has not been investigated in depth for many
\emph{Mustelus} species. In fact, the gummy shark \emph{M. antarcticus}
is reported to be less mobile than other elasmobranchs in the area of
Western Australia \citep[according to a three-year acoustic monitoring
study involving 100 tagged \emph{M. antarcticus},][]{braccini_2017}, but
still able to cover distances of \textgreater{} 60 km per day, and
overall distances of almost 1000 km. The narrownose smooth-hound shark
\emph{Mustelus schmitti}, an aplacentally viviparious relative of
\emph{M. asterias} that inhabits waters in the southwest Atlantic, is
thought to exhibit similar seasonal migration patterns as \emph{M.
asterias}. \citet{elisio_2019} argue that although migration patterns of
\emph{M. schmitti} are still unknown, large-sized individuals are
increasingly abundant in deeper waters during autumn and winter
\citep{cortes_2011}, suggesting a seasonal migration to shallower
waters. The same study reported parturition events of \emph{M. schmitti}
in coastal waters during spring and summer, following warming
temperatures above 16 °C. These findings align well with conclusions
from \citet{breve_2016} that reported decreased presence of \emph{M.
asterias} in the Dutch Delta for temperatures \textless{} 13 °C. The
water temperature of the two long-term DST logs from the present study
is slightly above 16 °C for the female and 15 °C for the male in October
2018, when the two individuals are thought to start migrating into the
winter habitat, further supporting the findings of \citet{breve_2016}
and \citet{elisio_2019}.

Overall, the results of the present study go in line with already
gathered knowledge on \emph{M. asterias}. The geolocation model results
suggest that the male shark spends the winters in waters north of the
Scheldt estuary and the female goes into the English Channel
(Figure~\ref{fig-mapgeolocation}), as assumed by \citet{breve_2020} and
\citet{griffiths_2020}. Different overwintering spots for the female and
the male shark are further supported by the presumably stronger effect
of moon phases and tidal currents on the vertical activity of the female
\emph{M. asterias}, potentially sheltering from the intense tidal
currents in the English Channel. The present study, however, only
involved year-long DST logs from two individuals, so any inferences from
this data have to be taken cautiously. The high return rate (67 \%) of
individuals into the Western Scheldt as shown by the acoustic detections
in the acoustic receiver array further underlines philopatric behaviour
in regard to the species' summer habitat, the Southern North Sea
\citep{breve_2016, breve_2020, griffiths_2020}.

\hypertarget{conclusion}{%
\section{Conclusion}\label{conclusion}}

To conclude, the findings of this study in relation to the five research
questions stated in the introduction (see Section~\ref{sec-intro-aims})
will be presented. The research questions aim to investigate the
presence of \emph{M. asterias} in the Western Scheldt and the BPNS
(research line 1) regarding seasonality (RQ 1), sex-specificity (RQ 2)
and spatial preferences (RQ 3), as well as depth utilisation of the
species (research line 2) regarding seasonality (RQ 4) and
sex-specificity (RQ 5).

This work presents the first in-depth investigation of both the
seasonality of vertical behaviour throughout the annual movement cycle
and the presence of \emph{M. asterias} in the Western Scheldt. Adding to
already existing knowledge, the seasonal presence of mostly adult female
Starry smooth-hounds around the Western Scheldt could be confirmed,
proving the importance of the area as a summer habitat. Males
potentially are less present in the Western Scheldt, but more present in
the BPNS than females. Moreover, the high return rates of individuals
(both males and females) into the Western Scheldt and the BPNS provide
further evidence for philopatric behaviour of the species (Research
Question 1, 2 and 3).

Although males and females most likely overwinter in different areas
\citep{breve_2016, griffiths_2020}, their seasonal vertical movement
behaviour appears to be similar (involving generally less vertical
activity during summer and higher vertical activity during winter),
according to the two DST tags that were subject of this study. During
summers, the female primarily showed resting behaviour while the male
exhibited both resting and feeding behaviour, indicating sex-specificity
in behavioural states on a within-season scale. During winter, the
female's (and to a lesser extent, the male's) vertical activity is
potentially influenced by the moon phase which might be linked to the
intensity of tidal currents (Research Question 4 and 5).

Thus, this study could gain new insights into the seasonal distribution
and potential sex-specificity of horizontal and vertical movement
behaviour of \emph{Mustelus asterias}. There still is, however, a
substantial knowledge gap about the effects of sex and life stage on the
species' distribution \citep{griffiths_2020} and to be able to establish
a successful species management plan in the future, further studies are
needed
\citep{mccullyphillips_2015, breve_2016, breve_2020, griffiths_2020}.

\hypertarget{outlook}{%
\section{Outlook}\label{outlook}}

\hypertarget{tagging-studies-involving-m.-asterias}{%
\subsection{\texorpdfstring{Tagging studies involving \emph{M.
asterias}}{Tagging studies involving M. asterias}}\label{tagging-studies-involving-m.-asterias}}

As discussed in Section~\ref{sec-disc-seasonalpresencefemales}, an
insightful addition to future tagging studies involving adult \emph{M.
asterias} would be to assess the females' reproductive stage upon
sampling. This could be done either through ultrasonography or the use
of sex steroid hormones
\citep{awruch_2014, smukall_2019, anderson_2018, fujinami_2020, fujinami_2021}.
A new tagging technology was launched recently: The Birth Alert Tag
\citep[BAT,][]{sulikowski_2023}, which is, however, only advised to be
used on ``large sharks''. If smaller BATs are developed in the future,
this might be a suitable tag to use on \emph{M. asterias}. In addition,
the placement of an acoustic receiver off the coast of Zoetelande might
give further insights into possible seasonal aggregation of \emph{M.
asterias} in summer, as described by fishers \footnote{Verhelst,
  Pieterjan (\emph{personal communication}), April 25, 2023.}. If
possible, DST with a wider depth range should be utilized for future
studies, since evidence of \emph{M. asterias} swimming in deeper waters
(\textgreater{} 100 m depth) than previously thought is increasing
\citep{ices_2019, griffiths_2020}. Generally, the effect of the
implantation of tags on the sharks should be investigated. Since female
\emph{M. asterias} have been previously reported to abort and expel
embryos when caught \citep{farrell_2010a}, special attention should be
drawn to the tagging effect on pregnant females, especially when
intending to study their pupping behaviour.

\hypertarget{analysis-of-data-storage-tag-logs}{%
\subsection{Analysis of Data Storage Tag
logs}\label{analysis-of-data-storage-tag-logs}}

The adequate analysis of DST logs and the development of well-performing
geolocation models remains a challenge. While the input of behavioural
states (i.e., high or low vertical activity, or residential versus
migrating behaviour) would provide improved input for a geolocation
model, segmenting the depth log in different behavioural states is not
trivial.

A first step to improve the geolocation model without adding behavioural
states could be the addition of a simple boundary condition that
reflects the demersal lifestyle of \emph{M. asterias}. This boundary
condition entails that the maximum depth logged by the tag each day
(given that the resolution of the geolocation model is in the unit of
days) is equal to the maximum depth of the seafloor at the location of
the shark. This would prevent model outputs that estimate the
individual's position on a day at locations with substantially deeper
seafloor depth than the shark's depth log has recorded that day.

Furthermore, wavelet analysis proved to be a helpful tool for assessing
the seasonal changes in spectral composition of the depth signal for
both individuals. This technique is suggested to be further taken into
account in the future, for instance regarding the segmentation of the
depth log into different behavioural states \citep[as already shown
by][]{soleymani_2017}. Since the implementation of simple CWT in this
study is still an exploratory technique it should be investigated how it
could be quantified in the future. If different behavioural states, for
example resting, feeding and migrating can be linked to specific
spectral signatures (i.e., a 12 h period for resting, a 24 h period for
feeding and high prevalence of periods from 4h to 64 h for migrating),
then wavelet coherence could be a next analysis step. This analysis
assesses the similarity of two wavelet scalograms \citep{grinsted_2004}.
Conducting wavelet coherence between the raw depth logs of the sharks
and characteristic spectral signatures of different behaviours could
quantify the presence of different behavioural states in the future.
Several methods to quantify behavioural states exist already
\citep{pedersen_2008, heerah_2017} but there is not one generic method
that works for all species equally. Thus, it would be valuable to test
the method suggested here and compare it to outputs from already
existing methods. This would be a next step towards improving the
geolocation modelling for \emph{M. asterias} and consequently, our
knowledge and understanding of the species.

\hypertarget{availability-of-data-and-materials}{%
\section*{Availability of data and
materials}\label{availability-of-data-and-materials}}
\addcontentsline{toc}{section}{Availability of data and materials}

The dataset supporting the conclusions of this article is available in
the Integrated Marine Information System (IMIS) with the Digital Object
Identifier (https://doi.org/10.14284/605). The code developed to conduct
the analyses, make all visualisations (both figures and maps), and to
write the manuscript are available on
(https://github.com/lottepohl/MasterThesis\_LottePohl).

The dataset supporting the conclusions of this article is furthermore
included within the additional files of this manuscript.

The metadata that support the raw data for this study can be found in
\textbf{Annex C}.

\newpage{}

\hypertarget{acknowledgements}{%
\section*{Acknowledgements}\label{acknowledgements}}
\addcontentsline{toc}{section}{Acknowledgements}

This work was supported by the Research Foundation Flanders (FWO) as
part of the Belgian contribution to LifeWatch. I am indebted to everyone
involved in the tagging of the 30 Starry smooth-hounds in 2018 and 2019
without whom this study would not have been possible. My gratitude
extends to everyone involved in the maintenance of the acoustic receiver
network involved in this study. This includes the funding bodies and the
crew of the RV Simon Stevin. Furthermore, I appreciate the effort by
everyone involved in the return of the tags, enabling me to study the
annual vertical movement of the Starry smooth-hounds in the first place.
I am thankful to the Flemish Marine Institute (VLIZ) for having granted
me the opportunity to conduct my master thesis with them, and the
IMBRSea coordinators for their continuous effort to make this study
programme happen.

I wish to express my heartfelt thanks to my promotors Jan Reubens and
Niels Brevé, and my supervisor Carlota Muñiz. Thank you Jan, for
providing me with constructive and valuable feedback, thank you Niels
for your invaluable input on the Starry smooth-hound, and thank you
Carlota for guiding me and looking after me when I was a bit stuck.
Furthermore, I am grateful for having been able to go out into the
Belgian Part of the North Sea for receiver maintenance work, this
experience gave me valuable insights into the practicalities behind the
receiver network. My sincere appreciation goes to Pieterjan Verhelst,
for taking me along the Twaite Shad tagging campaign and the car
conversations about fish migrations and fishers's knowledge. Thank you
to Jolien Goossens for carrying out the geolocation modelling of the
returned tags which provided an important base for me to follow up on.

I furthermore wish to thank my IMBRSea study mates for productive
co-working sessions and emotional support throughout the thesis. My
thesis would not have been the same without all the warmhearted
colleagues at VLIZ, thank you all for this extraordinary time in Gent
and Ostend. My eternal gratitude goes to the people that I deeply care
for, especially Silvi and Tobi, for emotional support and understanding
during this intense period. And lastly, endless thanks to my parents for
being there and supporting me no matter what.

\newpage{}

\hypertarget{references}{%
\section*{References}\label{references}}
\addcontentsline{toc}{section}{References}

\renewcommand{\bibsection}{}
\bibliography{My_Collection.bib}

\newpage{}

\hypertarget{annex}{%
\section*{Annex}\label{annex}}
\addcontentsline{toc}{section}{Annex}

\hypertarget{annex-a-raw-depth-and-temperature-logs}{%
\subsection*{\texorpdfstring{\textbf{Annex A}: Raw depth and temperature
logs}{Annex A: Raw depth and temperature logs}}\label{annex-a-raw-depth-and-temperature-logs}}
\addcontentsline{toc}{subsection}{\textbf{Annex A}: Raw depth and
temperature logs}

Figure~\ref{fig-dsttempannex} shows the raw temperature logs of tags 308
(f) and 321 (m). Tag 308 experienced the maximum temperature of 22.9 °C
on August 06, 2018 and its minimum temperature of 8.65 °C on March 23,
2019. Tag 321 experienced its maximum temperature of 22.1 °C on August
07, 2018 and its minimum temperature of 5.98 °C on February 04, 2019.

\begin{figure}

\begin{minipage}[t]{\linewidth}

{\centering 

\raisebox{-\height}{

\includegraphics{thesis_manuscript_files/figure-pdf/fig-dsttempannex-1.pdf}

}

}

\subcaption{\label{fig-dsttempannex-1}Tag 308 (f).}
\end{minipage}%
\newline
\begin{minipage}[t]{\linewidth}

{\centering 

\raisebox{-\height}{

\includegraphics{thesis_manuscript_files/figure-pdf/fig-dsttempannex-2.pdf}

}

}

\subcaption{\label{fig-dsttempannex-2}Tag 321 (m).}
\end{minipage}%

\caption{\label{fig-dsttempannex}Temperature logs from the long-term
DST.}

\end{figure}

Figure~\ref{fig-dstannex} displays the raw depth and temperature logs
from the short-term DST.

\begin{figure}

\begin{minipage}[t]{0.50\linewidth}

{\centering 

\raisebox{-\height}{

\includegraphics{thesis_manuscript_files/figure-pdf/fig-dstannex-1.pdf}

}

}

\subcaption{\label{fig-dstannex-1}Tag 295. This male was tagged on July
19, 2018.}
\end{minipage}%
%
\begin{minipage}[t]{0.50\linewidth}

{\centering 

\raisebox{-\height}{

\includegraphics{thesis_manuscript_files/figure-pdf/fig-dstannex-2.pdf}

}

}

\subcaption{\label{fig-dstannex-2}Tag 304. This female was tagged on
July 11, 2019.}
\end{minipage}%
\newline
\begin{minipage}[t]{0.50\linewidth}

{\centering 

\raisebox{-\height}{

\includegraphics{thesis_manuscript_files/figure-pdf/fig-dstannex-3.pdf}

}

}

\subcaption{\label{fig-dstannex-3}Tag 310. This male was tagged on July
11, 2019.}
\end{minipage}%
%
\begin{minipage}[t]{0.50\linewidth}

{\centering 

\raisebox{-\height}{

\includegraphics{thesis_manuscript_files/figure-pdf/fig-dstannex-4.pdf}

}

}

\subcaption{\label{fig-dstannex-4}Tag 312. This female was tagged on
July 12, 2019.}
\end{minipage}%
\newline
\begin{minipage}[t]{0.50\linewidth}

{\centering 

\raisebox{-\height}{

\includegraphics{thesis_manuscript_files/figure-pdf/fig-dstannex-5.pdf}

}

}

\subcaption{\label{fig-dstannex-5}Tag 319. This female was tagged on
July 19, 2018.}
\end{minipage}%
%
\begin{minipage}[t]{0.50\linewidth}

{\centering 

\raisebox{-\height}{

\includegraphics{thesis_manuscript_files/figure-pdf/fig-dstannex-6.pdf}

}

}

\subcaption{\label{fig-dstannex-6}Tag 322. This female was tagged on
July 19, 2018.}
\end{minipage}%

\caption{\label{fig-dstannex}Temperature and depth logs from the
short-term DST.}

\end{figure}

\hypertarget{annex-b-fft-periodograms}{%
\subsection*{\texorpdfstring{\textbf{Annex B}: FFT
periodograms}{Annex B: FFT periodograms}}\label{annex-b-fft-periodograms}}
\addcontentsline{toc}{subsection}{\textbf{Annex B}: FFT periodograms}

The resulting periodograms from the Fast Fourier Transform for tags 308
(\emph{f}) and 321 (\emph{m}) are depicted in Figure~\ref{fig-fft}.

Figure~\ref{fig-fft-1} shows spectral density peaks at about 12 hours
(highest), 24 hours, and smaller peaks at 23, 26, 6 and 8 hours. Tag 321
has the highest spectral density at 24 hours, then 12 hours, and small
peaks at 6, 23, 25 and 25 hours, see Figure~\ref{fig-fft-2}.

\begin{figure}

\begin{minipage}[t]{\linewidth}

{\centering 

\raisebox{-\height}{

\includegraphics{thesis_manuscript_files/figure-pdf/fig-fft-1.pdf}

}

}

\subcaption{\label{fig-fft-1}Tag 308 (f), 366 days of liberty.}
\end{minipage}%
\newline
\begin{minipage}[t]{\linewidth}

{\centering 

\raisebox{-\height}{

\includegraphics{thesis_manuscript_files/figure-pdf/fig-fft-2.pdf}

}

}

\subcaption{\label{fig-fft-2}Tag 321 (m), 485 days of liberty.}
\end{minipage}%

\caption{\label{fig-fft}Periodograms from the full depth log of
recovered data storage tags.}

\end{figure}

\newpage{}

\hypertarget{annex-c-metadata}{%
\subsection*{\texorpdfstring{\textbf{Annex C}:
Metadata}{Annex C: Metadata}}\label{annex-c-metadata}}
\addcontentsline{toc}{subsection}{\textbf{Annex C}: Metadata}

Four sets of tabular data form the basis of this study. These can be
found in the additional files of this study. A brief description of the
dataset and its variables is given below. Where available, the variable
descriptions were obtained from the ETN database
(https://www.lifewatch.be/etn/).

\hypertarget{information-on-tagged-individuals}{%
\subsubsection*{Information on tagged
individuals}\label{information-on-tagged-individuals}}
\addcontentsline{toc}{subsubsection}{Information on tagged individuals}

The dataset includes information on the 30 \emph{M. asterias}
individuals that were subject of this study (filename:
\texttt{tagged\_individuals\_raw\_data.csv}). The metadata linked to
this dataset is displayed in Table~\ref{tbl-taggedindividualsmetadata}.

\hypertarget{tbl-taggedindividualsmetadata}{}
\begin{longtable}[]{@{}
  >{\raggedright\arraybackslash}p{(\columnwidth - 2\tabcolsep) * \real{0.2000}}
  >{\raggedright\arraybackslash}p{(\columnwidth - 2\tabcolsep) * \real{0.8000}}@{}}
\caption{\label{tbl-taggedindividualsmetadata}Metadata supporting the
dataset \texttt{tagged\_individuals\_raw\_data.csv}.}\tabularnewline
\toprule\noalign{}
\begin{minipage}[b]{\linewidth}\raggedright
variable
\end{minipage} & \begin{minipage}[b]{\linewidth}\raggedright
explanation
\end{minipage} \\
\midrule\noalign{}
\endfirsthead
\toprule\noalign{}
\begin{minipage}[b]{\linewidth}\raggedright
variable
\end{minipage} & \begin{minipage}[b]{\linewidth}\raggedright
explanation
\end{minipage} \\
\midrule\noalign{}
\endhead
\bottomrule\noalign{}
\endlastfoot
release\_date\_time & Date and time of the release of the animal, in
YYYY-MM-DD HH:MM:SS format and UTC time. \\
tag\_serial\_number & The serial number that is unique to the Acoustic
Data Storage Tag. \\
release\_latitude & Longitude of release location, in decimal degress.
Note: in the western hemisphere all longitudes must be negative. \\
release\_longitude & Longitude of release location, in decimal degress.
Note: in the western hemisphere all longitudes must be negative. \\
sex & Sex of the animal. f = female, m = male. \\
weight & Bodymass of the animal carrying the tag. In Kilogram. \\
length1 & Total length of animal carrying the tag. In Metres. \\
recapture\_date\_time & If applicable, date of the recapture of the
tagged animal. In YYYY-MM-DD format, in UTC time. \\
\end{longtable}

\hypertarget{acoustic-detections}{%
\subsubsection*{Acoustic detections}\label{acoustic-detections}}
\addcontentsline{toc}{subsubsection}{Acoustic detections}

The dataset includes the acoustic detections of the 30 \emph{M.
asterias} individuals by the acoustic receivers that are part of the
Permanent Belgian Acoustic Receiver Network (filename:
\texttt{acoustic\_detections\_raw\_data.csv}). The metadata linked to
this dataset is displayed in Table~\ref{tbl-detectionsmetadata}.

\hypertarget{tbl-detectionsmetadata}{}
\begin{longtable}[]{@{}
  >{\raggedright\arraybackslash}p{(\columnwidth - 2\tabcolsep) * \real{0.2000}}
  >{\raggedright\arraybackslash}p{(\columnwidth - 2\tabcolsep) * \real{0.8000}}@{}}
\caption{\label{tbl-detectionsmetadata}Metadata supporting the dataset
\texttt{acoustic\_detections\_raw\_data.csv}.}\tabularnewline
\toprule\noalign{}
\begin{minipage}[b]{\linewidth}\raggedright
variable
\end{minipage} & \begin{minipage}[b]{\linewidth}\raggedright
explanation
\end{minipage} \\
\midrule\noalign{}
\endfirsthead
\toprule\noalign{}
\begin{minipage}[b]{\linewidth}\raggedright
variable
\end{minipage} & \begin{minipage}[b]{\linewidth}\raggedright
explanation
\end{minipage} \\
\midrule\noalign{}
\endhead
\bottomrule\noalign{}
\endlastfoot
detection\_id & Unique identifier of the detection event. \\
date\_time & Date and time in YYYY-MM-DD HH:MM:SS format. In UTC
time. \\
tag\_serial\_number & The serial number that is unique to the Acoustic
Data Storage Tag. \\
scientific\_name & Scientific name of the animal that carries the
tag. \\
station\_name & Name of the station where the deployment of the receiver
takes place. Related to a specific latitude and longitude. \\
deploy\_latitude & Latitude of the actual deployment location, in
decimal degrees. Note: in the southern hemisphere all latitudes must be
negative. \\
deploy\_longitude & Longitude of the actual deployment location, in
decimal degrees. Note: in the western hemisphere all longitudes must be
negative. \\
parameter & Value of one sensor transmitted to the acoustic receiver at
the time of detection. \\
sensor\_unit & Unit of the sensor at stake. \\
sensor\_type & Type of tag sensor. Predefined options: pressure,
temperature, acceleration. \\
acoustic\_tag\_id & Unique identifier of each sensor within the acoustic
tag. One ID for the pressure sensor, and one ID for the temperature
sensor per ADST. \\
\end{longtable}

\hypertarget{data-storage-tag-logs}{%
\subsubsection*{Data Storage Tag logs}\label{data-storage-tag-logs}}
\addcontentsline{toc}{subsubsection}{Data Storage Tag logs}

The dataset includes the depth and temperature logs of all recovered
data storage tags (filename: \texttt{DST\_logs\_raw\_data.csv}). The
metadata linked to this dataset is displayed in
Table~\ref{tbl-dstlogmetadata}.

\hypertarget{tbl-dstlogmetadata}{}
\begin{longtable}[]{@{}
  >{\raggedright\arraybackslash}p{(\columnwidth - 2\tabcolsep) * \real{0.2000}}
  >{\raggedright\arraybackslash}p{(\columnwidth - 2\tabcolsep) * \real{0.8000}}@{}}
\caption{\label{tbl-dstlogmetadata}Metadata supporting the dataset
\texttt{DST\_logs\_raw\_data.csv}.}\tabularnewline
\toprule\noalign{}
\begin{minipage}[b]{\linewidth}\raggedright
variable
\end{minipage} & \begin{minipage}[b]{\linewidth}\raggedright
explanation
\end{minipage} \\
\midrule\noalign{}
\endfirsthead
\toprule\noalign{}
\begin{minipage}[b]{\linewidth}\raggedright
variable
\end{minipage} & \begin{minipage}[b]{\linewidth}\raggedright
explanation
\end{minipage} \\
\midrule\noalign{}
\endhead
\bottomrule\noalign{}
\endlastfoot
date\_time & Date and time in YYYY-MM-DD HH:MM:SS format. In UTC
time. \\
tag\_serial\_number & The serial number that is unique to the Acoustic
Data Storage Tag. \\
depth\_m & The recorded depth of the depth sensor in Metres. \\
temp\_c & The recorded temperature of the temperature sensor in degrees
Celcius. \\
\end{longtable}

\hypertarget{geolocation-modelling-output}{%
\subsubsection*{Geolocation modelling
output}\label{geolocation-modelling-output}}
\addcontentsline{toc}{subsubsection}{Geolocation modelling output}

The dataset includes the most probable tracks calculated by the
geolocation model \citep[the reader is referred to][ for further
explanation of the model, filename:
\texttt{geolocation\_output\_raw\_data.csv}]{goossens_2023}. The
metadata linked to this dataset is displayed in
Table~\ref{tbl-geolocationmetadata}. The geolocation modelling was
carried out by Jolien Goossens (ORCID:
\href{https://orcid.org/0000-0002-0853-9153}{0000-0002-0853-9153}).

\hypertarget{tbl-geolocationmetadata}{}
\begin{longtable}[]{@{}
  >{\raggedright\arraybackslash}p{(\columnwidth - 2\tabcolsep) * \real{0.2000}}
  >{\raggedright\arraybackslash}p{(\columnwidth - 2\tabcolsep) * \real{0.8000}}@{}}
\caption{\label{tbl-geolocationmetadata}Metadata supporting the dataset
\texttt{geolocation\_output\_raw\_data.csv}.}\tabularnewline
\toprule\noalign{}
\begin{minipage}[b]{\linewidth}\raggedright
variable
\end{minipage} & \begin{minipage}[b]{\linewidth}\raggedright
explanation
\end{minipage} \\
\midrule\noalign{}
\endfirsthead
\toprule\noalign{}
\begin{minipage}[b]{\linewidth}\raggedright
variable
\end{minipage} & \begin{minipage}[b]{\linewidth}\raggedright
explanation
\end{minipage} \\
\midrule\noalign{}
\endhead
\bottomrule\noalign{}
\endlastfoot
date\_time & Date and time in YYYY-MM-DD format. In UTC time. \\
tag\_serial\_number & The serial number that is unique to the Acoustic
Data Storage Tag. \\
detection\_latitude & Latitude that was calculated as the `most probable
track' in the geolocation model. See Goossens et al., (2023) and Woillez
et al., (2016) for details. \\
detection\_longitude & Longitude that was calculated as the `most
probable track' in the geolocation model. See Goossens et al., (2023)
and Woillez et al., (2016) for details. \\
\end{longtable}




\end{document}
