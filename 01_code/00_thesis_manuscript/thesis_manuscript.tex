% Options for packages loaded elsewhere
\PassOptionsToPackage{unicode}{hyperref}
\PassOptionsToPackage{hyphens}{url}
\PassOptionsToPackage{dvipsnames,svgnames,x11names}{xcolor}
%
\documentclass[
  authoryear,
  review,
  3p]{elsarticle}

\usepackage{amsmath,amssymb}
\usepackage{iftex}
\ifPDFTeX
  \usepackage[T1]{fontenc}
  \usepackage[utf8]{inputenc}
  \usepackage{textcomp} % provide euro and other symbols
\else % if luatex or xetex
  \usepackage{unicode-math}
  \defaultfontfeatures{Scale=MatchLowercase}
  \defaultfontfeatures[\rmfamily]{Ligatures=TeX,Scale=1}
\fi
\usepackage{lmodern}
\ifPDFTeX\else  
    % xetex/luatex font selection
\fi
% Use upquote if available, for straight quotes in verbatim environments
\IfFileExists{upquote.sty}{\usepackage{upquote}}{}
\IfFileExists{microtype.sty}{% use microtype if available
  \usepackage[]{microtype}
  \UseMicrotypeSet[protrusion]{basicmath} % disable protrusion for tt fonts
}{}
\makeatletter
\@ifundefined{KOMAClassName}{% if non-KOMA class
  \IfFileExists{parskip.sty}{%
    \usepackage{parskip}
  }{% else
    \setlength{\parindent}{0pt}
    \setlength{\parskip}{6pt plus 2pt minus 1pt}}
}{% if KOMA class
  \KOMAoptions{parskip=half}}
\makeatother
\usepackage{xcolor}
\setlength{\emergencystretch}{3em} % prevent overfull lines
\setcounter{secnumdepth}{5}
% Make \paragraph and \subparagraph free-standing
\ifx\paragraph\undefined\else
  \let\oldparagraph\paragraph
  \renewcommand{\paragraph}[1]{\oldparagraph{#1}\mbox{}}
\fi
\ifx\subparagraph\undefined\else
  \let\oldsubparagraph\subparagraph
  \renewcommand{\subparagraph}[1]{\oldsubparagraph{#1}\mbox{}}
\fi


\providecommand{\tightlist}{%
  \setlength{\itemsep}{0pt}\setlength{\parskip}{0pt}}\usepackage{longtable,booktabs,array}
\usepackage{calc} % for calculating minipage widths
% Correct order of tables after \paragraph or \subparagraph
\usepackage{etoolbox}
\makeatletter
\patchcmd\longtable{\par}{\if@noskipsec\mbox{}\fi\par}{}{}
\makeatother
% Allow footnotes in longtable head/foot
\IfFileExists{footnotehyper.sty}{\usepackage{footnotehyper}}{\usepackage{footnote}}
\makesavenoteenv{longtable}
\usepackage{graphicx}
\makeatletter
\def\maxwidth{\ifdim\Gin@nat@width>\linewidth\linewidth\else\Gin@nat@width\fi}
\def\maxheight{\ifdim\Gin@nat@height>\textheight\textheight\else\Gin@nat@height\fi}
\makeatother
% Scale images if necessary, so that they will not overflow the page
% margins by default, and it is still possible to overwrite the defaults
% using explicit options in \includegraphics[width, height, ...]{}
\setkeys{Gin}{width=\maxwidth,height=\maxheight,keepaspectratio}
% Set default figure placement to htbp
\makeatletter
\def\fps@figure{htbp}
\makeatother

\usepackage{booktabs}
\usepackage{longtable}
\usepackage{array}
\usepackage{multirow}
\usepackage{wrapfig}
\usepackage{float}
\usepackage{colortbl}
\usepackage{pdflscape}
\usepackage{tabu}
\usepackage{threeparttable}
\usepackage{threeparttablex}
\usepackage[normalem]{ulem}
\usepackage{makecell}
\usepackage{xcolor}
\newcommand{\mycommand}[1]{\textbf{#1}}
\makeatletter
\makeatother
\makeatletter
\makeatother
\makeatletter
\@ifpackageloaded{caption}{}{\usepackage{caption}}
\AtBeginDocument{%
\ifdefined\contentsname
  \renewcommand*\contentsname{Table of contents}
\else
  \newcommand\contentsname{Table of contents}
\fi
\ifdefined\listfigurename
  \renewcommand*\listfigurename{List of Figures}
\else
  \newcommand\listfigurename{List of Figures}
\fi
\ifdefined\listtablename
  \renewcommand*\listtablename{List of Tables}
\else
  \newcommand\listtablename{List of Tables}
\fi
\ifdefined\figurename
  \renewcommand*\figurename{Figure}
\else
  \newcommand\figurename{Figure}
\fi
\ifdefined\tablename
  \renewcommand*\tablename{Table}
\else
  \newcommand\tablename{Table}
\fi
}
\@ifpackageloaded{float}{}{\usepackage{float}}
\floatstyle{ruled}
\@ifundefined{c@chapter}{\newfloat{codelisting}{h}{lop}}{\newfloat{codelisting}{h}{lop}[chapter]}
\floatname{codelisting}{Listing}
\newcommand*\listoflistings{\listof{codelisting}{List of Listings}}
\makeatother
\makeatletter
\@ifpackageloaded{caption}{}{\usepackage{caption}}
\@ifpackageloaded{subcaption}{}{\usepackage{subcaption}}
\makeatother
\makeatletter
\@ifpackageloaded{tcolorbox}{}{\usepackage[skins,breakable]{tcolorbox}}
\makeatother
\makeatletter
\@ifundefined{shadecolor}{\definecolor{shadecolor}{rgb}{.97, .97, .97}}
\makeatother
\makeatletter
\makeatother
\makeatletter
\makeatother
\journal{Journal of Animal Biotelemetry}
\ifLuaTeX
  \usepackage{selnolig}  % disable illegal ligatures
\fi
\usepackage[]{natbib}
\bibliographystyle{elsarticle-harv}
\IfFileExists{bookmark.sty}{\usepackage{bookmark}}{\usepackage{hyperref}}
\IfFileExists{xurl.sty}{\usepackage{xurl}}{} % add URL line breaks if available
\urlstyle{same} % disable monospaced font for URLs
\hypersetup{
  pdftitle={Vertical movement behaviour of the starry smooth-hound shark Mustelus asterias in the North Sea},
  pdfauthor={Lotte Pohl; Niels Brevé; Carlota Muñiz; Jan Reubens},
  pdfkeywords={acoustic telemetry, geolocation modelling, Mustelus
asterias},
  colorlinks=true,
  linkcolor={blue},
  filecolor={Maroon},
  citecolor={Blue},
  urlcolor={Blue},
  pdfcreator={LaTeX via pandoc}}

\setlength{\parindent}{6pt}
\begin{document}

\begin{frontmatter}
\title{Vertical movement behaviour of the starry smooth-hound shark
\emph{Mustelus asterias} in the North Sea \\\large{Master Thesis} }
\author[1,2]{Lotte Pohl%
\corref{cor1}%
\fnref{fn1}}
 \ead{lotte.pohl@imbrsea.eu} 
\author[3,4]{Niels Brevé%
%
\fnref{fn2}}
 \ead{breve@sportvisserijnederland.nl} 
\author[1]{Carlota Muñiz%
%
\fnref{fn3}}
 \ead{carlota.muniz@vliz.be} 
\author[1]{Jan Reubens%
%
\fnref{fn4}}
 \ead{jan.reubens@vliz.be} 

\affiliation[1]{organization={Flemish Marine Institute, Marine
Observation Centre},addressline={Slipwaykaai
2},city={Ostend},postcode={8400},postcodesep={}}
\affiliation[2]{organization={Ghent University, Marine Biology research
group},addressline={Krijgslaan
281/S8},city={Ghent},postcode={9000},postcodesep={}}
\affiliation[3]{organization={Wageningen University and Research, Marine
Ecology Group},addressline={Droevendaalsesteeg
1},city={Wageningen},postcode={6700},postcodesep={}}
\affiliation[4]{organization={Sportvisserij
Nederlands},addressline={Leyenseweg
115},city={Bilthoven},postcode={3721},postcodesep={}}

\cortext[cor1]{Corresponding author}
\fntext[fn1]{Msc Student}
\fntext[fn2]{Second Promotor}
\fntext[fn3]{Supervisor}
\fntext[fn4]{Promotor}
        
\begin{abstract}
tempor incididunt ut labore et dolore magna aliqua.ffffffff
ffffffffffffffffffffff fffffffffffffffffff ffffffffffffffffffffffffff
fffffffffffffffffffffff fffffffffffffffffffff ffffffffff
ffffffffffffffff ffffffffffffffff fffffffffffffffffff
fffffffffffffffffffffffff fffffffffffffff ffffffffffffffffff
fffffffffffffff fffffffffffffffffffffff.tempor incididunt ut labore et
dolore magna aliqua.ffffff ffffffffffffffffffffffff fffffff
ffffffffffffffff ffffffffffffffffff fffff ffffffffffffffffff
fffffffffffffffffffff ffffffffffffffffff fffffffffffffffff
fffffffffffffffffffff ffffffffffffffff ffffffffffffff
ffffffffffffffffffffffff fffffffffffffffffffffffffffffffff
ffffffffffffffffftempor incididunt ut labore et dolore magna
aliqua.fffffffffffff fffffffffffff ffffffffff fffffffff
ffffffffffffffffffff fffffffff fffffffffffffffffffffff
fffffffffffffffffff fffffffff ffffffffffffffff ffffffff fffffff f
fffffffffffffff ffffff fffffffffffffffffff fffffffffffffff
ffffffffffffffffffff fffffffffffff fffffffffffff
fffffffffffffffffftempor incididunt ut labore et dolore magna
aliqua.fffffffffffff ffffffffffffffffffff ffffffffffffffffffffff
ffffffffffffffffff fffffffffffff ffffffffffffffffff fffffffffffffffff
ffffffffffffffffffffffffffffff fffffffffffffff ffffffffff
fffffffffffffff fffffffffffffffffff fffffffffffff
fffffffffffffffffffffffff fffffffffffffffffffffffffffftempor incididunt
ut labore et dolore magna aliqua.fffffffffffffff
ffffffffffffffffffffffff ffffffffffffffff fffffffffffffffffffffff
fffffffffffffffff ffffffffffffffffff fffffffffffffffffff ffffffffffffff
fffff ffffffffffffffffffff fffff fffffffffffffffffffffffffffffff
fffffffffffffffffff fffffffffffffffffffffffffffffffff
fffffffffffffffftempor incididunt ut labore et dolore magna aliqua
tempor incididunt ut labore et dolore magna aliqua.ffffffff
ffffffffffffffffffffff fffffffffffffffffff ffffffffffffffffffffffffff
fffffffffffffffffffffff fffffffffffffffffffff ffffffffff
ffffffffffffffff ffffffffffffffff fffffffffffffffffff
fffffffffffffffffffffffff fffffffffffffff ffffffffffffffffff
fffffffffffffff fffffffffffffffffffffff.tempor incididunt ut labore et
dolore magna aliqua.ffffff ffffffffffffffffffffffff fffffff
ffffffffffffffff ffffffffffffffffff fffff ffffffffffffffffff
fffffffffffffffffffff ffffffffffffffffff fffffffffffffffff
fffffffffffffffffffff ffffffffffffffff ffffffffffffff
ffffffffffffffffffffffff fffffffffffffffffffffffffffffffff
ffffffffffffffffftempor incididunt ut labore et dolore magna
aliqua.fffffffffffff fffffffffffff ffffffffff fffffffff
ffffffffffffffffffff fffffffff fffffffffffffffffffffff
fffffffffffffffffff fffffffff ffffffffffffffff ffffffff fffffff f
fffffffffffffff ffffff fffffffffffffffffff fffffffffffffff
ffffffffffffffffffff fffffffffffff fffffffffffff
fffffffffffffffffftempor incididunt ut labore et dolore magna
aliqua.fffffffffffff ffffffffffffffffffff ffffffffffffffffffffff
ffffffffffffffffff fffffffffffff ffffffffffffffffff fffffffffffffffff
ffffffffffffffffffffffffffffff fffffffffffffff ffffffffff
fffffffffffffff fffffffffffffffffff fffffffffffff
fffffffffffffffffffffffff fffffffffffffffffffffffffffftempor incididunt
ut labore et dolore magna aliqua.fffffffffffffff
ffffffffffffffffffffffff ffffffffffffffff fffffffffffffffffffffff
fffffffffffffffff ffffffffffffffffff fffffffffffffffffff ffffffffffffff
fffff ffffffffffffffffffff fffff fffffffffffffffffffffffffffffff
fffffffffffffffffff fffffffffffffffffffffffffffffffff
fffffffffffffffftempor incididunt ut labore et dolore magna aliqua
tempor incididunt ut labore et dolore magna aliqua.ffffffff
ffffffffffffffffffffff fffffffffffffffffff ffffffffffffffffffffffffff
fffffffffffffffffffffff fffffffffffffffffffff ffffffffff
ffffffffffffffff ffffffffffffffff fffffffffffffffffff
fffffffffffffffffffffffff fffffffffffffff ffffffffffffffffff
fffffffffffffff fffffffffffffffffffffff.tempor incididunt ut labore et
dolore magna aliqua.ffffff ffffffffffffffffffffffff fffffff
ffffffffffffffff ffffffffffffffffff fffff ffffffffffffffffff
fffffffffffffffffffff ffffffffffffffffff fffffffffffffffff
fffffffffffffffffffff ffffffffffffffff ffffffffffffff
ffffffffffffffffffffffff fffffffffffffffffffffffffffffffff
ffffffffffffffffftempor incididunt ut labore et dolore magna
aliqua.fffffffffffff fffffffffffff ffffffffff fffffffff
ffffffffffffffffffff fffffffff fffffffffffffffffffffff
fffffffffffffffffff fffffffff ffffffffffffffff ffffffff fffffff f
fffffffffffffff ffffff fffffffffffffffffff fffffffffffffff
ffffffffffffffffffff fffffffffffff fffffffffffff
fffffffffffffffffftempor incididunt ut labore et dolore magna
aliqua.fffffffffffff ffffffffffffffffffff ffffffffffffffffffffff
ffffffffffffffffff fffffffffffff ffffffffffffffffff fffffffffffffffff
ffffffffffffffffffffffffffffff fffffffffffffff ffffffffff
fffffffffffffff fffffffffffffffffff fffffffffffff
fffffffffffffffffffffffff fffffffffffffffffffffffffffftempor incididunt
ut labore et dolore magna aliqua.fffffffffffffff
ffffffffffffffffffffffff ffffffffffffffff fffffffffffffffffffffff
fffffffffffffffff ffffffffffffffffff fffffffffffffffffff ffffffffffffff
fffff ffffffffffffffffffff fffff fffffffffffffffffffffffffffffff
fffffffffffffffffff fffffffffffffffffffffffffffffffff
fffffffffffffffftempor incididunt ut labore et dolore magna aliqua
tempor incididunt ut labore et dolore magna aliqua.ffffffff
ffffffffffffffffffffff fffffffffffffffffff ffffffffffffffffffffffffff
fffffffffffffffffffffff fffffffffffffffffffff ffffffffff
ffffffffffffffff ffffffffffffffff fffffffffffffffffff
fffffffffffffffffffffffff fffffffffffffff ffffffffffffffffff
fffffffffffffff fffffffffffffffffffffff.tempor incididunt ut labore et
dolore magna aliqua.ffffff ffffffffffffffffffffffff fffffff
ffffffffffffffff ffffffffffffffffff fffff ffffffffffffffffff
fffffffffffffffffffff ffffffffffffffffff fffffffffffffffff
fffffffffffffffffffff ffffffffffffffff ffffffffffffff
ffffffffffffffffffffffff fffffffffffffffffffffffffffffffff
fffffffffffffffff fffffffffffffffffffffff.tempor incididunt ut labore et
dolore magna aliqua.ffffff ffffffffffffffffffffffff fffffff
ffffffffffffffff ffffffffffffffffff fffff ffffffffffffffffff
fffffffffffffffffffff ffffffffffffffffff fffffffffffffffff
fffffffffffffffffffff ffffffffffffffff ffffffffffffff
ffffffffffffffffffffffff fffffffffffffffffffffffffffffffff
fffffffffffffffff fffffffffffffffffffffff.tempor incididunt ut labore et
dolore magna aliqua.ffffff ffffffffffffffffffffffff fffffff
ffffffffffffffff ffffffffffffffffff fffff ffffffffffffffffff
fffffffffffffffffffff ffffffffffffffffff fffffffffffffffff
fffffffffffffffffffff ffffffffffffffff ffffffffffffff
ffffffffffffffffffffffff fffffffffffffffffffffffffffffffff
fffffffffffffffff
\end{abstract}





\begin{keyword}
    acoustic telemetry \sep geolocation modelling \sep 
    Mustelus asterias
\end{keyword}
\end{frontmatter}
    \ifdefined\Shaded\renewenvironment{Shaded}{\begin{tcolorbox}[interior hidden, breakable, sharp corners, borderline west={3pt}{0pt}{shadecolor}, enhanced, boxrule=0pt, frame hidden]}{\end{tcolorbox}}\fi

\renewcommand*\contentsname{Table of contents}
{
\hypersetup{linkcolor=}
\setcounter{tocdepth}{3}
\tableofcontents
}
\newpage{}

\hypertarget{list-of-abbreviations}{%
\section*{List of Abbreviations}\label{list-of-abbreviations}}
\addcontentsline{toc}{section}{List of Abbreviations}

\begin{table}[!h]
\centering
\begin{tabular}{l|l}
\hline
Abbreviation & Explanation\\
\hline
ADST & Acoustic Data Storage Tag\\
\hline
BPNS & Belgian Part of the North Sea\\
\hline
CWT & Continuous Wavelet Transform\\
\hline
DST & Data Storage Tag\\
\hline
ETN & European Tracking Network\\
\hline
FFT & Fast Fourier Transform\\
\hline
FT & Fourier Transformation\\
\hline
HMM & Hidden Markov Model\\
\hline
PBARN & Permanent Belgian Receiver Network\\
\hline
RQ & Research Question\\
\hline
TL & Total Length\\
\hline
UTC & Coordinated Universal Time\\
\hline
f & female\\
\hline
m & male\\
\hline
\end{tabular}
\end{table}

\newpage{}

\hypertarget{executive-summary}{%
\section{Executive Summary}\label{executive-summary}}

The starry smooth-hound Mustelus asterias (Cloquet, 1819) is a widely
distributed demersal shark in the Northeast Atlantic, yet under
increasing fishing pressure. To lay the grounds for future species
management plans, information on its complex annual migration and
residency behaviour is needed. The movement of fish is commonly studied
with electronic tags. A novel technology is the Acoustic Data Storage
Tag (ADST), which can be detected by acoustic receivers and additionally
logs temperature and water depth in predefined intervals. To access the
data logs the tag needs to be recovered. This study employed ADST to
characterise the migration and residency of M. asterias in the North Sea
regarding seasonality and sex. Exploratory and spectral analyses were
performed using an ADST dataset spanning from July 2018 to July 2020.

In 2018 and 2019, 30 adult M. asterias (19 females, 11 males) were
equipped with ADST-V13TP (Innovasea, 518 days estimated battery life) in
the Scheldt Estuary. The Permanent Belgian Receiver Network (PBARN)
spans the Belgian Part of the North Sea (BPNS) and the Scheldt Estuary
with 160 active receivers during the study period. 14 females and 4
males were detected 10940 times by 27 acoustic receivers. All detections
were between April and November, suggesting seasonal presence of M.
asterias in the receiver area. 96.7\% of detections were at receivers
just outside the Western Scheldt, and 97.4\% of detections were from
females. This indicates that the outer part of the Scheldt Estuary is a
relevant habitat for females between April and November.

8 ADST were recovered, one female and one male logging data for over a
year. Both tags logged depths between 0 and 20 m between June and
October, and depths between 10 and 75 m between October and June. The
changing depth range reflects the migration of the sharks into deeper
waters during winter months. Continuous Wavelet Transform (CWT) resolves
periodic patterns in the time domain and displays differences in
periodic patterns between seasons and the two individuals. The CWT
results suggest that the female is resting during summer and feeding in
a biweekly pattern in winter, and that the male is continuously feeding.

This study provides first details about seasonality in vertical movement
behaviour of M. asterias and confirms its seasonal presence in the
Western Scheldt. Data on the vertical movement of more individuals is
needed to determine if the patterns observed here are related to sex.

\hypertarget{abstract}{%
\section{Abstract}\label{abstract}}

The starry smooth-hound \emph{Mustelus asterias} is widely distributed
in the Northeast Atlantic, yet under increasing fishing pressure. To lay
the grounds for future species management plans, information on its
complex annual migration and residency behaviour is needed. Fish
movement is commonly studied with electronic tags, one novel technology
being the Acoustic Data Storage Tag (ADST). It can be detected by
acoustic receivers and additionally logs temperature and water depth in
predefined intervals. This study characterises the migration and
residency of \emph{M. asterias} in the North Sea regarding season and
sex. 30 \emph{M. asterias} were equipped with ADST in the Scheldt
Estuary in 2018 and 2019. Acoustic detections of 18 individuals in the
Western Scheldt estuary between July 2018 and July 2020 suggest seasonal
presence of females just outside of the Western Scheldt between April
and October. Spectral analysis of the depth logs of two individuals
logging for over a year indicates feeding behaviour of the male
throughout most of the year. The female shark primarily rests on the
bottom during summer and feeds in a biweekly rhythm during winter. This
study presents the first detailed insights into vertical movement
behaviour differences between seasons and individuals of \emph{M.
asterias}.

\hypertarget{sec-intro}{%
\section{Introduction}\label{sec-intro}}

Gathering knowledge on migratory fish species with complex life cycles
is challenging, yet necessary to improve management of such species in
the future \citep{brownscombe_2022}. Fish with late reproduction and low
offspring numbers such as elasmobranchs are especially vulnerable to
anthropogenic effects and require special attention
\citep{stevens_2000}.

An elasmobranch with a complex life cycle involving biannual
reproduction and a sex-specific annual migration is the starry
smooth-hound \emph{Mustelus asterias} (Cloquet, 1819), a demersal
triakid shark widely distributed in the Northeast Atlantic and
Mediterranean Sea \citep{breve_2016, griffiths_2020}. In the past
decade, both commercial landings \citep{ices_2022, bitonporsmoguer_2022}
and abundance in the North Sea \citep{mccullyphillips_2015} have
increased. The International Council for the Exploration of the Sea
(ICES) classified the species to be \emph{Near Threatened}
\citep{ices_2022}. Meanwhile, knowledge on the seasonal presence of
\emph{M. asterias} in the North Sea, especially in Belgian Part of the
North Sea (BPNS) and the estuary of the Western Scheldt is missing.
Tracking the movement and presence of migratory aquatic species in space
and time, however, requires advanced technologies which have improved
substantially in recent years \citep{whoriskey_2019}.

\hypertarget{aquatic-telemetry}{%
\subsection{Aquatic telemetry}\label{aquatic-telemetry}}

Telemetry refers to the remote monitoring of animals using electronic
transmitters or tags, and receivers \citep{whoriskey_2019}, allowing for
the tracking of marine animal movement. A multitude of telemetry methods
exists today, popular technologies to study aquatic animals include
acoustic telemetry and archival telemetry \citep{thorstad_2013}.
Telemetry furthermore enables the quantification of animal migrations
and (vertical and horizontal) habitat use \citep{hussey_2015}, which
informs species management and conservation spanning marine, freshwater
and terrestrial ecosystems \citep{beger_2010}.

In acoustic telemetry, animals are internally or externally equipped
with electronic tags that emit acoustic signals unique to each tag.
Acoustic receivers are able to detect these acoustic signals within a
certain range of the device and identify the individual tag. Such
networks exist primarily in shallow, coastal areas \citep{hussey_2015}
such as the Acoustic Tracking Array Platform (ATAP) in South Africa
\citep{cowley_2017} and the Permanent Belgian Acoustic Receiver Network
\citep[PBARN,][]{reubens_2019}.

Archival telemetry uses tags or biologgers that record and locally store
environmental variables such as water pressure and temperature. Data
Storage Tags (DST) require the retrieval of the tag for downloading data
\citep{thorstad_2013}.

Acoustic data storage tags (ADST) are a combination tag, merging
acoustic and data storage tags \citep{goossens_2023}. As such, the tag
emits unique acoustic signals that can be detected by an acoustic
receiver within a certain range, and additionally records water pressure
and temperature in predefined intervals. Per detection, one value of
either temperature or water depth gets transmitted. ADST gather data on
horizontal fish movement inside acoustic receiver networks while
simultaneously collecting fine-scale vertical movement data that can be
used to infer movement of the fish beyond acoustic receiver arrays.

Different telemetry methods result in different datasets. In acoustic
telemetry, recorded data contains the time that a specific tag was
detected by a specific receiver (with a known location). These datasets
come in a horizontal dimension, i.e., with the latitude and longitude
position of the acoustic receiver. Archival telemetry (using DST, for
example) generates regular time series in pre-defined intervals, logging
water depth and other environmental variables. These data come in a
vertical dimension, spanning the range of the animal's depth use.
Datasets resulting from ADST contain time stamps of the tag's detection
at a specific acoustic receiver, and, in case the tag was recovered,
logs of water depth and other environmental variables.

To determine the approximate location (i.e., latitude and longitude), of
a DST log (which represents the vertical dimension), stochastic
geolocation models are commonly used \citep{gatti_2021}. These models
involve bottom bathymetry and other environmental factors, including
water temperature, to estimate the trajectory followed by the fish
\citep{nielsen_2004}. Common geolocation modelling approaches use Hidden
Markov Models \citep[HMM,][]{pedersen_2008, woillez_2016}. For certain
species with high economic importance (Atlantic cod \emph{Gadus morhua}
and European seabass \emph{Dicentrarchus labrax}, for example), those
models have been improved by including different behavioural states of
the animals \citep{pedersen_2008, heerah_2017}. Behavioural states (such
as low or high activity) are driven by the animal's environment and
internal ecophysiological processes \citep{gurarie_2016}. Geolocation
modelling has been employed for \emph{M. asterias} with the
implementation of behavioural states \citep[following the approach
by][]{pedersen_2008} and without \citep{goossens_2023}. Geolocation
model outputs for one female and one male \emph{M. asterias} are shown
in Figure~\ref{fig-mapgeolocation}.

\begin{figure}

\begin{minipage}[t]{\linewidth}

{\centering 

\raisebox{-\height}{

\includegraphics{thesis_manuscript_files/figure-pdf/fig-mapgeolocation-1.pdf}

}

}

\subcaption{\label{fig-mapgeolocation-1}Female \emph{M. asterias} (tag
308).}
\end{minipage}%
\newline
\begin{minipage}[t]{\linewidth}

{\centering 

\raisebox{-\height}{

\includegraphics{thesis_manuscript_files/figure-pdf/fig-mapgeolocation-2.pdf}

}

}

\subcaption{\label{fig-mapgeolocation-2}Male \emph{M. asterias} (tag
321).}
\end{minipage}%

\caption{\label{fig-mapgeolocation}Results of geolocation modelling
(without behavioural states) on the raw DST depthlogs.}

\end{figure}

\hypertarget{the-starry-smooth-hound-shark-mustelus-asterias}{%
\subsection{\texorpdfstring{The starry smooth-hound shark \emph{Mustelus
asterias}}{The starry smooth-hound shark Mustelus asterias}}\label{the-starry-smooth-hound-shark-mustelus-asterias}}

The starry smooth-hound shark (Mustelus asterias, Family: Triakidae)
inhabits waters of the Northeast Atlantic and the Mediterranean Sea,
spanning from the Irish Sea and Northern North Sea in the North to an
unknown limit in the south \citep{ices_2022}. This species typically
measures around 30 cm in total length (TL) at birth and can grow up to
140 cm TL \citep{mccullyphillips_2015}. Sexual maturity is attained at
100 cm TL for males and 120 cm TL for females \citep{farrell_2010},
which usually occurs at 4-5 years for males and 6 years for females. The
average lifespan of starry smooth-hound sharks is around 13 years for
males and 18 years for females \citep{farrell_2010}.

The species is matrotrophic aplacentally viviparous \citep[i.e., embryos
absorb nutrients from a yolk sack that is used up during
gestation,][]{farrell_2010a, mccullyphillips_2015}. \citet{farrell_2014}
found \emph{M. asterias} to display polyandry (i.e., one female mating
with several males within one breeding season) and multiple paternity
(i.e., pups within one litter originate from several males). Moreover,
it has a biannual reproduction cycle with a gestation period of about 12
months, followed by a 12-month resting period. \emph{M. asterias}
females can store sperm for up to 12 months \citep{farrell_2010a}. The
same study found embryos from different developmental stages in
gestating females (termed embryo asynchronism). \citet{griffiths_2020}
reported philopatric behaviour (i.e., females returning to the same area
for parturition) for the first time. Identified parturition grounds for
\emph{M. asterias} include the estuary of the Eastern Scheldt and, more
generally, the Bristol Channel, Southern North Sea and the English
Channel \citep{dureuil_2013, mccullyphillips_2015, breve_2016}. Pupping
possibly takes place between April and September
\citep{farrell_2010a, mccullyphillips_2015}.

\emph{M. asterias} is morphologically adapted to a demersal life style,
exhibiting a ventrally located snout that enables it to feed on prey on
the bottom. It is a specialist feeder on benthic and suprabenthic
crustaceans \citep{mccullyphillips_2020, bitonporsmoguer_2022}. Common
prey include the hermit crab (\emph{Pagurus bernhardus}), the flying
crab (\emph{Liocarcinus holsatus}), the common shore crab
(\emph{Carcinus maenas}) and the edible crab (\emph{Cancer pagurus})
\citep{mccullyphillips_2015, mccullyphillips_2020}. Potential predators
of \emph{M. asterias} are the grey seal (\emph{Halichoerus grypus})
which occurs in the North Sea, the blue shark (\emph{Prionace glauca})
and the common dolphin (\emph{Delphinus delphis}), both of which occur
in the Western English Channel \citep{griffiths_2020}.

\emph{M. asterias} shows higher vertical activity during night than
during the day \citep{griffiths_2020}. This indicates foraging and
preying behaviour at night, a commonly observed behaviour in
elasmobranchs \citep{sims_2006}. During the day, \emph{M. asterias}
potentially avoids predation by resting on the bottom
\citep{griffiths_2020}. The species displays annual, potentially
sex-specific migration. Females are thought to move into the English
Channel and the Bay of Biscay in winter, and males move into the
Northern North Sea and into the English Channel
\citep{breve_2016, breve_2020, griffiths_2020}. \citet{breve_2016} and
\citet{griffiths_2020} hypothesise that the \emph{M. asterias}
population in the Northeast Atlantic consists of two (or more)
sub-populations that mix on seasonal grounds.

\hypertarget{sec-intro-aims}{%
\subsection{Aims of this work}\label{sec-intro-aims}}

The seasonal presence of \emph{M. asterias} in the Western Scheldt
estuary, a presumed summer habitat of the species, has not been studied
yet. The confirmation of seasonal presence of \emph{M. asterias} would
be relevant to species management. This could involve stock assessments
or seasonal restrictions for commercial fishing in the future
\citep{benot_2003}. Moreover, detailed information on potential
differences in vertical movement regarding seasonality and sex,
including the timing and duration of migrations, are currently missing.
Such information can improve geolocation models through the
implementation of behavioural states \citep{pedersen_2008}. With the aim
of addressing these knowledge gaps, this study pursues two lines of
research.

\textbf{Research line 1}: Presence of \emph{M. asterias} in the Western
Scheldt Estuary and the Belgian Part of the\\
\hspace*{0.333em}\hspace*{0.333em}North Sea (BPNS)

\begin{itemize}
\tightlist
\item
  Research Question 1: What seasonal patterns of presence does \emph{M.
  asterias} display in the Western Scheldt and the BPNS?
\item
  Research Question 2: Are there differences in seasonal presence
  between males and females?
\item
  Research Question 3: How do the spatial preferences of \emph{M.
  asterias} vary within the Western Scheldt and the BPNS?
\end{itemize}

\textbf{Research line 2}: Depth utilisation of \emph{M. asterias} on a
seasonal scale

\begin{itemize}
\tightlist
\item
  Research Question 4: What seasonal patterns of vertical behaviour does
  \emph{M. asterias} show?
\item
  Research Question 5: Are there differences in depth utilisation
  between males and females?
\end{itemize}

Due to its ability to collect information both within the area of
acoustic receiver networks and beyond those by locally storing data,
Acoustic Data Storage Tags (ADST) are a suitable technology to assess
these questions.

\hypertarget{sec-mm}{%
\section{Materials and Methods}\label{sec-mm}}

\hypertarget{sec-mm_studyarea}{%
\subsection{Study Area}\label{sec-mm_studyarea}}

The study area encompasses two scales, reflecting the two lines of
research of this study. The first research line focuses on the Belgian
Part of the North Sea (BPNS) and the estuary of the Western Scheldt. The
second research line includes year-round behaviour and thus, the study
area is broader and less defined. Since the species is thought to move
both into the English Channel and into the North Sea north of the BPNS,
both regions constitute the study area for the second research line. An
overview on the different study areas is displayed in
\textbf{?@fig-mapstudyarea}.

\hypertarget{sec-mmscheldtBPNS}{%
\subsubsection{The BPNS and Scheldt Estuary}\label{sec-mmscheldtBPNS}}

The Belgian Part of the North Sea (BPNS) is generally characterised by
shallow water depths below 40 m \citep{thierry_2019}, sandy or muddy
sediment \citep{wolff_1973, vasquez_2021} and multiple sandbanks and
troughs. The river Scheldt meets the North Sea north of the border
between the Netherlands and Belgium, forming an Estuary with a complex
regime of sandflats and channels \citep{claessens_1988}. Both the BPNS
and the Scheldt Estuary are highly mixed due to strong currents and
tides \citep{otto_1990}, introducing saline waters well into the Scheldt
Estuary \citep{ouboter_1998}. The Scheldt Estuary is characterised by
medium to coarse sand and salinity between ranges 2 and 33.5 ppt
\citep{baeyens_1998}, peaking in summer \citep{maes_1998}. The bottom
depth averages around 11 m with occasional troughs of up to 20 or 30 m
\citep{thierry_2019}. The BPNS and the Scheldt Estuary are influenced by
two water bodies: The Channel water and the Continental coastal water,
meeting about 30 km off the Belgian coast \citep{wolff_1973}. The former
is characterised by high salinity (\textgreater{} 18 ppt), low nutrient
levels and water temperatures ranging between 6 °C in winter to 16°C in
summer. The latter has a lower salinity, has high nutrient levels and a
larger temperature range than the Channel water averaging between 3°C in
winter and 17°C in summer. An overview map of the BPNS and the Scheldt
Estuary is provided in \textbf{?@fig-mapstudyarea-2}. Fish and
crustacean species composition and abundance underlies strong seasonal
patterns, peaking in summer \citep{maes_1998, maes_2005}. Seasonally
occuring species are marine fish like Atlantic herring (\emph{Clupea
harengus}), Sprat (\emph{Sprattus sprattus}) or Whiting
(\emph{Merlangius merlangus}), and abundant crustacean species include
the common shrimp (\emph{Crangon crangon}) and the palaemonid shrimp
\citep[\emph{Palaemon varians},][]{maes_1998}. The lower part of the
Scheldt Estuary from Flushing (Dutch: Vlissingen) 55-60 km upstream is
referred to as Western Scheldt
Estuary\citep{baeyens_1998, ouboter_1998}. Between Bergen op Zoom and
the island Neeltje Jans (both Netherlands), the Estuary is referred to
as the Eastern Scheldt Estuary.

\hypertarget{the-east-anglian-coast}{%
\subsubsection{The East Anglian coast}\label{the-east-anglian-coast}}

The east Anglian coast lies within the Southern North Sea and is
characterised by shallow coastal waters between 20 and 25 m water depth,
fast tidal streams and sandy and muddy bottom sediments
\citep{harrison_1990}. Tidal currents are strongest at 1.3
\(\frac{m}{s}\) during new and full moon \citep{arnold_1994}. Abundant
crustacea include \emph{Hyas coarctatus} \citep{dyer_1985},
\emph{Pandalus montagui}, \emph{Carcinus spp.} and \emph{Eupagurus spp.}
\citep{sergeant_1951}.

\hypertarget{sec-mmstudyareaec}{%
\subsubsection{The English channel}\label{sec-mmstudyareaec}}

The English Channel (EC) extends from the Dover Strait to England's
southwestern tip, spanning 77 000 km\textsuperscript{2}
\citep{dauvin_2012}. The mean bottom depth increases from around 25 m in
the east to 75 m in the west \citep{dauvin_2012}, except for the 174 m
deep \emph{Hurd deep}, a trench located close to the channel island
Guernsey. An overview of the English Channel and the North Sea is
provided in \textbf{?@fig-mapoverview}. Shipping, Strong tidal currents
from west to east are present, which are higher on the French than on
the English coast due to Coriolis force, reaching maximum values of 4
\(\frac{m}{s}\) at the Cap de la Hague \citep{salomon_1993}. Bottom
temperatures vary less in the western EC (range: 9°C to 14°C) than in
the eastern EC (8°C and 16°C). Species living on the bottom include
Crustacea like the edible crab (\emph{Cancer pagurus}), the common
spider crab (\emph{Maja squinado}), the European lobster (\emph{Homarus
gammarus}) and \emph{Upogebia spp.} \citep{holme_1966, vaz_2007}.

\hypertarget{the-permanent-belgian-receiver-network}{%
\subsection{The Permanent Belgian Receiver
Network}\label{the-permanent-belgian-receiver-network}}

The acoustic receivers of the Permanent Belgian Receiver Network (PBARN)
were used in this study \citep{reubens_2018}. The PBARN is part of the
European Tracking Network (ETN), which was established in 2017 as a
pan-european aquatic telemetry network. As of 2019, the PBARN consisted
of 160 permanently installed and monitored acoustic receivers within the
BPNS, the Scheldt Estuary, the Dijle, Rupel, Albert Canal and the Meuse
River. \textbf{?@fig-mapstudyarea-3} displays an overview of the
acoustic receivers in the estuary of the Western Scheldt. A detection
probability of \textgreater50 \% was estimated for a radius of 500-700 m
around the receiver \citep{goossens_2022}, and \textgreater70 \% for 200
m radius \citep{reubens_2018}.

\hypertarget{acoustic-tags-and-tagging-procedure}{%
\subsection{Acoustic tags and tagging
procedure}\label{acoustic-tags-and-tagging-procedure}}

\emph{Innovasea} ADST-V13TP tags (Innovasea Ltd., Boston, MA, USA) were
used for this study, measuring 46 mm in length and weighing 13 g. Each
tag had an estimated battery lifetime of 518 days, and contained a
pressure and a temperature sensor with logging intervals of 120 s and
240 s, respectively. For each acoustic detection, a pressure or
temperature value was transmitted with a ratio of 1:3
(pressure:temperature). Further details about the tags can be found in
the Supporting Information of \citet{goossens_2023}. The pressure
sensors recorded with a resolution of 0.3 m and an accuracy of \(\pm\)
3.4 m until the depth of 68 m, according to the manufacturer
\citep{Innovasea_ADST}. The maximum depth value recorded in this study
is 75.2 m. Since this is deeper than the maximum depth measureable
according to the manufacturers, measurements around the limit of 68 m
have to be interpreted with caution.

Tagging of 30 \emph{M. asterias} individuals occurred in July and
August, 2018 (July 19, n = 13, August 02 and 03, n = 8 and n = 1) and in
July, 2019 (July 11 and 12, n = 4 for both dates). Details on the tagged
individuals are listed in Table~\ref{tbl-animalinfo}.

Tagged animals were both caught and released in two locations, just west
of \emph{Neeltje Jans} (individuals tagged in July, both 2018 and 2019,
Latitude: 51.6135652, Longitude: 3.6561763), and outside the mouth of
the Western Scheldt (individuals tagged in August 2018, Latitude:
51.4285383, Longitude: 3.4074816). The tagging locations are marked in
\textbf{?@fig-mapstudyarea-3}.

Animals were caught using line fishing or handline. After capture, each
animal was placed in a holding tank filled with sea water. Before the
surgery the animals were turned with the ventral side facing upwards to
induce tonic immobility \citep{kessel_2015} and the midventral line was
incised 2-3 cm with a scalpel. Next, the Acoustic Data Storage Tag was
inserted into the abdominal cavity with forceps, and the incision was
closed with 2-3 stitches of mono-filament \citep{goossens_2023}. Photos
of the capture and tagging of \emph{M. asterias} are shown in
Figure~\ref{fig-taggingpics}.

Females (n = 19, mean TL = 0.86 cm) were significantly larger than males
(n = 11, mean TL = 0.78 cm, one-sided Welch Two Sample t-test,
\(\alpha\) = 0.95, \emph{p} = 0.0021448).

\hypertarget{tbl-animalinfo}{}
\begin{table}[H]
\caption{\label{tbl-animalinfo}Information on the tagged individuals. Release location 1 refers to
Neeltje Jans, release location 2 refers to just outside of the Western
Scheldt. det. = detected, loc. = location, RI = residency index (days
detected / 518 days). The 518 days correspond to the estimated battery
life by the tag manufacturer. }\tabularnewline

\centering
\begin{tabular}{llrrlrrrlrrr}
\toprule
\makecell[c]{tag serial\\no.} & sex & \makecell[c]{TL\\ in m} & \makecell[c]{bodymass\\ in kg} & \makecell[c]{release\\date} & \makecell[c]{times\\det.} & \makecell[c]{d\\det.} & \makecell[c]{h\\det.} & \makecell[c]{date last\\det.} & \makecell[c]{days at\\liberty} & RI & \makecell[c]{release\\loc.}\\
\midrule
1293314 & f & 0.71 & NA & 2018-07-19 & 19 & 1 & 3 & 2018-07-31 & 12 & 0.002 & 1\\
1293321 & m & 0.81 & NA & 2018-07-19 & 0 & 0 & 0 & NA & 485 & NA & 1\\
1293315 & m & 0.82 & NA & 2018-07-19 & 159 & 6 & 20 & 2019-05-22 & 307 & 0.012 & 1\\
1293316 & m & 0.73 & NA & 2018-07-19 & 41 & 1 & 4 & 2019-07-09 & 355 & 0.002 & 1\\
1293322 & f & 0.80 & NA & 2018-07-19 & 0 & 0 & 0 & NA & 19 & NA & 1\\
\addlinespace
1293317 & f & 0.80 & NA & 2018-07-19 & 4 & 1 & 1 & 2018-07-21 & 2 & 0.002 & 1\\
1293318 & f & 0.79 & NA & 2018-07-19 & 14 & 3 & 4 & 2019-09-24 & 432 & 0.006 & 1\\
1293319 & f & 0.73 & NA & 2018-07-19 & 7 & 1 & 1 & 2018-08-01 & 19 & 0.002 & 1\\
1293320 & m & 0.81 & NA & 2018-07-19 & 0 & 0 & 0 & NA & 0 & NA & 1\\
1293293 & m & 0.78 & NA & 2018-07-19 & 0 & 0 & 0 & NA & 0 & NA & 1\\
\addlinespace
1293294 & f & 0.70 & NA & 2018-07-19 & 0 & 0 & 0 & NA & 0 & NA & 1\\
1293295 & m & 0.78 & NA & 2018-07-19 & 21 & 2 & 5 & 2018-08-19 & 31 & 0.004 & 1\\
1293296 & m & 0.77 & NA & 2018-07-19 & 1 & 1 & 0 & NA & 1 & NA & 1\\
1293297 & f & 0.88 & 3.30 & 2018-08-02 & 1134 & 49 & 138 & 2019-09-26 & 420 & 0.095 & 2\\
1293300 & f & 0.98 & 3.00 & 2018-08-02 & 304 & 28 & 69 & 2018-10-27 & 86 & 0.054 & 2\\
\addlinespace
1293298 & f & 0.95 & 3.80 & 2018-08-02 & 239 & 8 & 24 & 2019-09-18 & 412 & 0.015 & 2\\
1293307 & f & 0.83 & 4.20 & 2018-08-02 & 596 & 36 & 81 & 2019-09-23 & 417 & 0.069 & 2\\
1293308 & f & 0.99 & 3.70 & 2018-08-02 & 2067 & 55 & 257 & 2019-10-05 & 429 & 0.106 & 2\\
1293301 & f & 0.95 & 3.30 & 2018-08-02 & 1 & 1 & 0 & NA & 1 & NA & 2\\
1293299 & f & 0.91 & 3.50 & 2018-08-02 & 3520 & 84 & 410 & 2019-10-30 & 454 & 0.162 & 2\\
\addlinespace
1293309 & f & 0.92 & 3.30 & 2018-08-02 & 9 & 1 & 2 & 2018-09-24 & 53 & 0.002 & 2\\
1293302 & f & 1.02 & 4.60 & 2018-08-03 & 77 & 6 & 14 & 2019-08-23 & 385 & 0.012 & 2\\
1293303 & f & 0.93 & 3.20 & 2019-07-11 & 2210 & 31 & 196 & 2020-06-18 & 343 & 0.060 & 1\\
1293304 & f & 0.72 & 1.35 & 2019-07-11 & 0 & 0 & 0 & NA & 9 & NA & 1\\
1293305 & m & 0.72 & 0.90 & 2019-07-11 & 0 & 0 & 0 & NA & 0 & NA & 1\\
\addlinespace
1293310 & m & 0.72 & 1.40 & 2019-07-11 & 0 & 0 & 0 & NA & 32 & NA & 1\\
1293306 & m & 0.79 & 1.70 & 2019-07-12 & 0 & 0 & 0 & NA & 0 & NA & 1\\
1293312 & f & 0.91 & 2.80 & 2019-07-12 & 0 & 0 & 0 & NA & 25 & NA & 1\\
1293311 & m & 0.81 & 1.70 & 2019-07-12 & 64 & 1 & 6 & 2020-04-15 & 278 & 0.002 & 1\\
1293313 & f & 0.79 & 1.90 & 2019-07-12 & 455 & 21 & 62 & 2020-07-09 & 363 & 0.041 & 1\\
\bottomrule
\end{tabular}
\end{table}

\begin{figure}

\begin{minipage}[t]{0.33\linewidth}

{\centering 

\raisebox{-\height}{

\includegraphics{C:/Users/lotte.pohl/Documents/github_repos/MasterThesis_LottePohl/00_data/MustelusAsterias_scalebar.png}

}

}

\subcaption{\label{fig-taggingpics-1}A \emph{M. asterias} individual
after capture.}
\end{minipage}%
%
\begin{minipage}[t]{0.33\linewidth}

{\centering 

\raisebox{-\height}{

\includegraphics{C:/Users/lotte.pohl/Documents/github_repos/MasterThesis_LottePohl/00_data/MustelusAsterias_tagging_scalebar.png}

}

}

\subcaption{\label{fig-taggingpics-2}Tag implantation surgery.}
\end{minipage}%
%
\begin{minipage}[t]{0.33\linewidth}

{\centering 

\raisebox{-\height}{

\includegraphics{C:/Users/lotte.pohl/Documents/github_repos/MasterThesis_LottePohl/00_data/MustelusAsterias_suture_scalebar.png}

}

}

\subcaption{\label{fig-taggingpics-3}Suture after ADST implantation.}
\end{minipage}%

\caption{\label{fig-taggingpics}Images from the tagging of \emph{M.
asterias}. The white scale bars in the bottom left corners correspond to
5 cm. © Pieterjan Verhelst.}

\end{figure}

\hypertarget{tag-recoveries}{%
\subsubsection{Tag recoveries}\label{tag-recoveries}}

Upon writing (June, 2023), eight tags were recovered. Information about
the tags with available raw data is listed in Table~\ref{tbl-dstsum}.

\hypertarget{tbl-dstsum}{}
\begin{table}[H]
\caption{\label{tbl-dstsum}Information on the recovered tags. The death date was visually
determined from the raw dephlog and marks the date after which only a
tidal signal remains in the depth timeseries. The time at liberty is the
time period between the release and death date. The tag recapture date
was not recorded for tags 1293321, 1293308 and 1293304. }\tabularnewline

\centering
\begin{tabular}{llllrl}
\toprule
tag serial no. & sex & release date & death date & days at liberty & recapture date\\
\midrule
1293319 & f & 2018-07-19 & 2018-08-07 & 19 & 2018-08-21\\
1293322 & f & 2018-07-19 & 2018-08-07 & 19 & 2019-09-29\\
1293295 & m & 2018-07-19 & 2018-08-18 & 30 & 2018-09-27\\
1293321 & m & 2018-07-19 & 2019-11-16 & 485 & NA\\
1293308 & f & 2018-08-02 & 2019-08-03 & 366 & NA\\
\addlinespace
1293304 & f & 2019-07-11 & 2019-07-20 & 9 & NA\\
1293310 & m & 2019-07-11 & 2019-08-12 & 32 & 2019-09-03\\
1293312 & f & 2019-07-12 & 2019-08-06 & 25 & 2019-08-29\\
\bottomrule
\end{tabular}
\end{table}

6 out of 8 recovered tags logged for a month or less, only the tags with
the serial numbers 1293308 (f) and 1293321 (m) logged for a year,
allowing for the analysis of migration behaviour. These two tags will be
referred to as long term DST in the following, and their tag serial
numbers will be abbreviated to \texttt{308} and \texttt{321}. The rest
of the tags will be referred to as short term DST.

\hypertarget{ethics-statement}{%
\subsection{Ethics statement}\label{ethics-statement}}

The care and use of the animals complied with the Belgian Animal Welfare
Act, guidelines and policies as approved by the \emph{Ethische Commissie
Dierproeven} (project reference EC2017-080). All animal treatments were
undertaken in 2018 and 2019, and all tagging procedures were carried out
by competent and trained personal license holders.

\hypertarget{data-analysis-and-visualisation}{%
\subsection{Data Analysis and
visualisation}\label{data-analysis-and-visualisation}}

All data analyses were carried out using R \citep{R_2022}, primarily
using the packages \texttt{dplyr} \citep{dplyr} and \texttt{lubridate}
\citep{lubridate}. Geospatial data was queried from the marineregions
database \citep{claus_2014} in the case of (marine) boundaries
\citep[with the \texttt{mregions2}-package,][]{mregions2}, and EMODnet
Human Activities \citep{solaun_2021} data \citep[with the
\texttt{EMODnetWFS}-package,][]{EMODnetWFS}. All plots were generated
using the \texttt{ggplot2}-package \citep{ggplot2}. Packages used for
specific data analyses are indicated in the respective section.

\hypertarget{acoustic-detections}{%
\subsubsection{Acoustic Detections}\label{acoustic-detections}}

Detections were accessed from the ETN database
(https://www.lifewatch.be/etn/){[}https://www.lifewatch.be/etn/{]} using
the package \texttt{etn} \citep{etn}. Tags with only a single detection
were omitted from all further analyses. Acoustic detections were
visualised on an individual and acoustic receiver station level.

\hypertarget{sec-mm_dsts}{%
\subsubsection{Data Storage Tags}\label{sec-mm_dsts}}

Depth and temperature logs from the recovered tags were accessed via the
Marine Data Archive (MDA, dataset DOI:
(https://doi.org/10.14284/605){[}https://doi.org/10.14284/605{]}). The
temperature log was not included in any analysis. The first week of
depth values was excluded from all analyses to remove potential bias
resulting from the tagging procedure \citep[following][]{flavio_2021}.
The date of death, used to calculate the time at liberty for the
recovered tags, was visually extracted from plots of the raw depth
series, and marks the date after which only a tidal signal remains in
the depthlog \citep[characterised by cyclically rising and falling
depths within a rough 2 m depth range,][]{kvale_2006}.

The analyses described in the following were only performed for tags
that logged for \(\geq\) 365 days, i.e., tag 308 (female) and tag 321
(male). Those will be referred to as \emph{long term DST} in the
following, returned tags logging for less than a year will be referred
to as \emph{short term DST}. Summary statistics, i.e., median (since
depth did not follow a normal distribution), maximum and minimum depth
were calculated per date, day and night. For the day/night summaries,
sunrise and sunset times were extracted using the
\texttt{getSunlightTimes()}-function from the \texttt{suncalc}-package
\citep{suncalc}. The illuminated fraction of the moon was extracted
using the \texttt{moonAngle()}-function from the \texttt{oce} package
\citep{oce}.

A Savitzky-Golay smoothing filter \citep[ filter order \emph{p = 1},
filter length \emph{n = 5}]{press_1990} was applied on all summaries
using the \texttt{sgolayfilt()}-function from the
\texttt{signal}-package \citep{signal}. This particular filter smoothes
the data but retains spikes in the data better than other smoothing
filters like butterworth or averaging filters \citep[like rolling
mean,][]{schafer_2011}. The autocorrelation was calculated using the
\texttt{stats::acf()}-function. Autocorrelation refers to the serial
correlation of a data series with itself \citep{bartlett_1946},
resulting in an autocorrelation coefficient between -1 and 1 (-1
referring to perfect anti-correlation and 1 referring to perfect
correlation, 0 referring to no correlation at all) over the lag, which
refers to the overlap between the data series with itself.

A Fast Fourier Transform (FFT) was performed on all DST logs, using the
\texttt{base::fft()}-function. This method decomposes a signal in the
time domain (which is a combination of multiple sine waves with
different frequencies) into its components in the frequency domain.
Dominant frequencies (or rather periods, which is the inverse of the
frequency) give insights about the scale of periodicity of a signal
\citep[in the present case the signal is the depth-timeseries of the
sharks,][]{cochran_1967}.

Next, Continuous Wavelet Transform (CWT) was performed the daily summary
statistics of the long-term DST depthlogs were subjected to the
Continuous Wavelet Transform (CWT) using the wt() function from the
biwavelet-package (Gouhier et al., 2021). In contrast to the Fast
Fourier Transform (FFT), the CWT reveals not only the dominant
frequencies but also the specific time intervals when these frequencies
occur. Thus, the frequency composition of the time series in the time
domain can be examined. To achieve this, the resolution of both the
frequency and time domains was minimized. However, to ensure reliable
and accurate results despite the reduced resolution, the frequencies
were compared against white noise using a 𝜒2 test (Grinsted et al.,
2004). The CWT was performed on both the raw depthlog data and the
median depth per date, resulting in hourly and daily resolution of
periodicities, respectively. The wavelet scalograms of the raw depthlog
display the maximum power per day in the scalogram.

Next, Continuous Wavelet Transform (CWT, a specific type of Wavelet
Analysis) was performed on the daily summary statistics of the longterm
DST depthlogs, using the \texttt{wt()}-function from the
\texttt{biwavelet}-package \citep{biwavelet}. In contrast to Fourier
Transformation, Wavelet Analysis reveals not only the dominant
frequencies but also the specific time intervals when these frequencies
occur. Thus, the frequency composition of the time series in the time
domain can be examined. This, however, comes at the cost of minimised
resolution of both the frequency and time domain. To obtain robust and
valid results despite the resolution loss, frequencies were compared
against white noise using a \(\chi^2\) test \citep{grinsted_2004}. CWT
was performed on both the raw depthlog data and the median depth per
date, resulting in hourly and daily resolution of periodicities,
respectively. The wavelet scalograms of the raw depthlog display the
maximum power per day in the scalogram.

--\textgreater{}

\hypertarget{sec-results}{%
\section{Results}\label{sec-results}}

\hypertarget{sec-res-acoustic}{%
\subsection{Acoustic detections}\label{sec-res-acoustic}}

Two tags only had a single detection (serial numbers 1293296 and
1293301) and were thus omitted from all further analyses. The acoustic
detections of the remaining tags over time are shown in
Figure~\ref{fig-abacus}. Within the study period from July, 2018 to
August, 2020, there are no acoustic detections between November and
April in either year. 96.71\% of detections are in the WS1 area, and of
those, 68.26\% are at station OG10 (see \textbf{?@fig-mapstudyarea-3}).
Of the 11 males that were tagged, 4 were detected more than once. 3 of
the detected males were detected in the year following their tagging. Of
the 19 tagged females, 14 were detected more than once and 11
individuals were detected in the year following their tagging
(Figure~\ref{fig-abacus}).

Of the 4 detected males, 3 were detected inside the BPNS and one in the
WS1 array. Across all years, most detections (n = 2484) were in the
month of May, while most individuals (n = 10) were detected in August.
12 out of 18 individuals (67 \%) were detected in the year following
their release.

--\textgreater{}

\begin{figure}[H]

{\centering \includegraphics[width=1\textwidth,height=\textheight]{thesis_manuscript_files/figure-pdf/fig-abacus-1.pdf}

}

\caption{\label{fig-abacus}Acoustic detections of the tagged starry
smooth-hounds, the colour responds to the receiver area. The shapes with
black outlines refer to the tagging date and the sex of the individual.}

\end{figure}

The acoustic detections at each receiver station of female \emph{M.
asterias} within the Western Scheldt are shown in
Figure~\ref{fig-heatmapdetections}. The receiver stations are ordered by
latitude (from north to south), and by receiver array (station OG10 to
W7 are in the WS1 array, and station borssele to 4 are in the WS2
array). Most detections (n = 2103) were at station OG10 in October,
2019. This is almost twice as much as the second most detections (n =
1097 at the same station in May, 2020). Most individuals (n = 6) were
detected at station WN2 in August and September, 2018.

\begin{figure}[H]

{\centering \includegraphics[width=1\textwidth,height=\textheight]{thesis_manuscript_files/figure-pdf/fig-heatmapdetections-1.pdf}

}

\caption{\label{fig-heatmapdetections}Heatmap of the acoustic detections
per acoustic receiver of female \emph{M. asterias} in the Western
Scheldt (receiver arrays \emph{WS1}, \emph{WS2} and \emph{WS3}).}

\end{figure}

Acoustic detections of female \emph{M. asterias} at receiver station
OG10 in 2019 are shown in Figure~\ref{fig-heatmapOG10}. The acoustic
detections of an individual at a day (47.09 \(\pm\) 48.4, median = 35)
are maximal for tag 299 with n = 232 detections on October 13, 2019.
Tags 313, 308, 299 and 297 show a cluster of about 14 - 21 days with
consecutive detections. Tag 308 has two clusters of subsequent
detections, between May 13 and June 09, and between July 16 and August
12, 2019.

\begin{figure}[H]

{\centering \includegraphics[width=0.9\textwidth,height=\textheight]{thesis_manuscript_files/figure-pdf/fig-heatmapOG10-1.pdf}

}

\caption{\label{fig-heatmapOG10}Heatmap of the acoustic detections in
2019 of female \emph{M. asterias} individuals at receiver station OG10.
The Residency Index (RI, i.e.~the number of days detected divided by the
period between the first and last detection among all individuals at
station OG10 in 2019) is displayed next to the tag serial number.}

\end{figure}

\hypertarget{sec-results-dst}{%
\subsection{Data Storage Tags}\label{sec-results-dst}}

\hypertarget{raw-depth-and-temperature-logs}{%
\subsubsection{Raw depth and temperature
logs}\label{raw-depth-and-temperature-logs}}

The depth and temperature logs of tags 308 and 321 (the two long term
DST) are shown in Figure~\ref{fig-dst308-1} and
Figure~\ref{fig-dst321-1}, respectively. Example subsets from summer and
winter are also shown.

The full depthlog of tag 308 (female, Figure~\ref{fig-dst308-1}),
comprising 366 days, shows depth between 0 and 20 m from the time of
tagging in August, 2018, to the beginning of October. Then the depth
gets deeper over a period of about two weeks, and between mid-November
and mid-December the sensor recorded depths between 50 and 75 m. From
December, 2018 until March, 2019, the depth ranges between 75 m and
close to the water surface. From mid-March on, the depth remains
shallower than 60 m. From May to June, 2019, depth is relatively
constant around 17 m, from mid-June to mid-July it is relatively
constant around 10 m and until the sensor stops recording the depth goes
back to just under 20 m. Figure~\ref{fig-dst308-2} shows an example
subset of the depthlog of tag 308 in September, 2018. Roughly, the depth
alternates between 12 - 14 and 16 - 20 m in a sinusoidal pattern with a
wavelength of about 12 hours. The subset in Figure~\ref{fig-dst308-3}
shows a period of a month between February and March, 2019. Here,
periods of smaller depth ranges from 40 to 60 m (between February 11 and
18 and between February 25 and March 3) can be observed with periods of
larger depth ranges from 5 to 75 m (between February 18 and 24) in
between, can be observed.

The full depthlog of tag 321 (male, Figure~\ref{fig-dst321-1}) is 485
days long and shows depth ranges between 0 and 30 m from tagging until
the end of October, 2018. Then, depth gets deeper over a period of
roughly 10 days, stays at 35 to 50 m until December, 2018 and then
ranges from 5 m to 50 m until March, 2019. Until May, 2019 depth ranges
between 10 and 35 m, then the sensor recorded depth values of 60 m on
May 03, 2019. After that, depth gets shallower with ranges from 0 to
mostly 20 m until mid-September. Until the end of the depthlog November
15, 2019, depth gets deeper to around 35 m and then to around 40 m. The
example depthlog of September, 2018 in Figure~\ref{fig-dst321-2} shows a
depth range between 10 and 15 m during daytime (a sinusoidal pattern
similar to the one in Figure~\ref{fig-dst308-2}), and minimum depths of
around 2 m during night time. The subset from February to March, 2019
(Figure~\ref{fig-dst321-3}) shows depth ranges from 5 to 55m with no
observable pattern of larger or smaller depth ranges on a weekly scale.

The raw depth and temperature logs of the short term DST are included in
\textbf{Annex A}. The short term DST logged depth and temperature data
for 22 \(\pm\) 9 days, reaching maximum water depths of 53.51 m \(\pm\)
10.24 m.

\hypertarget{sec-resacf}{%
\subsubsection{Autocorrelation}\label{sec-resacf}}

The autocorrelograms of the two long term DST logs (computed using the
Savitztky-Golay filtered daily median depth) are shown in
Figure~\ref{fig-acf}. An autocorrelation value of -1 refers to perfect
anti-correlation, 1 referring to perfect correlation, and 0 refers to no
correlation at all. Autocorrelation is above 0.5 at lags below 35 days
for tag 308 (f), reaching 0 at lags of 81 and 264 days. The minimum is
at 170 days and the local maximum at 300 days, see
Figure~\ref{fig-acf-1}. Autocorrelation above 0.5 is attained below lags
of 24 days for tag 321 (m). Autocorrelation is 0 at 75, 256 and 383 days
and is minimal at a lag of 138 days. A local maximum occurs at 345 days,
see Figure~\ref{fig-acf-2}.

\begin{figure}

\begin{minipage}[t]{\linewidth}

{\centering 

\raisebox{-\height}{

\includegraphics{thesis_manuscript_files/figure-pdf/fig-dst308-1.pdf}

}

}

\subcaption{\label{fig-dst308-1}Depthlog, subsampled to one value every
30 minutes.}
\end{minipage}%
\newline
\begin{minipage}[t]{0.50\linewidth}

{\centering 

\raisebox{-\height}{

\includegraphics{thesis_manuscript_files/figure-pdf/fig-dst308-2.pdf}

}

}

\subcaption{\label{fig-dst308-2}Example subset from September 2018.}
\end{minipage}%
%
\begin{minipage}[t]{0.50\linewidth}

{\centering 

\raisebox{-\height}{

\includegraphics{thesis_manuscript_files/figure-pdf/fig-dst308-3.pdf}

}

}

\subcaption{\label{fig-dst308-3}Example subset from February and March
2019.}
\end{minipage}%

\caption{\label{fig-dst308}Data from the recovered tag 308. This female
shark was tagged on August 2, 2018.}

\end{figure}

\begin{figure}

\begin{minipage}[t]{\linewidth}

{\centering 

\raisebox{-\height}{

\includegraphics{thesis_manuscript_files/figure-pdf/fig-dst321-1.pdf}

}

}

\subcaption{\label{fig-dst321-1}Depthlog, subsampled to one value every
30 minutes.}
\end{minipage}%
\newline
\begin{minipage}[t]{0.50\linewidth}

{\centering 

\raisebox{-\height}{

\includegraphics{thesis_manuscript_files/figure-pdf/fig-dst321-2.pdf}

}

}

\subcaption{\label{fig-dst321-2}Example subset from September 2018.}
\end{minipage}%
%
\begin{minipage}[t]{0.50\linewidth}

{\centering 

\raisebox{-\height}{

\includegraphics{thesis_manuscript_files/figure-pdf/fig-dst321-3.pdf}

}

}

\subcaption{\label{fig-dst321-3}Example subset from February and March
2019.}
\end{minipage}%

\caption{\label{fig-dst321}Data from the recovered tag 321. This female
shark was tagged on August 2, 2018.}

\end{figure}

\begin{figure}

\begin{minipage}[t]{\linewidth}

{\centering 

\raisebox{-\height}{

\includegraphics{thesis_manuscript_files/figure-pdf/fig-acf-1.pdf}

}

}

\subcaption{\label{fig-acf-1}Tag 308 (f)}
\end{minipage}%
\newline
\begin{minipage}[t]{\linewidth}

{\centering 

\raisebox{-\height}{

\includegraphics{thesis_manuscript_files/figure-pdf/fig-acf-2.pdf}

}

}

\subcaption{\label{fig-acf-2}Tag 321 (m)}
\end{minipage}%

\caption{\label{fig-acf}Autocorrelograms of daily median depth.}

\end{figure}

\hypertarget{sec-results-dst-summary}{%
\subsubsection{Summary statistics}\label{sec-results-dst-summary}}

The daily summary statistics of the depthlogs of tags 308 and 321 are
displayed in Figure~\ref{fig-dstsum308} and Figure~\ref{fig-dstsum321},
respectively. Summaries per date, day and night are included. The solid
lines show the Savitzky-Golay filtered daily median. The ribbons span
between the Savitzky-Golay filtered daily maximum and minimum depth. The
dotted line displays the fraction of the moon that is illuminated,
representing the moon phase.

\begin{figure}

\begin{minipage}[t]{\linewidth}

{\centering 

\raisebox{-\height}{

\includegraphics{thesis_manuscript_files/figure-pdf/fig-dstsum308-1.pdf}

}

}

\subcaption{\label{fig-dstsum308-1}Median, maximum and minimum depth per
date.}
\end{minipage}%
\newline
\begin{minipage}[t]{0.50\linewidth}

{\centering 

\raisebox{-\height}{

\includegraphics{thesis_manuscript_files/figure-pdf/fig-dstsum308-2.pdf}

}

}

\subcaption{\label{fig-dstsum308-2}Summary statistics per day.}
\end{minipage}%
%
\begin{minipage}[t]{0.50\linewidth}

{\centering 

\raisebox{-\height}{

\includegraphics{thesis_manuscript_files/figure-pdf/fig-dstsum308-3.pdf}

}

}

\subcaption{\label{fig-dstsum308-3}Summary statistics per night.}
\end{minipage}%

\caption{\label{fig-dstsum308}Summary statistics, per date, day and
night. Median depth as a solid line, maximum and minimum depth as shaded
ribbon. All depth summaries were Savitzky-Golay filtered. Illuminated
fraction of the moon as a dotted line.}

\end{figure}

Figure~\ref{fig-dstsum308-1} displays a daily maximum of about 20 m
until end September and then, until the start of October the daily
median depth gets deeper, reaching 75 m. Until the beginning of December
the shark uses waters with minimum daily depths not shallower than 50 m.
After that until mid March the daily maximum depth is between 40 and 68
m and the daily minimum depth is between 5 and 50 m. After March 19,
2019, the daily median depth is always shallower than 60 m. Between
March until the end of the time series, some plateaus in median depth
occur: Around 35 m median depth from end March to end April, around 17 m
depth from mid May to the start of June and from end July onward, and
around 8 m depth between June 11 and July 13, 2019.

The minimum depth during the night is significantly shallower than
during the day (one-sided Welch Two Sample t-test, \(\alpha\) = 0.95,
\emph{p} = \ensuremath{8.0829717\times 10^{-9}}). From December, 2018 to
May, 2019, the minimum and median depth per day
(Figure~\ref{fig-dstsum308-2}) show peaks following a rough biweekly
pattern, getting the shallowest when the illuminated moon fraction is
maximal and minimal. During the night (Figure~\ref{fig-dstsum321-3}), a
rather monthly pattern can be observed for that time period. Here, the
median and minimum depth are shallowest when the illuminated fraction of
the moon is maximal.

\begin{figure}

\begin{minipage}[t]{\linewidth}

{\centering 

\raisebox{-\height}{

\includegraphics{thesis_manuscript_files/figure-pdf/fig-dstsum321-1.pdf}

}

}

\subcaption{\label{fig-dstsum321-1}Median, maximum and minimum depth per
date.}
\end{minipage}%
\newline
\begin{minipage}[t]{0.50\linewidth}

{\centering 

\raisebox{-\height}{

\includegraphics{thesis_manuscript_files/figure-pdf/fig-dstsum321-2.pdf}

}

}

\subcaption{\label{fig-dstsum321-2}Summary statistics per day.}
\end{minipage}%
%
\begin{minipage}[t]{0.50\linewidth}

{\centering 

\raisebox{-\height}{

\includegraphics{thesis_manuscript_files/figure-pdf/fig-dstsum321-3.pdf}

}

}

\subcaption{\label{fig-dstsum321-3}Summary statistics per night.}
\end{minipage}%

\caption{\label{fig-dstsum321}Summary statistics, per date, day and
night. Median depth as a solid line, maximum and minimum depth as shaded
ribbon. Illuminated fraction of the moon as a dotted line.}

\end{figure}

The male shark (tag 321, Figure~\ref{fig-dstsum321-1}) uses waters
shallower than 20 m between August and October, 2018. From October 21
until November 03, 2018, the daily median depth gets consistently
deeper, reaching depths of around 45 m throughout November, 2018.
Between December 09, 2019 and January 21, 2019, the daily median depth
ranges from 22 to 33 m and the maximum and minimum depth between 53 and
16 m, respectively. After dropping to 39 m on January 28, 2019, the
daily median depth gets shallower until April 30, reaching depths of 22
m. Then, the daily median depth increases substantially between April 27
and May 05, ranging between 22 m and 6 m until October 05. From October
08 to 19, 2019, the daily median depth changes from 17 to 45 m. Until
the log stops on November 16, the daily median depth ranges between 45
and 32 m.

The minimum depth during the night is significantly shallower than
during the day (one-sided Welch Two Sample t-test, \(\alpha\) = 0.95,
\emph{p} = \ensuremath{1.6887159\times 10^{-19}}). The median depth per
day (Figure~\ref{fig-dstsum321-2}) shows slight peaks in a biweekly
pattern between January and July, 2019. The peaks of the median depth at
night (Figure~\ref{fig-dstsum321-3}) show no visually noticeable
pattern.

\hypertarget{wavelet-analysis}{%
\subsubsection{Wavelet analysis}\label{wavelet-analysis}}

CWT was performed on both the raw depthlog, subsampled to one value
every 10 minutes, and the Savitzky-Golay filtered daily median depth.
The resulting wavelet scalograms for the two longterm DST depthlog are
shown below, in Figure~\ref{fig-waveletresults308} and
Figure~\ref{fig-waveletresults321}.

\begin{figure}

\begin{minipage}[t]{\linewidth}

{\centering 

\raisebox{-\height}{

\includegraphics{thesis_manuscript_files/figure-pdf/fig-waveletresults308-1.pdf}

}

}

\subcaption{\label{fig-waveletresults308-1}Raw depth log, subsampled to
one value every 10 minutes.}
\end{minipage}%
\newline
\begin{minipage}[t]{\linewidth}

{\centering 

\raisebox{-\height}{

\includegraphics{thesis_manuscript_files/figure-pdf/fig-waveletresults308-2.pdf}

}

}

\subcaption{\label{fig-waveletresults308-2}Daily median depth}
\end{minipage}%

\caption{\label{fig-waveletresults308}Wavelet transform results for tag
308 (f). The fully opaque areas refer to periods that significantly
differ from white noise.}

\end{figure}

In the scale of hours, Figure~\ref{fig-waveletresults308-1} shows high
prevalence across all periods in October 2018, and between December,
2018 and May, 2019. In that time period significant periodicities
spanning from 4 to 14 hours exist in a rough interval of 2 weeks. To a
lesser extent, period bands of 12 hours are observable from August to
September, 2018 and May to August, 2019.

Periods in the scale of days for tag 308 show the prevalence of a
roughly 14 day period in October, 2018 and between December, 2018 and
June, 2019. During this period, also periodicities of about 30 days are
significantly present in the daily median depth.

\begin{figure}

\begin{minipage}[t]{\linewidth}

{\centering 

\raisebox{-\height}{

\includegraphics{thesis_manuscript_files/figure-pdf/fig-waveletresults321-1.pdf}

}

}

\subcaption{\label{fig-waveletresults321-1}Raw depth log, subsampled to
one value every 10 minutes.}
\end{minipage}%
\newline
\begin{minipage}[t]{\linewidth}

{\centering 

\raisebox{-\height}{

\includegraphics{thesis_manuscript_files/figure-pdf/fig-waveletresults321-2.pdf}

}

}

\subcaption{\label{fig-waveletresults321-2}Daily median depth}
\end{minipage}%

\caption{\label{fig-waveletresults321}Wavelet transform results for tag
321 (m). The fully opaque areas refer to periods that significantly
differ from white noise.}

\end{figure}

For tag 321 on the hourly scale (Figure~\ref{fig-waveletresults321-1})
punctually high periodicities across the range from 4 to 64 hours are
present at the beginning and end of October, 2018, throughout May, 2019,
and at the beginning of October, 2019. Significant periodicities are
sporadically present during these periods, between roughly 4 and 12
hours. A period band of 24 hours is present August to October, 2018,
January and February, 2019 and between July and October, 2019. A period
band of 12 hours is present to a lesser extent during the same periods.

The daily wavelet scalogram in Figure~\ref{fig-waveletresults321-2}
shows significantly present periodicities of 16 to 24 days between
October, 2018 and August, 2019. From September to November, 2019,
periods between about 8 and 50 days are present. From December, 2018 to
February, 2019 and from mid-April to June, 2019, periods between about 7
and 14 days are present.

\hypertarget{acoustic-detections-and-depthlog}{%
\subsection{Acoustic detections and
depthlog}\label{acoustic-detections-and-depthlog}}

Figure~\ref{fig-adst308} shows the raw depthlog of tag 308, the female
shark that logged data for one year, superposed with the acoustic
detections of that individual. After the DST had stopped recording on
August 03, 2019, the shark was recorded at station OG10. And on October
05, 2019 it was recorded at the Birkenfels station in the BPNS.

\begin{figure}

\begin{minipage}[t]{\linewidth}

{\centering 

\raisebox{-\height}{

\includegraphics{thesis_manuscript_files/figure-pdf/fig-adst308-1.pdf}

}

\caption{\label{fig-adst308}Raw depthlog of tag 308 (female) in summer
2019 (subsampled to one depth value every 30 minutes) and acoustic
detections from the Permanent Belgian Acoustic Receiver Network.}

}

\end{minipage}%

\end{figure}

\hypertarget{discussion}{%
\section{Discussion}\label{discussion}}

The two lines of research encompass the assessment of seasonality and
sex-specificity regarding the presence of \emph{M. asterias} in the BPNS
and the Western Scheldt, and its vertical movement behaviour. The
following section will interpret and discuss the chosen analyses and
their results with respect to the research questions stated in
Section~\ref{sec-intro-aims}.

\hypertarget{sec-disc-seasonalpresencefemales}{%
\subsection{\texorpdfstring{Presence of \emph{M. asterias} in the
Western Scheldt and
BPNS}{Presence of M. asterias in the Western Scheldt and BPNS}}\label{sec-disc-seasonalpresencefemales}}

This section discusses the results of the acoustic detections (see
Section~\ref{sec-res-acoustic}) in the context of the three research
questions research line 1. Those which involve seasonal patterns (RQ 1),
differences between sexes (RQ 2), and spatial preferences (RQ 3) within
the BPNS and the Western Scheldt.

There are no acoustic detections at all between November and April in
all years with acoustic detections (2018-2020, Figure~\ref{fig-abacus}).
This suggests a seasonal presence of \emph{M. asterias} in the Western
Scheldt between April and November. Most females (n = 7) were detected
in the months of July, August and September, in the WS1 array. This
suggests that these months are especially relevant for \emph{M.
asterias} females in the Scheldt estuary.

According to the acoustic detections, a hotspot of seasonal presence of
mature \emph{M. asterias} females is around the receiver OG10 (66.01 \%
of detections, and 10 out of 14 detected females, see
Figure~\ref{fig-heatmapOG10}). According to fishers, the area offshore
from Zoetelande (north of Dishoek) is a \emph{M. asterias} fishing
hotspot, so the coastal waters between Diskoek and Zoetelande are
suggested to be important for the species. Whether or not the potential
aggregations of mature females is related to parturition should be
subject of further studies. There are several indications, however, that
point towards the Western Scheldt and/or its surrounding waters to be a
pupping area for \emph{M. asterias}. The Eastern Scheldt is a confirmed
pupping ground \citep{breve_2016}, and during the tagging for this study
in August, 2018, two neonates \citep[around 30 cm TL,][]{farrell_2010}
were caught\footnote{Verhelst, Pieterjan (\emph{personal
  communication}), April 25, 2023.}.

Not more than one male was ever detected in a month. Three males were
detected in the BPNS (one each on August, 2018, and in April and May,
2019), and one male was detected in the WS1 array in April, 2020. Three
of those four individuals were detected in the year following their
release, indicating philopatric behaviour for males.
\citet{griffiths_2020} reported direct philopatry for a female \emph{M.
asterias}, and \citet{breve_2016} suggested philopatry for the species
based on tag and recapture data. The return of the three males (tags
311, 315 and 316) into the BPNS and the WS1 area could be the first
direct reporting of philopatry for male \emph{M. asterias}.

Another important fact to consider is the biannual reproductive cycle of
\emph{M. asterias} and whether that implies differing seasonal habitat
use between years of pregnancy and years without a pregnancy. Females
were either detected at a lot of days (20 - 40) or few (5 - 10) days
during summer 2019 (see Figure~\ref{fig-heatmapOG10}). This could
indicate different behaviourial states between the females with many and
few acoustically detected days, further supported by the wide dispersion
in Residency Indices which are shown in the figure.

The periods of 10 to 20 days of subsequent detections of females
(especially tags 299, 308 and 313) at station OG10 suggest partial
residency of these individuals. Uncovering potential reasons for this
behaviour and A relevant aspect to study in the future would be to
explore reasons of this behaviour and whether the females could be
giving birth during their residential clusters at station OG10. As such,
it could be helpful to assess if adult females are pregnant at the time
of release, through ultrasonography or the use of sex steroid hormones,
both of which have already been employed to determine pregnancy in
sharks
\citep{awruch_2014, smukall_2019, anderson_2018, fujinami_2020, fujinami_2021}.
This would allow for further clarification if the Western Scheldt is
used as a pupping ground, and if it might be a relevant area for
non-pregnant females, too.

An area with little presence of the tagged \emph{M. asterias}
individuals is the Western Scheldt upstream past Flushing. Shallow and
sheltered coastal habitats are thought to be suitable pupping grounds
for sharks in general \citep{speed_2010} and for \emph{M. asterias} in
particular \citep{ellis_2004, ellis_2005}. The low number of
acoustically detected sharks (38 detections in the WS2 array throughout
the whole study period from 2018 to 2020) indicates that the waters past
Flushing are not as frequently used as the waters around Dishoek. The
Scheldt estuary can be considered a suitable pupping habitat for
\emph{M. asterias} due to high nutrient levels and food availability of
the Continental coastal water \citep[which enters the estuary through
tidal currents,][]{wolff_1973}, and monthly water temperatures above 15
°C between June and September \citep{speed_2010}. The fact that few
sharks were detected upstream past Flushing could be due to low
salinities, or to anthropogenic influences like dredging. This, however,
should be further investigated in follow-up studies. \emph{M. asterias}
might leave the area of the Western Scheldt during winter because of
decreasing water temperature \citep{breve_2016}, a main predictor of
elasmobranch occurrence \citep{martin_2010}. The warmer waters in the
Channel area might be favourable especially for gestating females since
higher water temperatures increase metabolic rate and thereby improve
embryo development \citep{hurst_1999}.

\hypertarget{sec-disc-dst-movementpatterns}{%
\subsection{\texorpdfstring{Depth utilisation of \emph{M.
asterias}}{Depth utilisation of M. asterias}}\label{sec-disc-dst-movementpatterns}}

The following section discusses the analysis results of the two long
term DST logs (of tag 308, female, and tag 321, male) in the light of
the second research line, which entails the depth utilisation of
\emph{M. asterias}. The two research questions asked involve seasonal
patterns (RQ 4) and differences in depth use between sexes (RQ 5).

Overall, cyclicity can be observed for both the female (tag 308) and the
male (tag 321) shark logging data for more than a year. The
autocorrelograms (Figure~\ref{fig-acf}) show similar seasonal patterns
with an autocorrelation of zero at lags of 81 and 264 days for the
female, and 75 and 256 days for the male. The lags correspond to roughly
3 and 9 months, suggesting a change of vertical movement behavior at
these time scales. Global minima lie at lags of 170 and 138 days for
female and male, respectively. These lags roughly correspond to 5 months
and indicate that vertical behaviour is anti-correlated within this time
scale, further underlining the observation of higher vertical activity
and the use of deeper waters during winter and a contrary behaviour with
low vertical activity use of shallow waters during summer. Local
autocorrelation maxima lie at 300 and 345 days for female and male,
respectively. This corresponds to 10-12 months and points towards a
yearly cyclicity for both the male and the female shark. In general,
owing to the demersal lifestyle of \emph{M. asterias}, shallower depths
in the depthlog indicate a shoreward movement along the bottom, as
opposed to a vertical movement into the water column. If, for example,
the depthlog shows shallower depths during the night (as can be seen in
Figure~\ref{fig-dst321-2}) this suggests a movement into shallower,
coastal waters to feed.

\hypertarget{summer}{%
\subsubsection{Summer}\label{summer}}

From both recovered DSTs that logged data for over a year, the daily
vertical range and median depth are shallower in summer (June --
September, around 17 m median depth for the female and around 10 m
median depth for the male, both with ranges between 0 and 20 m) than in
winter (December -- March). The wavelet scalograms of the raw depthlogs
(Figure~\ref{fig-waveletresults308-1} and
Figure~\ref{fig-waveletresults321-1}) show a difference in periodicity
on the hourly scale between males and females. Both in the summer of
2018 and 2019, the female shows mainly a periodicity of 12 h (indicated
by band of high power in red colour), which is a pure tidal signal
\citep[since from one high tide to the next one it takes approximately
12 h, the same holds true for low tide,][]{shepard_2006}. In 2018, there
are some periodicities of 24 h (represented by yellow colours) as
opposed to 2019, where the power of the 24 h period is below zero. This
could be an indicator of the biannual reproductive cycle \citep[with
females feeding when they are not pregnant, and they stop feeding during
the migration to parturition grounds and parturition
itself,][p.~23]{michael_2006} but this requires further investigation in
the future. The 12 h periodicity, however, is the most prevalent one in
both summer for the female and thus implies that the female itself
expresses no to little vertical activity and is only resting on the
bottom (observable in the example subset of the raw depthlog in
Figure~\ref{fig-dst308-2}). The male, on the other hand, shows a
periodicity of 24 h. This can be seen as a high-power band in red
colours at the 24 h period in the wavelet scalogram on the hourly scale
(Figure~\ref{fig-waveletresults321-1}) and in Figure~\ref{fig-dst321-2}.
This suggests vertical migration, which has been observed in several
shark species, for example the spiny dogfish \emph{Squalus acanthias}
\citep{carlson_2014} and the basking shark \emph{Cetorhinus maximus}
\citep{shepard_2006}. Diel vertical migration is believed to be related
to foraging and feeding, following the nightly vertical migration of
zooplankton to feed \citep{griffiths_2020}.

\hypertarget{sec-disc-wintermig}{%
\subsubsection{Migration into winter habitat}\label{sec-disc-wintermig}}

From September 25 on for the female and October 21 for the male, the
daily median depth gets progressively deeper over a period of around 15
to 20 days, which potentially is the migration period of these
individuals. Interestingly, both individuals go to the maximum depth
logged by the tag within the logging period directly the potential
migration into the winter habitat. During this time, the female shark
goes to depths below the tag's measuring limit of 68 m, as indicated by
a straight line in the raw depthlog (see Figure~\ref{fig-dst308-1}).
This provides further evidence of the species utilizing deeper waters
than previously thought, as argued by \citet{griffiths_2020} that found
a maximum recorded depth of 118 m in a female shark in December
\citep[S4]{griffiths_2020}. The male shark goes to depths between 40 and
45 m, both in 2018 and 2019. Since this tag logged for 485 days, the
migration into the winter habitat was recorded for two years. The start
date of the potential winter migration in 2019 differs to the one in
2018 by 10 days. The autocorrelation of the depthlog of tag 321 has a
local maximum of 345 days see~\ref{fig-acf-2} which further reflects the
almost perfect yearly cyclicity of tag 321. Both individuals stay at
their respective maximum depths for approximately one month (the female
between the end of October until the end of November, 2018 and the male
between the start of November until the start of December, 2018). This
behaviour could be related to resting after the migration or escaping
potential predators. The behaviour after the completion of the migration
was not further assessed at this point and could be subject of future
studies.

\hypertarget{winter}{%
\subsubsection{Winter}\label{winter}}

During the winter months between December and March, both individuals
show roughly biweekly periodicities (see
Figure~\ref{fig-waveletresults308-2} and
Figure~\ref{fig-waveletresults321-2}). This vertical movement behaviour
could be linked to moon phases, which result in a 14-day cycle of neap
and spring tides \citep{shepard_2006}. The female shark shows a rather
biweekly vertical movement pattern during night and and a monthly
pattern for depth use during the day (Figure~\ref{fig-dstsum308-2} and
Figure~\ref{fig-dstsum308-3}), In the raw depthlog, periods of resting
and subsequent periods of high vertical activity are visually
discernable (Figure~\ref{fig-dst308-3}). Both of these periodicities can
be seen in the wavelet scalogram at daily scale
(Figure~\ref{fig-waveletresults308-2}). During the night, this
individual goes to depth shallower than 10 m between December, 2018 and
May, 2019. The peaks in minimum depth align well with a full moon,
potentially reflecting visual predation behaviour coupled with diel
vertical migration, i.e., the shark moves to shallower depths during
full moon (when the moonlight is most intense), feeding on crustaceans
that follow the nocturnal zooplankton migration. Visual foraging and
preying of \emph{M. asterias} was already assumed by
\citet{griffiths_2020}. During the day, median and minimum depth are
deepest at full moon and new moon. These moon phases generally result in
the most intense tidal currents throughout a moon cycle
\citep{arnold_1994}. \citet{shepard_2006} found the same pattern of high
vertical activity at full and new moon in one tagged Basking shark
(\emph{Cetorhinus maximus}), and \citet{graham_2006} reported lunar
periodicity in vertical behaviour of one whale shark (\emph{Rhincodon
typus}). The exact strength of tidal currents, however, depends on the
bathymetry of the area, and since the location of the shark is not
exactly known, interpretations regarding tidal currents should be taken
cautiously. Since the female, however, is thought to have migrated into
the English Channel \citep[an area with exceptionally high tidal
currents,][]{salomon_1993}, it is a plausible explanation that this
individual was resting on the bottom during periods of intense tidal
currents (as can be seen in Figure~\ref{fig-dst308-3}), and moving to
shallower waters to visually feed at full moon.

For the male shark, the biweekly periodicity is less present, in
general, and no apparent periodicity can be visually detected during the
night (Figure~\ref{fig-dstsum321-2} and Figure~\ref{fig-dstsum321-3}).
This could be an indication of a different feeding behaviour of the
male, involving less visual predation. The hourly wavelet scalogram
(Figure~\ref{fig-waveletresults321-1}) shows the presence of 24 h
periodicities in January and February, 2019, but the daily wavelet
scalogram (Figure~\ref{fig-waveletresults321-2}) does not contain
significantly present monthly periodicities. This suggests that the male
shark relies less on the moonlight to predate than the female shark. The
significant presence of 14 day periods between December, 2018 and May,
2019, however, could potentially relate to tidal currents. This is less
visually observable from the median and minimum depth during the day
(Figure~\ref{fig-dstsum321-2}). The male shark is thought to move
northwards from the Scheldt Estuary during its winter migration where
tidal currents are generally less intense than in the English Channel
\citep{salomon_1993, arnold_1994}. This could be a potential explanation
for the weaker biweekly periodicity in the male shark's median and
minimum depth. The male does not show clearly detectable periods with
pure resting behaviour during winter (as shown in
Figure~\ref{fig-dst321-3}). Whether this difference in vertical
behaviour patterns is sex-specific and related to a potential gestation
of the female shark, or to different feeding modes, could be a point of
further investigation in the future.

In general, the effect of the moon illumination and tidal currents would
be interesting to study further since this can give relevant insights
about (potentially sex-specific) winter behaviour of \emph{M. asterias},
for example through comparing wavelet scalograms from median depth
during the day and during the night.

\hypertarget{sec-disc-summermig}{%
\subsubsection{Migration into summer habitat}\label{sec-disc-summermig}}

In spring, roughly between April and May, the daily median depth gets
shallower in a stepwise manner for both individuals which potentially
reflects the migration into the summer habitat. The female (tag 308) was
detected in the receiver areas of the BPNS and WS1 in the year following
its release, proving the individual's return, and providing further
evidence of philopatric behaviour for the species, as suggested by
\citet{breve_2016} and first directly reported by
\citet{griffiths_2020}. The start and duration of the summer migration
cannot be identified as clearly from the daily median depth as the
winter migration since the depth does not decrease drastically over a
period of few weeks. From the wavelet scalograms of the raw depthlog
(Figure~\ref{fig-waveletresults308-1} and
Figure~\ref{fig-waveletresults321-1}) the time of potential migration
into the summer habitat shows a band of high power throughout periods
from 4 to 64 hours (i.e., high prevalence of these periods), albeit this
is does not significantly differ from white noise (and is thus slightly
transparent). Since the winter period also involves vertical activity,
there is no clearly visible change in periodicity patterns between
winter months up until March and the months of April and May, during
which the daily median depth decreases stepwise for the female shark.
During April 2019, the male shark does not show any prevailing
periodicity patterns in the wavelet scalogram at hourly scale
(Figure~\ref{fig-waveletresults321}-). The month of May 2019, however,
shows prevalent periodicities between 12 and 64 h, potentially
indicating migration behaviour. There is decreased vertical activity
compared to the winter months from mid-May onwards for the female and
from June on for the male, indicating that the individuals might have
arrived at their summer habitat by then. At this point, knowledge about
the exact spectral signature of \emph{M. asterias} migration behaviour
is lacking and it is suggested to be further studied. Moreover, a more
detailed investigation of the difference in vertical movement between
the migration into the winter and the summer habitat could be an
interesting focus for future investigations.

\hypertarget{gained-ecological-knowledge-on-m.-asterias-in-the-context-of-m.-asterias-and-related-species}{%
\subsection{\texorpdfstring{Gained ecological knowledge on \emph{M.
asterias} in the context of \emph{M. asterias} and related
species}{Gained ecological knowledge on M. asterias in the context of M. asterias and related species}}\label{gained-ecological-knowledge-on-m.-asterias-in-the-context-of-m.-asterias-and-related-species}}

\emph{Mustelus asterias} is a common species in the North Sea and the
English Channel, both of which areas experience high anthropogenic
impact \citep{hernandezfarinas_2014}. Since knowledge about the species
and its relatives is still limited, the following section aims to give a
brief overview of how the knowledge gathered in this study fits into
already existing ecological knowledge about \emph{M. asterias} and
related species.

Comparisons about reproductive behaviour must be made cautiously since
not all members of the genus \emph{Mustelus} exhibit aplacental
viviparity like \emph{M. asterias}. \emph{M. asterias}, for instance,
was often mistaken for the common smooth-hound \emph{M. mustelus} before
the development of a simple genetic method to distinguish the two
species \citep{farrell_2009}, yet this relative is placentally
viviparous \citep{dasilva_2018}. The gummy shark \emph{M. antarcticus}
inhabits Australian waters and exhibits matrotrophic aplacental
viviparity like \emph{M. asterias}, making it well comparable
\citep{walker_2007}. Moreover the same study uncovered asynchronous
breeding cycles (of one and two years) related to different locations
for \emph{M. antarcticus}. This behaviour is also known to occur in
Squalid sharks \citep{braccini_2006}, but there is no evidence of
\emph{M. asterias} exhibiting different breeding cycle lengths. The
reproductive biology of \emph{M. asterias}, however, involves complex
characteristics like embryo asynchronism, for instance. This refers to
pups with different developmental stages, potentially resulting from
sperm storage and selective fertilization by the females
\citep{farrell_2010a}. While an in-depth stock assessment has already
been carried out for \emph{M. antarcticus} by \citet{pribac_2005}, this
is still lacking for \emph{M. asterias} and should be done
\citep{mccullyphillips_2015}.

Annual migration behaviour has not been investigated in depth for many
\emph{Mustelus} species. In fact, the gummy shark \emph{M. antarcticus}
is reported to be less mobile than other elasmobranchs in the area of
Western Australia \citep[according to a three-year acoustic monitoring
study involving 100 tagged \emph{M. antarcticus},][]{braccini_2017}, but
still able to cover distances of \textgreater{} 60 km per day, and
overall distances of almost 1000 km. The narrownose smooth-hound shark
\emph{Mustelus schmitti}, an aplacentally viviparious relative of
\emph{M. asterias} that inhabits waters in the south-west Atlantic, is
thought to exhibit similar seasonal migration patterns as \emph{M.
asterias}. \citet{elisio_2019} argues that although migration patterns
of \emph{M. schmitti} are still unknown, large-sized individuals are
increasingly abundant in deeper waters during autumn and winter
\citep{cortes_2011}, suggesting a seasonal migration to shallower
waters. The same study reported parturition events of \emph{M. schmitti}
in coastal waters during spring and summer, following warming
temperatures above 16 °C. These findings align well with conclusions
from \citet{breve_2016} that reported decreased presence of \emph{M.
asterias} in the Dutch Delta for temperatures \textless{} 13°C. The
water temperature of the two long term DST logs from the present study
is slightly above 16°C for the female and 15°C for the male in October
2018, when the two individuals are thought to start migrating into the
winter habitat, further supporting the findings of \citet{breve_2016}
and \citet{elisio_2019}.

Overall, the results of the present study go in line with already
gathered knowledge on \emph{M. asterias}. The geolocation model results
suggest that the male shark spends the winters in waters north of the
Scheldt Estuary and the female goes into the English Channel, as assumed
by \citet{breve_2020} and \citet{griffiths_2020}. Different
overwintering spots for the female and the male shark are further
supported by the presumable stronger effect of moon phases and tidal
currents on the vertical activity of the female \emph{M. asterias},
potentially sheltering from the intense tidal currents in the English
Channel. The present study, however, only involved year-long DST logs
from two individuals, so any inferences from this data have to be taken
cautiously. The high return rate (67 \%) of individuals into the Western
Scheldt as shown by the acoustic detections in the acoustic receiver
array further underlines philopatric behaviour in regard to the species'
summer habitat, the Southern North Sea \citep{griffiths_2020}.

\hypertarget{the-potential-of-adst-for-m.-asterias}{%
\subsection{The potential of ADST for M.
asterias}\label{the-potential-of-adst-for-m.-asterias}}

For the present study, a novel type of electronic combination tag was
utilized, the Acoustic Data Storage Tag, combining acoustic detections
with data storage logs. For maximum data overlap, the tag should both be
acoustically detected and retrieved (to access the depth and temperature
logs). This was the case for 3 out of the 30 tags deployed in this
study. Tag 319 was detected 7 times, and the individual (f) died 19 days
post-release, tag 295 was detected 21 times and the shark (m) died 30
days post-release, and tag 308 was detected 2067 times and this female
was acoustically detected last 366 days post-release. Thus, only one tag
(tag 308) shows sufficient overlap between acoustic detections and the
depth and temperature logs. Figure~\ref{fig-adst308} shows that tag 308
was detected at the Birkenfels receiver station (which is in the BPNS)
63 days after the DST depthlog stopped. This suggests the departure of
the individual from the BPNS and the Scheldt Estuary and aligns almost
perfectly with the depthlog of that time one year before, where the
female went to waters deeper than 25 m from October 06, 2018, onwards.
While the annual cyclicity for the male could already be shown from its
depthlog since it logged well over a year (see
Section~\ref{sec-disc-wintermig}), the detection of tag 308 at the
Birkenfels station points towards an almost perfect annual cyclicity for
this individual, too. Generally, ADST are a well-suited tracking
technology for \emph{M. asterias}, owing to their seasonally differing
resident and migratory behaviour. It allows for gaining high-resolution
presence/absence data within the areas of receiver networks and insights
about the vertical movement of an individual independent of the acoustic
detections, given that the tag is retrieved. Moreover, the acoustic
detections provide punctual ground truth for geolocation modelling. This
is especially helpful in highly mixed areas with homogeneous bathymetry
like the Scheldt Estuary (such areas generally result in limited
reliability of geolocation models as the models cannot make out a
definite position by depth and temperature if these variables are
roughly constant throughout the area).

\hypertarget{limits-and-biases-of-the-telemetry-methods-used}{%
\subsubsection{Limits and biases of the telemetry methods
used}\label{limits-and-biases-of-the-telemetry-methods-used}}

In the following section, shortcomings of the telemetry technologies and
the resulting datasets will be discussed.

\hypertarget{sec-disc-tagging-effects}{%
\subsubsection{Tag recoveries and possible tagging
effects}\label{sec-disc-tagging-effects}}

Of 30 deployed ADST, 9 were recovered so far, with data availability of
8 ADST at the time of writing. Out of those, 6 \emph{M. asterias} died
within the first month post-release. There are numerous factors that
potentially contribute to a negative effect of the tagging procedure.
For instance, the choice of tag to implant must be made considering the
animal's body mass. Generally, it is advised that the tag should not
exceed 10 \% of the individuals's body mass \citep{wagner_2011}. In the
present study, weight was recorded for 17 out of 30 individuals tagged.
The lightest individual of those sharks weighed 900 g, resulting in a
tag-to-bodymass ratio of 1.4 \% and thereby being well under the
proposed threshold of 10 \%. In addition, \citet{smukall_2019} report
the non-lethal recovery of an acoustic tag 13 years post-release,
implanted into a female Lemon shark (\emph{Negaprion brevirostris}) at
119 cm TL. Thus, it does not seem likely that the tag's weight had a
severe negative influence on the survival of the tagged individuals.
Other possible factors include insufficient asepsis of surgery tools and
the tags itself, or negative physiological responses due to insufficient
handling \citep{rub_2014}. This includes too much time outside of the
water, injuries from the fishing method (hook and line, in this case),
or the tag implantation taking too long.

The cause of death for the the 6 \emph{M. asterias} individuals cannot
be identified at this point. Death due to predation by an endothermic
mammal predator such as the grey seal can be excluded, because the
temperature did not substantially rise (to around 38°C,
\citet{austin_2006}) prior to the animals' deaths (all raw depth an
temperature logs are shown in Figure~\ref{fig-dsttempannex}. Generally,
the temperature logs of the short term DST appear erroneous, note for
example that the temperature remains 21.2°C for tag 319 at all time
between August 4 and 12, 2018. The possibility of predation by an
ectothermic predator cannot be ruled out at this point, since
temperature cannot provide any relevant insights but instead the
depthlog would show extraordinary vertical movement \citep[see][ for
examples of predation by endothermic versus ectothermic
predators]{seitz_2019}. The possibility of predation by an ectothermic
animal such as the blue shark \emph{Prionace glauca} could be a next
investigative step.

\hypertarget{limits-of-the-datasets}{%
\subsubsection{Limits of the datasets}\label{limits-of-the-datasets}}

Essentially, two type of dataset were used for study: Firstly, the
acoustic detections of the tagged sharks, and secondly, the DST logs
from recovered tags.

The 10,940 acoustic detections of 18 \emph{M. asterias} individuals over
a period of roughly two years are already a relevant dataset to start
with, but more tagging effort and thus more acoustic detections are
needed to gain further insights into the presence of the species in the
Scheldt Estuary. For comparison, \citet{hereu_2023} tagged 33 European
seabass (\emph{Dicentrarchus labrax}), a seasonally migratory fish
\citep{pawson_2007}, and had 493.817 acoustic detections over roughly 2
years in an acoustic receiver network comprised of around 100 receivers
\citep{aspillaga_2017}. The substantial difference in acoustic
detections (approximately factor 50) indicates a difference in
explanatory power between the dataset of \emph{M. asterias} and the
dataset used in \citet{hereu_2023}.

Moreover, the gate-like receiver setup in the Western Scheldt is not
ideal to study horizontal distribution but rather serves to detect
animals that enter or leave the river Scheldt \citep{reubens_2019}. If
possible, additional receivers could be placed along the coast between
Dishoek and Zoetelande to further resolve the seasonal presence of adult
females in the area and potentially gain information on pupping grounds.

The availability of only long term DST logs (that is, a log with more
than a year of data) poses the greatest limit on the archival dataset.
More logs showing the annual migration should be analysed in the future
to assess whether the sex-specific patterns described in this study are
due to variation between individuals or due to the animal's sex. The
short term DST logs were not further investigated in the present study
since altered behaviour resulting from the tagging procedure could not
be excluded. However, investigating the short term DST logs could be
valuable to assess in the future.

An additional restriction of the archival data is the limit of the depth
sensor of 68 m. Since the female shark swam deeper than that limit
between October and December, 2018, limited inferences can be made from
the female's depthlog during that period and the current results from
the geolocation modelling are therefor imprecise during that time
period.

\hypertarget{reflection-of-the-chosen-analyses}{%
\subsection{Reflection of the chosen
analyses}\label{reflection-of-the-chosen-analyses}}

Since this work did not involve data collection but instead was purely
focused on the analysis of the acoustic detection and data storage tag
datasets, the chosen analyses will be reflected upon in the following.

\hypertarget{acoustic-detections-1}{%
\subsubsection{Acoustic detections}\label{acoustic-detections-1}}

For this work, the acoustic detection data were analysed in an
exploratory way. Heatmaps with the number of detections per receiver
station, individual and month proved to be a suitable method of
investigation. In the future, the detections of single females during
summer could be looked at in detail to assess if there are differences
in presence around certain receivers that could indicate pupping. A
useful piece of information for such investigations would be to assess
the reproductive status of tagged females (i.e., if the individual is
pregnant or not) upon release, as argued in
Section~\ref{sec-disc-seasonalpresencefemales}.

\hypertarget{autocorrelation}{%
\paragraph{Autocorrelation}\label{autocorrelation}}

Autocorrelation analysis allows for a simplistic description of scales
of patterns found in time series. It is a helpful exploratory tool to
get an overview of the dataset \citep{dray_2010} and it moreover can
provide the base for the detection of behavioural switches
\citep{gurarie_2016}. While autocorrelation might not be the most
resultful exploratory analysis it can be a helpful auxiliary tool to
explore the depth time series data and potentially confirm identified
periods with occurring behavioural switches.

\hypertarget{wavelet-analysis-1}{%
\subsubsection{Wavelet Analysis}\label{wavelet-analysis-1}}

Conducting a continuous wavelet transform (CWT) and thereby resolving
prevalent periodicities in the time domain proved to be a main analysis
tool in this study. Furthermore, performing CWT both on the raw depthlog
and the daily summary statistics allowed for different scales of
periodicities and was able to provide further proof of patterns that
could already been seen in the daily summary statistics in
Figure~\ref{fig-dstsum308-1} and Figure~\ref{fig-dstsum321-1}. Wavelet
analysis is no ubiquitous analysis tool in movement ecology but has been
employed in the past \citep{wittemyer_2008, zhang_2020}. CWT was
preferred over Fourier Transformation (FT) due to its ability to resolve
periodicities in the time domain. While FT can also be performed on
subsets of the depthlog, this manual separation introduces a bias and
within that subset, the periodicities are not resolved in the time
domain. The resolution in the time domain of CWT comes at the cost of a
lower resolution both in the time and frequency domain. Thus, FT
provides a first step of spectral analysis and can give insights into
generally existing patterns of periodicity (as shown in \textbf{Annex
A}), but CWT allows for unbiased assessment of change of prevalent
periodicities in time of time series data. In the wavelet scalograms on
the hourly scale, only periods in winter were significantly present and
are shown as fully opaque in the plots
(Figure~\ref{fig-waveletresults308-1} and
Figure~\ref{fig-waveletresults321-1}). In summer, periodicities of 12 h
for the female shark and 24 h of the male shark have a high prevalence
(indicated by a high power value and an orange/red colour in the
scalogram), albeit these are not significant. This is thought to be due
to the difference in magnitude of the periodicity during summer compared
to the winter period, for example, where overall vertical activity is
much higher.

\hypertarget{conclusion}{%
\section{Conclusion}\label{conclusion}}

To conclude, the findings related to the research questions stated in
the introduction (see Section~\ref{sec-intro-aims}) will be presented.

This study presents the first in-depth investigation of both the
seasonality of vertical behaviour throughout the annual movement cycle
and the presence of \emph{M. asterias} in the Western Scheldt. Adding to
already existing knowledge, the seasonal presence of mostly adult female
starry smooth-hounds around the Western Scheldt could be confirmed,
proving the importance of the area as a summer habitat. Males
potentially are less present in the Western Scheldt, but more present in
the BPNS than females. Moreover, the high return rates of individuals
(both males and females) into the Western Scheldt and the BPNS provide
further evidence for philopatric behaviour of the species (Research
Question 1, 2 and 3).

Although males and females most likely overwinter in different areas
\citep{breve_2016, griffiths_2020}, their seasonal vertical movement
behaviour appears to be similar (involving generally less vertical
activity during summer and higher vertical activity during winter),
according to the two DST tags that were subject of this study. During
summers, the female primarily showed resting behaviour while the male
exhibited both resting and feeding behaviour, indicating sex-specificity
in behavioural states on a within-season scale. During winter, the
female's (and to a lesser extent, the male's) vertical activity is
potentially influenced by the moon phase which might be linked to the
intensity of tidal currents (Research Question 4 and 5).

Thus, this study could gain new insights into the seasonal distribution
and potential sex-specificity of horizontal and vertical movement
behaviour of \emph{Mustelus asterias}. There still is, however, a
substantial knowledge gap about the effects of sex and life stage on the
species' distribution \citep{griffiths_2020} and to be able to establish
a successful species management plan in the future, further studies are
needed
\citep{mccullyphillips_2015, breve_2016, breve_2020, griffiths_2020}.

\hypertarget{outlook}{%
\section{Outlook}\label{outlook}}

\hypertarget{tagging-studies-involving-m.-asterias}{%
\subsection{\texorpdfstring{Tagging studies involving \emph{M.
asterias}}{Tagging studies involving M. asterias}}\label{tagging-studies-involving-m.-asterias}}

As discussed in Section~\ref{sec-disc-seasonalpresencefemales}, an
insightful addition to future tagging studies involving adult \emph{M.
asterias} would be to assess the females' reproductive stage upon
sampling. This could be done either through ultrasonography or the use
of sex steroid hormones
\citep{awruch_2014, smukall_2019, anderson_2018, fujinami_2020, fujinami_2021}.
A new tagging technology was launched recently: The Birth Alert Tag
\citep[BAT,][]{sulikowski_2023}, which is, however, only advised to be
used on ``large sharks''. If smaller BATs are developed in the future,
this might be a suitable tag to use on \emph{M. asterias}. In addition,
the placement of an acoustic receiver off the coast of Zoetelande might
give further insights into possible seasonal aggregation of \emph{M.
asterias} in summer, as described by fishers\footnote{Verhelst,
  Pieterjan (\emph{personal communication}), April 25, 2023.}. If
possible, DST with a wider depth range should be utilized for future
studies, since evidence of \emph{M. asterias} swimming in deeper waters
(\textgreater{} 100 m depth) than previously thought is increasing
\citep{ices_2019, griffiths_2020}. Generally, the effect of the
implantation of tags on the sharks should be investigated. Since female
\emph{M. asterias} have been previously reported to abort and expel
embryos when caught \citep{farrell_2010a}, special attention should be
drawn to the tagging effect on pregnant females, especially when
intending to study their pupping behaviour. \footnote{Verhelst,
  Pieterjan (\emph{personal communication}), April 25, 2023.}: Verhelst,
Pieterjan (\emph{personal communication}), April 25, 2023.

\hypertarget{analysis-of-data-storage-tag-logs}{%
\subsection{Analysis of data storage tag
logs}\label{analysis-of-data-storage-tag-logs}}

The adequate analysis of DST logs and the development of well-performing
geolocation models remains a challenge. While the input of behavioural
states (i.e., high or low vertical activity, or residential versus
migrating behaviour) would provide improved input for a geolocation
model, segmenting the depthlog in different behavioural states is not
trivial.

As a first step to improve the geolocation model without adding
behavioural states could be the addition of a simple boundary condition
that reflects the demersal lifestyle of \emph{M. asterias}. This
boundary condition entails that the maximum depth logged by the tag each
day (given that the resolution of the geolocation model is in the unit
of days) is equal to the maximum depth of the bottom at the location of
the shark. This would prevent model outputs that estimate the
individual's position on a day at locations with substantially deeper
bottom depth than the shark's depthlog has recorded that day.

Furthermore, wavelet analysis proved to be a helpful tool for assessing
the seasonal changes in spectral composition of the depth signal for
both individuals. This technique is suggested to be further taken into
account in the future, for instance regarding the segmentation of the
depthlog into different behavioural states \citep[as already shown
by][]{soleymani_2017}. Since the implementation of simple CWT in this
study is still an exploratory technique it should be investigated how it
could be quantified in the future. If different behavioural states, for
example resting, feeding and migrating can be linked to specific
spectral signatures (i.e., a 12 h period for resting, a 24 h period for
feeding and high prevalence of periods from 4h to 64 h for migrating),
then wavelet coherence could be a next analysis step. This analysis
assesses the similarity of two wavelet scalograms \citep{grinsted_2004}.
Conducting wavelet coherence between the raw depthlogs of the sharks and
characteristic spectral signatures of different behaviours could
quantify the presence of different behavioural states in the future.
Several methods to quantify behavioural states exist already
\citep{pedersen_2008, heerah_2017} but there is not one generic method
that works for all species equally. Thus, it would be valuable to test
the method suggested here and compare it to outputs from already
existing methods. This would be a next step towards improving the
geolocation modelling for \emph{M. asterias} and consequently, our
knowledge and understanding of the species.

\hypertarget{availability-of-data-and-materials}{%
\section*{Availability of Data and
Materials}\label{availability-of-data-and-materials}}
\addcontentsline{toc}{section}{Availability of Data and Materials}

The dataset(s) supporting the conclusions of this article is(are)
available in the {[}repository name{]} repository, {[}unique persistent
identifier and hyperlink to dataset(s) in http:// format{]}.

The dataset(s) supporting the conclusions of this article is(are)
included within the article (and its additional file(s)).

The metadata that support the raw data for this study can be found in
Annex B.

\hypertarget{references}{%
\section*{References}\label{references}}
\addcontentsline{toc}{section}{References}

\hypertarget{acknowledgements}{%
\section*{Acknowledgements}\label{acknowledgements}}
\addcontentsline{toc}{section}{Acknowledgements}

This work was supported by the Research Foundation Flanders (FWO) as
part of the Belgian contribution to LifeWatch. I am indebted to everyone
involved in the tagging of the 30 starry smooth-hounds in 2018 and 2019
without whom this study would not have been possible. My gratitude
extends to everyone involved in the maintenance of the acoustic receiver
network involved in this study. Furthermore, I appreciate the effort by
everyone involved in the return of the tags, enabling me to study the
annual vertical movement of the starry smooth-hounds in the first place.
I am thankful to the Flemish Marine Institute (VLIZ) for having granted
me the opportunity to conduct my master thesis with them, and the
IMBRSea coordinators for their continuous effort to make this study
programme happen.

I wish to express my heartfelt thanks to my promotors Jan Reubens and
Niels Brevé, and my supervisor Carlota Muñiz. Thank you Jan, for
providing me with constructive and valuable feedback, thank you Niels
for your invaluable input on the starry smooth-hound, and thank you
Carlota for guiding me and looking after me when I was a bit stuck.
Furthermore, I am grateful for having been able to go out into the
Belgian Part of the North Sea for receiver maintenance work, this
experience gave me valuable insights into the practicalities behind the
receiver network. My sincere appreciation goes to Pieterjan Verhelst,
for taking me along the Twaite Shad tagging campaign and the car
conversations about fish migrations and fishers's knowledge. Thank you
to Jolien Goossens for carrying out the geolocation modelling of the
returned tags which provided an important base for me to follow up on.

I furthermore wish to thank my IMBRSea study mates for productive
co-working sessions and emotional support throughout the thesis. My
thesis would not have been the same without all the warmhearted
colleagues at VLIZ, thank you all for this extraordinary time in Gent
and Ostend. My eternal gratitude goes to the people that I deeply care
for, especially Silvi and Tobi, for emotional support and understanding
during this intense period. And lastly, endless thanks to my parents for
being there and supporting me no matter what.

\hypertarget{annex}{%
\section*{Annex}\label{annex}}
\addcontentsline{toc}{section}{Annex}

\hypertarget{annex-a.-dst-data}{%
\subsection*{\texorpdfstring{\textbf{Annex A.} DST
data}{Annex A. DST data}}\label{annex-a.-dst-data}}
\addcontentsline{toc}{subsection}{\textbf{Annex A.} DST data}

\hypertarget{a.1-raw-depth-and-temperature-logs}{%
\subsubsection*{\texorpdfstring{\textbf{A.1} Raw depth and temperature
logs}{A.1 Raw depth and temperature logs}}\label{a.1-raw-depth-and-temperature-logs}}
\addcontentsline{toc}{subsubsection}{\textbf{A.1} Raw depth and
temperature logs}

Figure~\ref{fig-dsttempannex} shows the raw temperature logs of tags 308
and 321. Tag 308 experienced the maximum temperature of 22.9°C on August
06, 2018 and its minimum temperature of 8.65°C on March 23, 2019. Tag
321 experienced its maximum temperature of 22.1°C on August 07, 2018 and
its minimum temperature of 5.98°C on February 04, 2019.

\begin{figure}

\begin{minipage}[t]{\linewidth}

{\centering 

\raisebox{-\height}{

\includegraphics{thesis_manuscript_files/figure-pdf/fig-dsttempannex-1.pdf}

}

}

\subcaption{\label{fig-dsttempannex-1}Tag 308.}
\end{minipage}%
\newline
\begin{minipage}[t]{\linewidth}

{\centering 

\raisebox{-\height}{

\includegraphics{thesis_manuscript_files/figure-pdf/fig-dsttempannex-2.pdf}

}

}

\subcaption{\label{fig-dsttempannex-2}Tag 321.}
\end{minipage}%

\caption{\label{fig-dsttempannex}Temperature logs from the long term
DST.}

\end{figure}

Figure~\ref{fig-dstannex} displays the raw depth and temperature logs
from the short term DST.

\begin{figure}

\begin{minipage}[t]{0.50\linewidth}

{\centering 

\raisebox{-\height}{

\includegraphics{thesis_manuscript_files/figure-pdf/fig-dstannex-1.pdf}

}

}

\subcaption{\label{fig-dstannex-1}Tag 295. This male was tagged on July
19, 2018.}
\end{minipage}%
%
\begin{minipage}[t]{0.50\linewidth}

{\centering 

\raisebox{-\height}{

\includegraphics{thesis_manuscript_files/figure-pdf/fig-dstannex-2.pdf}

}

}

\subcaption{\label{fig-dstannex-2}Tag 304. This female was tagged on
July 11, 2019.}
\end{minipage}%
\newline
\begin{minipage}[t]{0.50\linewidth}

{\centering 

\raisebox{-\height}{

\includegraphics{thesis_manuscript_files/figure-pdf/fig-dstannex-3.pdf}

}

}

\subcaption{\label{fig-dstannex-3}Tag 310. This male was tagged on July
11, 2019.}
\end{minipage}%
%
\begin{minipage}[t]{0.50\linewidth}

{\centering 

\raisebox{-\height}{

\includegraphics{thesis_manuscript_files/figure-pdf/fig-dstannex-4.pdf}

}

}

\subcaption{\label{fig-dstannex-4}Tag 312. This female was tagged on
July 12, 2019.}
\end{minipage}%
\newline
\begin{minipage}[t]{0.50\linewidth}

{\centering 

\raisebox{-\height}{

\includegraphics{thesis_manuscript_files/figure-pdf/fig-dstannex-5.pdf}

}

}

\subcaption{\label{fig-dstannex-5}Tag 319. This female was tagged on
July 19, 2018.}
\end{minipage}%
%
\begin{minipage}[t]{0.50\linewidth}

{\centering 

\raisebox{-\height}{

\includegraphics{thesis_manuscript_files/figure-pdf/fig-dstannex-6.pdf}

}

}

\subcaption{\label{fig-dstannex-6}Tag 322. This female was tagged on
July 19, 2018.}
\end{minipage}%

\caption{\label{fig-dstannex}Temperature and depth logs from the short
term DST.}

\end{figure}

\hypertarget{a.2-fast-fourier-transformation}{%
\subsubsection*{\texorpdfstring{\textbf{A.2} Fast Fourier
Transformation}{A.2 Fast Fourier Transformation}}\label{a.2-fast-fourier-transformation}}
\addcontentsline{toc}{subsubsection}{\textbf{A.2} Fast Fourier
Transformation}

The resulting periodograms from the Fast Fourier Transform for tags 308
(\emph{f}) and 321 (\emph{m}) are depicted in Figure~\ref{fig-fft}.

Figure~\ref{fig-fft-1} shows spectral density peaks at about 12 hours
(highest), 24 hours, and smaller peaks at 23, 26, 6 and 8 hours. Tag 321
has the highest spectral density at 24 hours, then 12 hours, and small
peaks at 6, 23, 25 and 25 hours, see Figure~\ref{fig-fft-2}.

\begin{figure}

\begin{minipage}[t]{\linewidth}

{\centering 

\raisebox{-\height}{

\includegraphics{thesis_manuscript_files/figure-pdf/fig-fft-1.pdf}

}

}

\subcaption{\label{fig-fft-1}Tag 308 (f), 366 days of liberty.}
\end{minipage}%
\newline
\begin{minipage}[t]{\linewidth}

{\centering 

\raisebox{-\height}{

\includegraphics{thesis_manuscript_files/figure-pdf/fig-fft-2.pdf}

}

}

\subcaption{\label{fig-fft-2}Tag 321 (m), 485 days of liberty.}
\end{minipage}%

\caption{\label{fig-fft}Periodograms from the full depthlog of recovered
data storage tags.}

\end{figure}

\hypertarget{annex-b.-metadata}{%
\subsection*{\texorpdfstring{\textbf{Annex B.}
Metadata}{Annex B. Metadata}}\label{annex-b.-metadata}}
\addcontentsline{toc}{subsection}{\textbf{Annex B.} Metadata}

Four sets of tabular data form the basis of this study. These can be
found in the additional files of this study. A brief description of the
dataset and its variables is given below. Where available, the variable
descriptions were obtained from the ETN database
(https://www.lifewatch.be/etn/).

\hypertarget{b.1-information-on-tagged-individuals}{%
\subsubsection*{\texorpdfstring{\textbf{B.1} Information on tagged
individuals}{B.1 Information on tagged individuals}}\label{b.1-information-on-tagged-individuals}}
\addcontentsline{toc}{subsubsection}{\textbf{B.1} Information on tagged
individuals}

The dataset includes information on the 30 \emph{M. asterias}
individuals that were subject of this study (filename:
\texttt{tagged\_individuals\_raw\_data.csv}). The metadata linked to
this dataset is displayed in Table~\ref{tbl-taggedindividualsmetadata}.

\hypertarget{tbl-taggedindividualsmetadata}{}
\begin{longtable}[]{@{}
  >{\raggedright\arraybackslash}p{(\columnwidth - 2\tabcolsep) * \real{0.2000}}
  >{\raggedright\arraybackslash}p{(\columnwidth - 2\tabcolsep) * \real{0.8000}}@{}}
\caption{\label{tbl-taggedindividualsmetadata}Metadata supporting the
dataset \texttt{tagged\_individuals\_raw\_data.csv}.}\tabularnewline
\toprule\noalign{}
\begin{minipage}[b]{\linewidth}\raggedright
variable
\end{minipage} & \begin{minipage}[b]{\linewidth}\raggedright
explanation
\end{minipage} \\
\midrule\noalign{}
\endfirsthead
\toprule\noalign{}
\begin{minipage}[b]{\linewidth}\raggedright
variable
\end{minipage} & \begin{minipage}[b]{\linewidth}\raggedright
explanation
\end{minipage} \\
\midrule\noalign{}
\endhead
\bottomrule\noalign{}
\endlastfoot
release\_date\_time & Date and time of the release of the animal, in
YYYY-MM-DD HH:MM:SS format and UTC time. \\
tag\_serial\_number & The serial number that is unique to the Acoustic
Data Storage Tag. \\
release\_latitude & Longitude of release location, in decimal degress.
Note: in the western hemisphere all longitudes must be negative. \\
release\_longitude & Longitude of release location, in decimal degress.
Note: in the western hemisphere all longitudes must be negative. \\
sex & Sex of the animal. f = female, m = male. \\
weight & Bodymass of the animal carrying the tag. In Kilogram. \\
length1 & Total length of animal carrying the tag. In Metres. \\
recapture\_date\_time & If applicable, date of the recapture of the
tagged animal. In YYYY-MM-DD format, in UTC time. \\
\end{longtable}

\hypertarget{b.2-acoustic-detections}{%
\subsubsection*{\texorpdfstring{\textbf{B.2} Acoustic
detections}{B.2 Acoustic detections}}\label{b.2-acoustic-detections}}
\addcontentsline{toc}{subsubsection}{\textbf{B.2} Acoustic detections}

The dataset includes the acoustic detections of the 30 \emph{M.
asterias} individuals by the acoustic receivers that are part of the
Permanent Belgian Acoustic Receiver Network (filename:
\texttt{acoustic\_detections\_raw\_data.csv}). The metadata linked to
this dataset is displayed in Table~\ref{tbl-detectionsmetadata}.

\hypertarget{tbl-detectionsmetadata}{}
\begin{longtable}[]{@{}
  >{\raggedright\arraybackslash}p{(\columnwidth - 2\tabcolsep) * \real{0.2000}}
  >{\raggedright\arraybackslash}p{(\columnwidth - 2\tabcolsep) * \real{0.8000}}@{}}
\caption{\label{tbl-detectionsmetadata}Metadata supporting the dataset
\texttt{acoustic\_detections\_raw\_data.csv}.}\tabularnewline
\toprule\noalign{}
\begin{minipage}[b]{\linewidth}\raggedright
variable
\end{minipage} & \begin{minipage}[b]{\linewidth}\raggedright
explanation
\end{minipage} \\
\midrule\noalign{}
\endfirsthead
\toprule\noalign{}
\begin{minipage}[b]{\linewidth}\raggedright
variable
\end{minipage} & \begin{minipage}[b]{\linewidth}\raggedright
explanation
\end{minipage} \\
\midrule\noalign{}
\endhead
\bottomrule\noalign{}
\endlastfoot
detection\_id & Unique identifier of the detection event. \\
date\_time & Date and time in YYYY-MM-DD HH:MM:SS format. In UTC
time. \\
tag\_serial\_number & The serial number that is unique to the Acoustic
Data Storage Tag. \\
scientific\_name & Scientific name of the animal that carries the
tag. \\
station\_name & Name of the station where the deployment of the receiver
takes place. Related to a specific latitude and longitude. \\
deploy\_latitude & Latitude of the actual deployment location, in
decimal degrees. Note: in the southern hemisphere all latitudes must be
negative. \\
deploy\_longitude & Longitude of the actual deployment location, in
decimal degrees. Note: in the western hemisphere all longitudes must be
negative. \\
parameter & Value of one sensor transmitted to the acoustic receiver at
the time of detection. \\
sensor\_unit & Unit of the sensor at stake. \\
sensor\_type & Type of tag sensor. Predefined options: pressure,
temperature, acceleration. \\
acoustic\_tag\_id & Unique identifier of each sensor within the acoustic
tag. One ID for the pressure sensor, and one ID for the temperature
sensor per ADST. \\
\end{longtable}

\hypertarget{b.3-data-storage-tag-logs}{%
\subsubsection*{\texorpdfstring{\textbf{B.3} Data Storage Tag
logs}{B.3 Data Storage Tag logs}}\label{b.3-data-storage-tag-logs}}
\addcontentsline{toc}{subsubsection}{\textbf{B.3} Data Storage Tag logs}

The dataset includes the depth and temperature logs of all recovered
data storage tags (filename: \texttt{DST\_logs\_raw\_data.csv}). The
metadata linked to this dataset is displayed in
Table~\ref{tbl-dstlogmetadata}.

\hypertarget{tbl-dstlogmetadata}{}
\begin{longtable}[]{@{}
  >{\raggedright\arraybackslash}p{(\columnwidth - 2\tabcolsep) * \real{0.2000}}
  >{\raggedright\arraybackslash}p{(\columnwidth - 2\tabcolsep) * \real{0.8000}}@{}}
\caption{\label{tbl-dstlogmetadata}Metadata supporting the dataset
\texttt{DST\_logs\_raw\_data.csv}.}\tabularnewline
\toprule\noalign{}
\begin{minipage}[b]{\linewidth}\raggedright
variable
\end{minipage} & \begin{minipage}[b]{\linewidth}\raggedright
explanation
\end{minipage} \\
\midrule\noalign{}
\endfirsthead
\toprule\noalign{}
\begin{minipage}[b]{\linewidth}\raggedright
variable
\end{minipage} & \begin{minipage}[b]{\linewidth}\raggedright
explanation
\end{minipage} \\
\midrule\noalign{}
\endhead
\bottomrule\noalign{}
\endlastfoot
date\_time & Date and time in YYYY-MM-DD HH:MM:SS format. In UTC
time. \\
tag\_serial\_number & The serial number that is unique to the Acoustic
Data Storage Tag. \\
depth\_m & The recorded depth of the depth sensor in Metres. \\
temp\_c & The recorded temperature of the temperature sensor in degrees
Celcius. \\
\end{longtable}

\hypertarget{b.4-geolocation-modelling-output}{%
\subsubsection*{\texorpdfstring{\textbf{B.4} Geolocation modelling
output}{B.4 Geolocation modelling output}}\label{b.4-geolocation-modelling-output}}
\addcontentsline{toc}{subsubsection}{\textbf{B.4} Geolocation modelling
output}

The dataset includes the most probable tracks calculated by the
geolocation model \citep[the reader is referred to][ for further
explanation of the model, filename:
\texttt{geolocation\_output\_raw\_data.csv}]{goossens_2023}. The
metadata linked to this dataset is displayed in
Table~\ref{tbl-geolocationmetadata}. The geolocation modelling was
carried out by Jolien Goossens (ORCID:
\href{https://orcid.org/0000-0002-0853-9153}{0000-0002-0853-9153}).

\hypertarget{tbl-geolocationmetadata}{}
\begin{longtable}[]{@{}
  >{\raggedright\arraybackslash}p{(\columnwidth - 2\tabcolsep) * \real{0.2000}}
  >{\raggedright\arraybackslash}p{(\columnwidth - 2\tabcolsep) * \real{0.8000}}@{}}
\caption{\label{tbl-geolocationmetadata}Metadata supporting the dataset
\texttt{geolocation\_output\_raw\_data.csv}.}\tabularnewline
\toprule\noalign{}
\begin{minipage}[b]{\linewidth}\raggedright
variable
\end{minipage} & \begin{minipage}[b]{\linewidth}\raggedright
explanation
\end{minipage} \\
\midrule\noalign{}
\endfirsthead
\toprule\noalign{}
\begin{minipage}[b]{\linewidth}\raggedright
variable
\end{minipage} & \begin{minipage}[b]{\linewidth}\raggedright
explanation
\end{minipage} \\
\midrule\noalign{}
\endhead
\bottomrule\noalign{}
\endlastfoot
date\_time & Date and time in YYYY-MM-DD format. In UTC time. \\
tag\_serial\_number & The serial number that is unique to the Acoustic
Data Storage Tag. \\
detection\_latitude & Latitude that was calculated as the `most probable
track' in the geolocation model. See Goossens et al., (2023) and Woillez
et al., (2016) for details. \\
detection\_longitude & Longitude that was calculated as the `most
probable track' in the geolocation model. See Goossens et al., (2023)
and Woillez et al., (2016) for details. \\
\end{longtable}


  \bibliography{bibliography.bib}


\end{document}
